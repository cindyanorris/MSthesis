% \subsubsection{Contract FirstPastThePost}

% The first-past-the-post \solt{contract} is the simplest election \solt{contract}
% implementation.

\begin{solidity}[contract FirstPastThePost]
contract FirstPastThePost is Owned, Guard {
  uint public creationTime = now;

  Choice[] public choices;
  mapping(address => bool) public voted;
  Choice public winner;

  enum Phase { Configuration, Frozen, Vote, Tally }

  struct PhaseProperties {
    Phase phase;
    uint startTime;
    uint endTime;
    Phase nextPhase;
  }

  mapping (Phase => PhaseProperties) phases;
  Phase public phase = Phase.Configuration;

  constructor () {
    phases[Phase.Configuration] = PhaseProperties({
      phase: Phase.Configuration,
      start_time: now,
      end_time: 0,
      next_phase: Phase.Configuration
    });
  }

  // This is the current phase.
  uint public freezeTime;

  // Start and end time for the election.
  uint public voteStartTime;
  uint public voteEndTime;

  // This is a type for a single proposal.
  struct Choice {
    // Short name (up to 32 bytes).
    bytes32 name;

    // TODO: choice description?
    // bytes32 description;

    // Number of accumulated votes.
    uint40 voteCount;
  }

  // Restrict calls to _account.
  modifier restrictTo(address _account) {
    require(msg.sender == _account);
    _;
  }

  // Check current stage.
  modifier atPhase(Phase _phase) {
    require(phase == _phase);
    _;
  }

  // This modifier goes to the next phase after the function is done.
  modifier transitionNextPhase() {
    _;
    nextPhase();
  }

  // Move into next phase.
  function nextPhase() internal {
    phase = Phase(uint(phase) + 1);
  }

  // Perform timed transitions. Be sure to mention this modifier first,
  // otherwise the guards will not take the new stage into account.
  modifier timedTransitions() {
    if (phase == Phase.Frozen && now >= voteStartTime)
      nextPhase();
    if (phase == Phase.Vote && now >= voteEndTime)
      nextPhase();
    // The other stages transition by transaction
    _;
  }

  /* Election Functions */
  // Freeze the configuration.
  function freeze() restrictTo(owner) atPhase(Phase.Configuration) transitionNextPhase {
    require(voteStartTime > now);
    require(choices.length > 1);
    freezeTime = now;
    delete owner;
  }

  // Cast a ballot.
  function vote(uint8 choice) timedTransitions atPhase(Phase.Vote) {
    // Each address can only vote once.
    require(!voted[msg.sender]);
    voted[msg.sender] = true;

    choices[choice].voteCount += 1;
  }

  // Tally ballots.
  function tally() timedTransitions atPhase(Phase.Tally) {
    winner = choices[0];
    for (uint8 i = 0; i < choices.length; i++) {
      if (choices[i].voteCount > winner.voteCount)
        winner = choices[i];
    }
  }

  /* Configuration Functions */
  constructor () {}

  function setVoteStartTime(uint _voteStartTime) atPhase(Phase.Configuration) restrictTo(owner) {
    voteStartTime = _voteStartTime;
  }

  function setVoteEndTime(uint _voteEndTime) atPhase(Phase.Configuration) restrictTo(owner) {
    voteEndTime = _voteEndTime;
  }

  function addChoices(bytes32[] _choices) atPhase(Phase.Configuration) restrictTo(owner) {
    for (uint i = 0; i < _choices.length; i++) {
      choices.push(Choice({
        name: _choices[i],
        voteCount: 0
      }));
    }
  }

  function addChoice(bytes32 _name) atPhase(Phase.Configuration) restrictTo(owner) {
    choices.push(Choice({
      name: _name,
      voteCount: 0
    }));
  }
}
\end{solidity}

% \begin{state}
%   \begin{public}
%     \item \sol{AccessControlList choices} maintains the \solt{address} of an ACL
%       implementation, a \solt{contract} which implements \sol{interface
%       AccessControlList}.
%
%     \item \sol{mapping (bytes4 => uint8) requiredResourcePermissions} maintains a
%       mapping of \solt{function}[s], \sol{bytes4}, to required
%       \emph{permission}, \sol{uint8}.
%   \end{public}
% \end{state}
%
% \begin{code}
%   \begin{constructor}
%     \item \sol{constructor (bool _createACL)}, upon creation and initialization
%       of this \solt{contract} the \solt{constructor} can deploy a
%       \solt{contract}, \sol{contract BasicACL}, if the \solt{bool}, \sol{bool
%       _createACL}, is set to \solt{true}.
%
%       \begin{parameters}
%       \item \sol{bool _createACL}, set to \solt{true} to deploy an ACL
%         implementation, \sol{contract BasicACL}, in addition to this
%         \solt{contract}; otherwise \solt{false}.
%       \end{parameters}
%   \end{constructor}
%
%   \begin{functions}
%     \item \sol{function setAcl (AccessControlList _acl)}, updates the
%       \solt{contract} reference which is responsible for managing ACL requests.
%
%       \begin{modifiers}
%         \item \sol{modifier restrictTo (owner)}, restricts access to the
%           \solt{function} such that \emph{only} the current \solt{owner} of the
%           \sol{contract} can call it.
%       \end{modifiers}
%
%       \begin{parameters}
%         \item \sol{AccessControlList _acl}, the ACL implementation which access
%           control responsibilities are to be delegated to.
%       \end{parameters}
%
%       \begin{returns}
%         \item \sol{bool _success}, resolves to \solt{true} if the operation was
%           successful, otherwise \solt{false}.
%       \end{returns}
%
%
%     \item \sol{function setRequiredResourcePermission (bytes4 _resource, uint8 _permission)},
%       updates the mapping representing the \emph{permission} required to access
%       some \emph{resource}, \solt{function} signature.
%
%       \begin{modifiers}
%         \item \sol{modifier restrictTo (owner)}, restricts access to the
%           \solt{function} such that \emph{only} the current \solt{owner} of the
%           \sol{contract} can call it.
%       \end{modifiers}
%
%       \begin{parameters}
%         \item \sol{bytes4 _resource}, the \emph{resource}, \solt{function}
%           signature, which the \emph{permission} is to be modified for.
%
%         \item \sol{uint8 _permission}, a \solt{uint8} value representing the
%           \emph{permission-bit} required for the \emph{resource}.
%       \end{parameters}
%
%       \begin{returns}
%         \item \sol{bool _success}, resolves to \solt{true} if the operation was
%           successful, otherwise \solt{false}.
%       \end{returns}
%
%
%     \item \sol{function canCall (address _subject, bytes4 _resource)}, evaluates
%       whether a \emph{subject} can access to some \emph{resource} by validating
%       with the ACL implementation that the \emph{subject} has the required
%       \emph{resource} \emph{permission}.
%
%       \begin{parameters}
%         \item \sol{address _subject}, the \emph{subject}, Ethereum account,
%           whose \emph{permissions} are being evaluated.
%
%         \item \sol{bytes4 _resource}, the \emph{resource}, \solt{contract
%           function}, which the \emph{subject's} \emph{permissions} are being
%           evaluated against.
%       \end{parameters}
%
%       \begin{returns}
%         \item \sol{bool _canCall}, resolves to \solt{true} if the
%           \emph{subject}, \sol{address _subject}, is permitted to access the
%           \emph{resource}, \sol{bytes4 _resource}, otherwise \solt{false}.
%       \end{returns}
%
%
%     \item
%       \sol{function hasPermission (address _subject, bytes4 _resource, uint8 _permission)}, \\
%       \sol{function getPermissions (address _subject, bytes4 _resource)}, \\
%       \sol{function setPermissions (address _subject, bytes4 _resource, uint256 _permissions)}, \\
%       \sol{function grantPermission (address _subject, bytes4 _resource, uint8 _permission)}, \\
%       \sol{function revokePermission (address _subject, bytes4 _resource, uint8 _permission)}, \\
%       all ACL responsibilities originating from \sol{interface AccessControlList}
%       are delegated to the provided ACL implementation stored in
%       \sol{AccessControlList acl}.
%
%       \begin{modifiers}
%         \item \sol{modifier restrictTo (owner)}, restricts access to the
%           \solt{function} such that \emph{only} the current \solt{owner} of the
%           \solt{contract} can call this \solt{function}; this applies to
%           \solt{function}[s] \sol{setPermissions}, \sol{grantPermission}, and
%         \sol{revokePermission}.
%       \end{modifiers}
%   \end{functions}
% \end{code}
