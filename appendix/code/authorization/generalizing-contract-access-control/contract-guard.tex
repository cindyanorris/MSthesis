\subsubsection{Contract Guarded}

The \solt{contract}, \sol{contract Guarded}, is a \solt{contract} which offers a
convenient mechanism for easily integrating generalized \solt{contract} access
control functionality; e.g., \sol{contract MyContract is Guarded {}}.
\solt{contract Guarded}, by virtue of implementing the \solt{Authorization
interface}, supports deferring access control responsibilities to an external
\solt{contract} which implements the \solt{Authority interface} while leaving
open the possibility for a \solt{contract} to provide its own access control
implementation by itself implementing the \solt{Authority interface}.

\todo{If I'm changing this from Guard to Guarded then I need to update all of
the references to Guard.}

\begin{solidity}[contract Guarded]
contract Guarded is Owned, Authorization {
  Authority public authority;

  function getAuthority () public constant returns (address _authority) {
    return address(authority);
  }

  function setAuthority (address _authority) public auth returns (bool _success) {
    authority = Authority(_authority);
    return true;
  }

  function isAuthorized (address _subject, bytes4 _resource) public returns (bool _isAuthorized) {
    if (_subject == address(this)) return true;
    if (authority == Authority(0)) return false;
    if (_subject == owner) return true;
    return authority.canCall(_subject, _resource);
  }

  modifier auth {
    assert(isAuthorized(msg.sender, msg.sig));
    _;
  }

  modifier authorized (bytes4 _resource) {
    assert(isAuthorized(msg.sender, _resource));
    _;
  }
}
\end{solidity}

\begin{code}
  \begin{modifiers}
    \item \sol{modifier auth ()}, restricts access to \solt{function} calls
      based on an \solt{address} provided.

      \begin{displayquote}
        Restriction to \solt{function}[s] is accomplished by comparing the
        \solt{address} of the \solt{function} \emph{caller}, \sol{msg.sender},
        against the configured \solt{address}, \sol{address _subject}. If the
        \solt{address} of the \solt{msg.sender} does not match the
        \solt{address} of the \solt{_subject} then the \solt{require} statement
        will force the immediately arrest of \solt{contract} evaluation,
        \solt{revert} the \emph{state} of the \solt{contract}, refund any
        remaining \solt{gas}, \solt{gasleft()}, to the \emph{caller}, and
        exit.\footnotemark{}

        \todo{Should notes be documented differently?}
      \end{displayquote}

      \footnotetext{
        Note that the \emph{caller}, \sol{msg.sender}, of the \solt{function}
        will not necessarily be an \emph{external account}, e.g., human user;
        the \emph{caller} of the \solt{contract} \solt{function} may itself also
        be a \solt{contract}, i.e., a \emph{contract account}, which is calling
        the \solt{contract} \solt{function} from its own \solt{contract}
        \solt{function}.
      }

      \begin{parameters}
      \item \sol{address _subject}, the \emph{subject} who is to be granted
        access to the \solt{function}.

        \begin{displayquote}
          The \solt{address} of the \emph{subject} may be \emph{any} Ethereum
          account, including \emph{contract accounts}.
        \end{displayquote}
      \end{parameters}

    \item \sol{modifier authorized (address _resource)}, restricts access to
      \solt{function} calls based on an \solt{address} provided.

      \begin{displayquote}
        Restriction to \solt{function}[s] is accomplished by comparing the
        \solt{address} of the \solt{function} \emph{caller}, \sol{msg.sender},
        against the configured \solt{address}, \sol{address _subject}. If the
        \solt{address} of the \solt{msg.sender} does not match the
        \solt{address} of the \solt{_subject} then the \solt{require} statement
        will force the immediately arrest of \solt{contract} evaluation,
        \solt{revert} the \emph{state} of the \solt{contract}, refund any
        remaining \solt{gas}, \solt{gasleft()}, to the \emph{caller}, and
        exit.\footnotemark{}

        \todo{Should notes be documented differently?}
      \end{displayquote}

      \footnotetext{
        Note that the \emph{caller}, \sol{msg.sender}, of the \solt{function}
        will not necessarily be an \emph{external account}, e.g., human user;
        the \emph{caller} of the \solt{contract} \solt{function} may itself also
        be a \solt{contract}, i.e., a \emph{contract account}, which is calling
        the \solt{contract} \solt{function} from its own \solt{contract}
        \solt{function}.
      }

      \begin{parameters}
      \item \sol{address _subject}, the \emph{subject} who is to be granted
        access to the \solt{function}.

        \begin{displayquote}
          The \solt{address} of the \emph{subject} may be \emph{any} Ethereum
          account, including \emph{contract accounts}.
        \end{displayquote}
      \end{parameters}
  \end{modifiers}
\end{code}
