\subsubsection{Interface Authority}

The \solt{interface}, \sol{interface Authority}, introduces functionality for
managing whether some \emph{subject}, typically an Ethereum account represented
by \solt{address}, can access some \emph{resource}, typically an Ethereum
\solt{function} represented by \solt{function} signature, \solt{bytes4}. By
defining the \emph{resource} by it's \solt{function} signature and not by a
specific \solt{function} owned by a specific \solt{contract} we leave open the
possibility for managing \solt{contract} access control across
\solt{contract}[s], a functionality which will be necessary to build
pseudo-centralized registration authorities.

\begin{solidity}[interface Authority]
interface Authority {
  function canCall (address _subject, bytes4 _resource) public constant returns (bool _canCall);
}
\end{solidity}

\begin{interface}
  \begin{functions}
    \item \sol{function canCall (address _subject, bytes4 _resource)}, evaluates
      whether some \emph{resource}, \solt{contract function}, can be used by
      some \emph{subject}, Ethereum account.

      \begin{parameters}
        \item \sol{address _subject}, the \emph{subject}, Ethereum account,
          whose permissions are being evaluated.\footnotemark{}

          \footnotetext{
            The \solt{address} of the \emph{subject} may be \emph{any} Ethereum
            account, including \emph{contract accounts}.
          }

        \item \sol{bytes4 _resource}, the \emph{resource}, \solt{contract
          function}, which the \emph{subject's} permissions are being evaluated
          against.
      \end{parameters}

      \begin{returns}
        \item \sol{bool _canCall}, resolves to \solt{true} if the
          \emph{subject}, \sol{address _subject}, is permitted to access the
          \emph{resource}, \sol{bytes4 _resource}, otherwise \solt{false}.
      \end{returns}
  \end{functions}
\end{interface}
