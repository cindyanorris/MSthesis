\subsubsection{Interface Authorization}

The \solt{interface}, \sol{interface Authorization}, introduces functionality
for managing authorities, \sol{function getAuthority()} and \sol{function
setAuthority()}, and also functionality similar to that of an \solt{Authority},
\sol{function isAuthorized()}.% \footnotemark{}

% \footnotetext{
%   If I don't update this to just be \sol{interface AuthorityManager} then I
%   should \emph{really} update \sol{interface Authority} to supply
%   \sol{function isAuthorized ()}, instead of \sol{function canCall()}, and
%   update this \solt{interface} to implement/extend \sol{interface Authority}.
% }

\begin{solidity}[interface Authorization]
interface Authorization {
  function getAuthority () public constant returns (address _authority);
  function setAuthority (address _authority) public returns (bool _success);
  function isAuthorized (address _subject, bytes4 _resource) public returns (bool _isAuthorized);
}
\end{solidity}

\begin{interface}
  \begin{functions}
    \item \sol{function getAuthority ()}, returns the \solt{address} of a
      \solt{contract} which implements the \solt{interface Authority}.

      \begin{returns}
        \item \sol{address _authority}, the \solt{address} of a \solt{contract}
          which implments the \solt{interface Authority}.
      \end{returns}

    \item \sol{function setAuthority (address _authority)}, updates the state of
      the \solt{contract} to to reflect the new \solt{contract} which provides
      an implementation of the \solt{Authority interface}.
      \todo{
        Make sure the Guard checks that the contract actually implements the
        interface!!!
      }

      \begin{parameters}
        \item \sol{address _authority}, the \solt{address} of a \solt{contract}
          which implements the \solt{interface Authority}.
      \end{parameters}

      \begin{returns}
        \item \sol{bool _success}, resolves to \solt{true} if the operation was
          successful; otherwise \solt{false}.
      \end{returns}

    \item \sol{function isAuthorized (address _subject, bytes4 _resource)},
      evaluates whether some \emph{subject}, Ethereum account, is authorized to
      access some \emph{resource}, \solt{contract function}.

      \begin{parameters}
        \item \sol{address _subject}, the \emph{subject}, Ethereum account,
          whose permissions are being evaluated.\footnotemark{}

          \footnotetext{
            The \solt{address} of the \emph{subject} may be \emph{any} Ethereum
            account, including \emph{contract accounts}.
          }

        \item \sol{bytes4 _resource}, the \emph{resource}, \solt{contract
          function}, which the \emph{subject's} permissions are being evaluated
          against.
      \end{parameters}

      \begin{returns}
        \item \sol{bool _isAuthorized}, resolves to \solt{true} if the
          \emph{subject}, \sol{address _subject}, is permitted to access the
          \emph{resource}, \sol{bytes4 _resource}, otherwise \solt{false}.
      \end{returns}
  \end{functions}
\end{interface}
