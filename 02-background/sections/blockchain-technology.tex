\section{Blockchain Technologies}\label{sec:blockchain-technologies}
This research considers the utility of blockchain technologies with respect to
election systems. A blockchain is a distributed transactional database system
that tracks transactions in ever-growing linked blocks. Blockchain technologies
have been used to create immutable public ledgers for tracking currency, assets,
and rights data. The most notable technology blockchain technology in use today
is Bitcoin.

% Citation Needed
Blockchain technology exists to provide a solution to the Byzantine Generals'
Problem in a distributed computing environment.\footnotemark{} It allows parties
on an untrusted and unreliable network to build a trusted source of truth for
transaction history. Blockchains use various algorithms to score different
versions of transaction history over time to reach consensus within their
network. Blockchains might be best described as a tool used to commodify trust.

\footnotetext{
  The Byzantine Generals' Problem is a thought experiment which helps to
  illustrate some of the difficulties regarding coordination over untrusted
  networks. In short, two generals wish to coordinate an attack on an enemy
  fortress. They must attack simultaneously to succeed in taking the fortress.
  However, the general's only means of coordination is via courier, and the only
  routes available for the couriers to travel necessitate travel through the
  fortress. How do the generals coordinate their attack considering that their
  couriers might be intercepted, or worse still, replaced with dishonest spies?
}

There are interesting overlaps and parallels to be drawn from the consensus
mechanisms used in these, and other computer networks, and the consensus
mechanisms leveraged to make decisions as a group in democratic governance
systems.

% Sub-Section: Bitcoin
\subsection{Bitcoin}
In 2008 the seminal white paper, \emph{Bitcoin: A Peer-to-Peer Electronic Cash
System}, was published under the moniker Satoshi Nakamoto.\cite{bitcoin} This
white paper outlined ideas for a new form of currency, Bitcoin. Bitcoin promised
to be the first of its kind; it would become the world's first decentralized
digital currency that would require no trust to authenticate timestamped
transactions. It would do this by combining cryptography, a proof-of-work
system, and ``miners'' to create a revolutionary new concept that is now known
as blockchain technology. In short, blockchain technology enables individuals
who do not trust one another to reach consensus via a trustless platform. More
concretely, Bitcoin can be used by Alice and Bob to send money from one to the
other over a decentralized network.

% \subsubsection{Network Topology}
\subsubsection{Network Topology}
The Bitcoin network is structured as a peer-to-peer network composed of a
variety of node types. Each node supports varying functionalities and features
based on their use case. Common functionalities and features required
include:\cite{mastering-bitcoin}
\begin{itemize}
    \item Mining functionality, to support creating new blocks by solving
      proof-of-work problems.
    \item A wallet, to offer users a way to manage their keys plus send and
      receive transactions.
    \item A copy of the full blockchain, which allows nodes to verify
      transactions independent of other nodes in the network.
    \item Network routing capabilities, which allows nodes to propagate
      transactions and discover new nodes.
\end{itemize}

\paragraph{Node Types} The most common node types, classified by their
corresponding functionalities, are as follows:\cite{mastering-bitcoin}
\begin{itemize}
    \item A \emph{reference client} contains wallet, miner, a full copy of the
        blockchain, and network routing functionality.
    \item A \emph{full node} contains a full copy of the blockchain and
        network routing functionality.
    \item A \emph{mining node} contains a miner, a full copy of the
        blockchain, and network routing functionality.
    \item A \emph{lightweight wallet} contains a wallet and network routing
        functionality. These nodes depend on \emph{full nodes} to verify
        transactions for them. These nodes are usually found on mobile devices
        where storage is limited and the size of the blockchain is inhibitive.
\end{itemize}

\subsubsection{Fundamental Concepts}
The Bitcoin blockchain is structured as a linked list of blocks, as seen in
Figure~\ref{fig:blockchain}. Each block contains a set of transactions and a
reference to the previous block in the chain. Blocks are identified by the
SHA-256 hash of its header. It is helpful to imagine the blockchain as being
blocks stacked vertically, each additional block helping to secure the previous
blocks laid before it.

\begin{figure}[H]
    \centering
    \figurepdf[width=\textwidth]{blockchain}
    \caption{A chain of linked Bitcoin blocks.~\cite{bitcoin}}\label{fig:blockchain}
    % \includestandalone[width=\textwidth]{\fig{blockchain}}
\end{figure}

\paragraph{Blocks}
The block is the Bitcoin blockchain primitive. Blocks serve as a time-stamping
tool for transactions within the network and also prove that some amount of work
(computation) occurred. The contents of a block are listed in
Table~\ref{tab:block}. Figure~\ref{fig:block} illustrates the layout of a block.

\begin{figure}[H]
    \centering
    \figurepdf{block}
    \caption{A single Bitcoin block and its internal representation.~\cite{bitcoin}}\label{fig:block}
    % \includestandalone{\fig{block}}
\end{figure}

\begin{table}
    \caption{The contents of a block \protect\cite{mastering-bitcoin}.}\label{tab:block}
    \resizebox{\columnwidth}{!}{%
        \begin{tabu} to \textwidth{@{} l l l @{}}
            \toprule
            Size                    & Field                 & Description \\
            \midrule
            4 bytes                 & Block Size            & The size of the block, in bytes, following this field \\
            80 bytes                & Block Header          & Several fields form the block header \\
            1--9 bytes (VarInt)     & Transaction Counter   & How many transactions follow \\
            Variable                & Transactions          & The transactions recorded in this block \\
            \bottomrule
        \end{tabu}
    }
\end{table}

\subparagraph{Block Header}
The block header is responsible for tracking the metadata of the Bitcoin block.
It is also used to represent the block as a whole by SHA-256 hashing the
contents of it. The contents of the block header are as seen in
Table~\ref{tab:block-header}.

\begin{table}[H]
  \setstretch{1.5}
  \caption{The contents of a block header \protect\cite{mastering-bitcoin}.}\label{tab:block-header}
  \resizebox{\columnwidth}{!}{%
    \begin{tabu} to \textwidth{@{} l l l @{}}
      \toprule
      Size            & Field                 & Description \\
      \midrule
      4 bytes         & Version               & A version number to track software/protocol upgrades \\
      32 bytes        & Previous Block Hash   & A reference to the hash of the previous (parent) block in the chain \\
      32 bytes        & Merkle Root           & A hash of the root of the Merkle tree of this block's transactions \\
      4 bytes         & Timestamp             & The approximate creation time of this block (seconds from Unix Epoch) \\
      4 bytes         & Difficulty Target     & The proof-of-work algorithm difficulty target for this block \\
      4 bytes         & Nonce                 & A counter used for the proof-of-work algorithm \\
      \bottomrule
    \end{tabu}
  }
\end{table}

\subparagraph{Merkle Trees}
A Merkle tree is a data structure that allows one to efficiently verify the
contents of a large amount of data. Merkle trees are used extensively in
peer-to-peer networks to ensure that blocks of data arrive unaltered and
undamaged. The root of a Merkle tree is called a Merkle root. Merkle trees are
composed of nodes of hashes. They have the unique property of allowing the
verification of the existence of a hash in the tree in \solt{O(log(n))} time.

% [!Merkle Tree Image]()

\paragraph{Transactions}
In Satoshi's original white paper, a coin was defined as tokens transferred via
a chain of digital signatures.

\begin{displayquote}
  ``We define an electronic coin as a chain of digital signatures. Each owner
  transfers the coin to the next by digitally signing a hash of the previous
  transaction and the public key of the next owner and adding these to the end
  of the coin. A payee can verify the signatures to verify the chain of
  ownership.''\cite{bitcoin}
\end{displayquote}

\begin{figure}[H]
    \centering
    \figurepdf{transactions}
    \caption{A chain of signatures representing an electronic coin.~\cite{bitcoin}}\label{fig:coin}
    % \includestandalone{\fig{transactions}}
\end{figure}

Figure~\ref{fig:coin} demonstrates Owner 1 hashing the transaction that gave her
ownership of the coin with the public key of the new owner (Owner 2), then
signing that hash with her private key and publishing that as a new
transaction. Owner 2 repeats the process to send his coins to Owner 3.


% \textbf{Blockchain Stack}
% \begin{figure}[H]
%     \caption{A blockchain stack with 10 blocks.}\label{tab:blockchain-stack}
%     \begin{tabu} to \textwidth{@{} l l l @{}}
%         \toprule
%         Block \# \\
%         \midrule
%         Block 9 \\
%         Block 8 \\
%         $\cdots$ \\
%         Block 1 \\
%         Block 0 \\
%         \bottomrule
%     \end{tabu}
% \end{figure}

% |  Block #  |
% |:---------:|
% |  Block 9  |
% |  Block 8  |
% |    ...    |
% |  Block 1  |
% |  Block 0  |



% | Size               | Field               | Description                                           |
% |--------------------|---------------------|-------------------------------------------------------|
% | 4 bytes            | Block Size          | The size of the block, in bytes, following this field |
% | 80 bytes           | Block Header        | Several fields form the block header                  |
% | 1-9 bytes (VarInt) | Transaction Counter | How many transactions follow                          |
% | Variable           | Transactions        | The transactions recorded in this block               |


% | Size     | Field               | Description                                                           |
% |----------|---------------------|-----------------------------------------------------------------------|
% | 4 bytes  | Version             | A version number to track software/protocol upgrades                  |
% | 32 bytes | Previous Block Hash | A reference to the hash of the previous (parent) block in the chain   |
% | 32 bytes | Merkle Root         | A hash of the root of the Merkle tree of this block's transactions    |
% | 4 bytes  | Timestamp           | The approximate creation time of this block (seconds from Unix Epoch) |
% | 4 bytes  | Difficulty Target   | The proof-of-work algorithm difficulty target for this block          |
% | 4 bytes  | Nonce               | A counter used for the proof-of-work algorithm                        |
%Source: Mastering Bitcoin

\paragraph{Proof-of-Work}
The bitcoin protocol uses a \emph{Proof-Of-Work (POW)} algorithm similar to
hashcash.\cite{hashcash,bitcoin} The goal of this proof-of-work algorithm is to
create a problem that is easy to verify for correctness but difficult to solve
for. The proof-of-work algorithm provides a means for mining nodes to be
pseudorandomly selected to build a block of transactions. The probability that a
miner will be selected is directly tied to the amount of \emph{work} a miner
does.
 % (read: NP-complete (?)).
% cite: hashcash

\subparagraph{Algorithm}
The proof-of-work algorithm depends on the \emph{SHA-256} cryptographic hashing
function, a member of the \emph{Secure Hash Algorithm 2 (SHA-2)} family, which
produces a 256-bit, 32-byte, hash result.

The primary requirements which a \emph{cryptographic hash function} must fulfill
are that it be:

\begin{itemize}
  \item deterministic, i.e., the same input will always produce the same output,
  \item quick to compute,
  \item infeasible to determine the input from the output, i.e., a small change
    in the input will produce a major, seemingly random, change to the output,
    and
  \item infeasible to find a collision in resulting hashes.
\end{itemize}

These properties of cryptographic hash functions provide \emph{collision
resistance}, meaning it is computationally infeasible to find an input that
produces a randomly selected hash output. This property of collision resistance
is leveraged to build the pseudorandom selection process that determines which
node is able to build the next block of transactions.

\subparagraph{Difficulty}
The Bitcoin network has a global \emph{difficulty} --- a 256-bit,
32-byte, value --- that is recalculated every 2016 blocks. The value is
recalculated such that the pseudorandom ``mining'' process to mint a new block
will take approximately 10 minutes to complete for each block minted.

While it is helpful to describe miners as being ``selected'' to build, or mint,
new blocks, it is also inaccurate; in reality miners have a chance of building a
correct block which can be measured as a probability, i.e., the miner's hash
rate relative to the total network hash rate:
\(\frac{miner\_hashrate}{network\_hashrate}\).

Miners repeatedly compute SHA-256 hashes, combining the previous block's header
hash, the current block data, and a \emph{nonce} (a pseudorandom shifting value
selected by the miner). The miner's objective is to find a nonce which produces
a hash less than the current difficulty. Due to the properties of the SHA-256
hash this process is, practically speaking, a brute-force processes of trial and
error, (the ``work'' in Proof-of-Work). Once a miner has discovered a valid
hash they have generated a valid block.

\subparagraph{Network}
Once a miner generates a valid block it will propagate its solution into the
network. Other nodes will then verify that the block is correct and will append
it to their chain. One of the transactions which miners will include in their
block is a coinbase transaction, a transaction containing reward paid to the
miner in the form of newly minted bitcoin.

\subparagraph{Security}
These properties, combined with the incentive of coinbase rewards, provide
security in the form of cryptography, electricity, and hardware. Attacks that
would threaten this security depend on breaking the cryptographic primitives in
action, finding ways of reducing electricity/hardware costs to outperform the
rest of the network in the PoW algorithm, attacking nodes in the network, or
colluding with other nodes in the network to outperform the remainder of the
network.

\subsubsection{Example Transaction: Alice to Bob}

As an example, we might imagine a circumstance where \emph{Alice wants to send
Bob 0.100 BTC.} Assume Alice controls two addresses:

\begin{enumerate}
  \item \sol{1PwDUn9Vxn9CyaRkZfJrTzYRg6QNCygALY}, which contains 6900000
    satoshis (0.069 BTC).
  \item \sol{1Ah1CWQ1Zxax2yzg3EZBmSmZTKDoHTuUAi}, which contains 7900000
    satoshis (0.079 BTC).
\end{enumerate}

Alice does not have enough bitcoin in either address independently to send Bob
the 0.100 BTC.\ To resolve this Alice creates a new transaction which takes as
an input both of her addresses. Alice proves she controls both addresses by
signing each transaction address and publishing her public key for each
respectively. This process allows a node on the bitcoin network to verify both
that public key is associated with the address (through a series of
cryptographic hashes) and that the person publishing this transaction controls
the private keys (by verifying the signature). Next Alice sets an output address
to Bob's address, an amount to send to Bob (0.100 BTC), and the rules Bob must
fulfill to access the bitcoin she is sending to him (typically signing the
transaction and publishing the associated public key). Alice could now publish
this transaction to the bitcoin network; however, in doing so she would
sacrifice 0.048 BTC, the remainder from her two addresses, which the miners
would claim as a transaction fee. To avoid this Alice can set a second
transaction output address to an address she controls, sending any additional
coins there that she does not want to offer as a transaction fee.


% Sub-Section: Ethereum
\subsection{Ethereum}

Ethereum is a blockchain-based system initially proposed in late 2013,
post-Bitcoin, and released in 2015.\cite{ethereum-white} Ethereum introduced a
few novel features and functionalities with its blockchain network which are not
available in the Bitcoin network. Most notably Ethereum offers a distributed
decentralized Turing-complete computing platform via the Ethereum Virtual
Machine (EVM), which aims to provide an application layer which ``[runs] exactly
as programmed without any possibility of downtime, censorship, fraud or third
party interference.''\cite{ethereum.org,ethereum-yellow}

\subsubsection{Fundamental Concepts}

\paragraph{Currency}
Like Bitcoin, Ethereum's blockchain defines and leverages its own token,
\solt{ether} --- also referred to as \solt{eth} or ETH, and sometimes denoted
using the Greek symbol Xi, $\Xi$, the uppercase Old English letter Eth, \DH, or,
more rarely, $\blacklozenge$ --- which acts as the underlying currency of its
blockchain protocol. The smallest unit of \solt{ether} is
\emph{wei}.~\cite{mastering-ethereum,ethereum-yellow} The
denominations of \solt{ether} are broken down as follows:
% \ref{book:mastering-ethereum}
% \ref{paper:ethereum}

% \todo{Add denomination table here}

% \ref{book:mastering-ethereum}
% \ref{paper:ethereum}
%
% See page 2 of Ethereum yellow paper.
%
% Table 2-1 from page 14 of Mastering Ethereum.
%
% | Value (in wei) | Exponent | Common Name | SI Name               |
% |----------------|----------|-------------|-----------------------|
% | 1              | 1        | wei         | Wei                   |
% | 1000           | 10^3     | Babbage     | Kilowei or femroether |
% |                |          |             |                       |
% |                |          |             |                       |

\begin{center}
    \setstretch{1.5}
    \footnotesize
    \begin{longtabu} to \textwidth{@{} X[3] X[] X[] X[2] @{}}
    \caption{Ether Denominations~\cite{mastering-ethereum}}\label{tab:eth-denominations} \\
    \toprule
    Value (in wei)                    & Exponent & Common Name & SI Name \\
    \midrule
    \endfirsthead
    \toprule
    Value (in wei)                    & Exponent & Common Name & SI Name \\
    \midrule
    \endhead
    \taburowcolors{white..gray!15}
    1                                 & $10^{0}$    & wei       & Wei \\
    1,000                             & $10^{3}$    & Babbage   & Kilowei or femtoether \\
    1,000,000                         & $10^{6}$    & Lovelace  & Megawei or picoether \\
    1,000,000,000                     & $10^{9}$    & Shannon   & Gigawei or nanoether \\
    1,000,000,000,000                 & $10^{12}$   & Szabo     & Microether or micro \\
    1,000,000,000,000,000             & $10^{15}$   & Finney    & Milliether or milli \\
    1,000,000,000,000,000,000         & $10^{18}$   & Ether     & Ether \\
    1,000,000,000,000,000,000,000     & $10^{21}$   & Grand     & Kiloether \\
    1,000,000,000,000,000,000,000,000 & $10^{24}$   &           & Megaether \\

    %extern u256 const c_copyGas;			///< Multiplied by the number of 32-byte words that are copied (round up) for any *COPY operation and added.
    \bottomrule
    % \begin{varwidth}{\textwidth}
    %   These names pay homage to some of the great minds of cryptography and
    %   computer science.
    % \end{varwidth}
  \end{longtabu}
\end{center}

\paragraph{Accounts}
Accounts are an Ethereum primitive which provide an abstraction over the Bitcoin
equivalent signature chain process; this abstraction helps to both simplify the
concept of token ownership as well as extend the idea of what a token is, what a
token can be, and how blockchain state can be managed and organized.

Accounts are identified by a 160-bit code, their \solt{address}, and are
internally represented by four properties:

\begin{enumerate}
  \item \sol{nonce}, a monotonically increasing counter which represents the
    number of transactions sent from the account.
  \item \sol{balance}, the amount of \solt{ether}, expressed in wei, which is
    owned by the account.
  \item \sol{storageRoot}, a 256-bit hash of the root node of a Merkle Patricia
    tree which encodes the storage contents of the account.

    % RLP - Recursive Length Prefix
    % https://eth.wiki/en/fundamentals/rlp

    % https://blog.ethereum.org/2015/11/15/merkling-in-ethereum/
    %
    % From the yellow paper:
    % \displayquote{
    %   A 256-bit hash of the root node of a Merkle Patricia tree that encodes
    %   the storage contents of the account (a mapping between 256-bit integer
    %   values), encoded into the trie as a mapping from the Keccak 256-bit hash
    %   of the 256-bit integer keys to the RLP-encoded 256-bit integer values.
    %   This tree encodes the hash of the storage contents of this account, and
    %   is empty by default.
    % }


  \item \sol{codeHash}, an immutable hash of the EVM code corresponding to the
    account.
\end{enumerate}

Although all accounts are structurally identical it is useful to distinguish
between the two practical kinds of accounts which one is likely to encounter and
interact with on the Ethereum blockchain, \emph{external accounts} and
\emph{internal accounts}:
%
\begin{description}[font=\normalfont\emph]
  % \subsubparagraph{External Accounts}
  \item[External Accounts] --- also referred to as simple accounts, non-contract
    accounts, externally owned accounts (EOA), and sometimes user accounts ---
    are defined as accounts whose \sol{codeHash} value is the Keccak-256 hash of
    an empty string; i.e., the account contains no code.

  %\subsubparagraph{Internal Accounts}
  \item[Internal Accounts] --- also referred to as contract accounts --- are
    those accounts which are not external accounts; i.e., the account contains
    code.
\end{description}
%
Both kinds of accounts have the ability to send and receive \solt{ether} as well
as interact with \solt{contract}[s] which have been deployed to the Ethereum
network. However, there are some key differences between the two kinds of
accounts which are worth highlighting:\cite{ethereum:accounts}

\begin{itemize}[label=]
  \item External Accounts
    \begin{itemize}[label=$\bullet$]
      \item External accounts are managed by public-key cryptography.
      \item External account creation costs no \solt{ether}.
      \item Only external accounts can \emph{initiate} transactions.
      \item Transactions between external accounts can only transact
        \solt{ether}.
    \end{itemize}
  \item Internal Accounts
    \begin{itemize}[label=$\bullet$]
      \item Internal accounts are managed by code.
      \item Internal account creation costs \solt{ether} which reflects the cost
        of storing code on the Ethereum network.
      \item Internal accounts can execute code via the EVM upon receiving
        transactions, enabling a wide range network functionalities.
      \item Internal accounts can only send transactions in response to
        receiving transactions. \todo{This will have a major impact on the
        methodologies.}
    \end{itemize}
\end{itemize}

\paragraph{Transactions}
A transaction is a cryptographically-signed instruction constructed by an
external actor and submitted to the Ethereum network. There are two kinds of
transactions worth distinguishing, contract creation transactions and message
call transactions; both kinds of transactions share the following common
properties:

\begin{enumerate}
  \item \sol{to}, a 160-bit \solt{address} representing the \emph{recipient's}
    account. This value is omitted when building a contract creation
    transaction.

  \item \sol{from}, a signature which identifies the \emph{sender} of the
    transaction by account \solt{address}.\footnotemark{}

    \footnotetext{
      This field does not \emph{technically} exist, in actuality the signature
      is represented as three distinct fields (\emph{v}, \emph{r}, \emph{s}),
      which can be used to determine the \solt{address} representing the sender
      of the transaction.
    }

  \item \sol{value}, the amount of \solt{ether}, expressed in wei, which is
    to be transferred to the recipient.

  \item \sol{nonce}, a value equal to the number of transactions which have been
    sent by the \emph{sender}.

  \item \sol{gasPrice}, a value representing the amount of wei to be paid per
    unit of gas (expanded on below).

  \item \sol{gasLimit}, a value representing the maximum amount of gas which
    should be used executing the transaction.
\end{enumerate}

Contract creation transactions include the following additional property in the
transaction:

\begin{enumerate}
  \item \sol{init}, an EVM-code fragment which is executed only once and
    discarded thereafter; it returns the \sol{body}, a second fragment of code
    that executes each time the account receives a message call, which can occur
    either by transaction or internal execution of code.
\end{enumerate}

In contrast, a message call transaction includes the following additional
property:

\begin{enumerate}
  \item \sol{data}, an unlimited size byte array which contains the input data
    of the message call.
\end{enumerate}

\paragraph{Ethereum Virtual Machine}\label{eth:evm}
The Ethereum Virtual Machine (EVM) is the execution environment which Ethereum
code is processed in. The EVM processes low-level bytecode and takes actions
against the state of the Ethereum blockchain in response: reading, processing,
and writing data. The Ethereum Virtual Machine Specification introduces a
low-level instruction set which defines the available operations which an EVM
implementation should support: the opcodes, their inputs, outputs, and various
other implementation details. The EVM can be described as a \emph{state
transition function}, $Y(S, T) = S'$; given a set of transactions, $T$, and an
initial state, $S$, the state transition function, $Y(S, T)$, will produce a new
output state, $S'$.\cite{ethereum-yellow} Several EVM implementations exist
which have been written in various languages, e.g., Go, Python, and C++.

\subparagraph{Language Support}\label{eth:languages}
Several higher-level languages exist which target the Ethereum Virtual Machine
and can be used to build ``smart contracts;'' e.g., Solidity which draws
inspiration from C++ and JavaScript, and Vyper which describes itself as a
``Pythonic Smart Contract Language.'' These languages ship with compilers which
can be used to translate their code into low-level EVM bytecode; Solidity, for
example, is compiled using \solt{solc}, the Solidity compiler.

% \subsubsection{Network Topology}
\subsubsection{Network Topology}
Like Bitcoin, Ethereum exists as a network of nodes, each node supporting
differing functionalities, which work collectively to construct the Ethereum
blockchain and support the surrounding ecosystem. These nodes can be classified
by the functionalities which they support.

\paragraph{Node Types}
There exists several implementations of the Ethereum protocol, i.e., Ethereum
clients, which have been written in various languages, e.g., Geth in Go, Parity
in Rust, and pyethereum in Python. Some of these clients support running in
different modes.

\begin{itemize}
  \item A \emph{full node} is a complete implementation of the Ethereum
    protocol. A full nodes processes and validates all transactions which have
    been added to the Ethereum blockchain, thus helping to support the
    resiliency and reliability of the network. To support this functionality, a
    full node must maintain a complete copy of the blockchain. Full nodes are
    also capable of deploying and interacting with contracts, support mining and
    wallet functionality, and are able to route transactions throughout the
    network.

  \item A \emph{remote client} supports a subset of the functionality that a
    full node supports, generally wallet functionality and the ability to
    broadcast transactions. Other more complex functionalities generally require
    interacting with with a full node or other remote services which are capable
    of fulfilling requests on a remote client's behalf.
\end{itemize}

\paragraph{The Blockchain}
Like Bitcoin, the Ethereum blockchain is constructed by leveraging a
Proof-of-Work (PoW) algorithm to reach consensus throughout the network. The
proof-of-work algorithm leveraged by Ethereum, Ethash, helps to build trust and
reliability throughout the network while also securing the blockchain and
ensuring that EVM code execution has been processed and that the results
produced from said execution are as expected. Ethereum offers cryptoeconomic
incentivization in the form of \solt{ether} to promote participation in the
proof-of-work process.\cite{ethereum-yellow}

\subparagraph{Blocks}
The Ethereum protocol groups collections of transactions into \emph{blocks}.
Blocks are linked to previous blocks, via cryptographic hash, which reflect the
prior states of the blockchain. When processed collectively these blocks reflect
the current state of the Ethereum blockchain.

\subparagraph{Proof-of-Work}
Ethereum's proof-of-work algorithm, Ethash, is a proof-of-work algorithm which
was initially inspired by the Hashimoto and Dagger algorithms. The primary
motivation behind the Ethash algorithm was to produce a PoW algorithm which
would be resistant to application-specific integrated circuits (ASICs). The
primary mechanism leveraged to achieve ASIC-resistance lies in the algorithm's
memory-bound nature: a significant amount of memory, in addition to computation,
is required to correctly compute a proof-of-work solution. By requiring large
amounts of memory-bound operations the algorithm makes itself resistant to most
kinds of specialized memories and caches. Additionally, the memory requirements
are designed to grow and shift over time such that building rapid static caches
would become prohibitively expensive. In some sense the Ethash algorithm might
be better described as Proof-of-Memory.\cite{dagger-hashimoto,hashimoto,dagger}

The Ethereum network has been attempting to migrate away from Ethash to a
lower-cost consensus algorithm operating through Proof-of-Stake (PoS) but has
yet to complete the transition.

% Talk about the Greedy Heaviest Observed Subtree (GHOST) protocol.

\subparagraph{Gas}
In order to validate that EVM code has been executed, and executed as expected,
each full node on the network must recompute all transactions and whatever EVM
code those transactions have triggered when validating blocks. It is not
difficult to imagine how this might cause serious problems and introduce room
for exploitation within the network, \sol{while true { expensiveOperation() }}.
In order to address this the Ethereum specification introduces an abstraction,
\emph{gas}, which has a market-based value and must be paid by the
transaction-sender up-front when generating a transaction. Each operation
computed by the EVM has an associated gas-price and that gas price is ``paid''
to the node who successfully mints a block. If an insufficient amount of gas is
provided by the transaction-sender then the EVM will throw an out-of-gas (OOG)
exception: execution halts, the blockchain state is restored, and all gas
submitted is forfeited to the node. If an excess of gas is provided then any gas
remaining after transaction execution is refunded to the sender. When generating
a transaction on the Ethereum network the creator of the transaction has the
choice of defining how much wei they are willing to pay per unit of gas. Nodes
are incentivized to process transactions which offer more wei per unit of gas
relative to the rest of the transactions available for processing on the
network. Table~\ref{tab:gas-costs} introduces gas costs as defined by the
Ethereum Yellow Paper.


% Stuff about gas. What follows is a table of some gas costs as defined in the Ethereum
% yellow paper.

\begin{center}
  \setstretch{1.5}
  \scriptsize
  \begin{longtabu} to \textwidth{@{} X[1.1,l] X[0.5,r] X[7,l] @{}}
  \caption{Fee Schedule\cite{ethereum-yellow}}\label{tab:gas-costs} \\
  \toprule
  Name & Value & Description* \\
  \midrule
  \endfirsthead
  \toprule
  Name & Value & Description* \\
  \midrule
  \endhead
  \taburowcolors{white..gray!15}
  $G_{zero}$ & 0 & Nothing paid for operations of the set {\tiny $W_{zero}$}. \\
  $G_{base}$ & 2 & Amount of gas to pay for operations of the set {\tiny $W_{base}$}. \\
  $G_{verylow}$ & 3 & Amount of gas to pay for operations of the set {\tiny $W_{verylow}$}. \\
  $G_{low}$ & 5 & Amount of gas to pay for operations of the set {\tiny $W_{low}$}. \\
  $G_{mid}$ & 8 & Amount of gas to pay for operations of the set {\tiny $W_{mid}$}. \\
  $G_{high}$ & 10 & Amount of gas to pay for operations of the set {\tiny $W_{high}$}. \\
  $G_{extcode}$ & 700 & Amount of gas to pay for operations of the set {\tiny $W_{extcode}$}. \\
  $G_{balance}$ & 400 & Amount of gas to pay for a {\tiny BALANCE} operation. \\
  $G_{sload}$ & 200 & Paid for a {\tiny SLOAD} operation. \\
  $G_{jumpdest}$ & 1 & Paid for a {\tiny JUMPDEST} operation. \\
  $G_{sset}$ & 20000 & Paid for an {\tiny SSTORE} operation when the storage value is set to non-zero from zero. \\
  $G_{sreset}$ & 5000 & Paid for an {\tiny SSTORE} operation when the storage value's zeroness remains unchanged or is set to zero. \\
  $R_{sclear}$ & 15000 & Refund given (added into refund counter) when the storage value is set to zero from non-zero. \\
  $R_{selfdestruct}$ & 24000 & Refund given (added into refund counter) for self-destructing an account. \\
  $G_{selfdestruct}$ & 5000 & Amount of gas to pay for a {\tiny SELFDESTRUCT} operation. \\
  $G_{create}$ & 32000 & Paid for a {\tiny CREATE} operation. \\
  $G_{codedeposit}$ & 200 & Paid per byte for a {\tiny CREATE} operation to succeed in placing code into state. \\
  $G_{call}$ & 700 & Paid for a {\tiny CALL} operation. \\
  $G_{callvalue}$ & 9000 & Paid for a non-zero value transfer as part of the {\tiny CALL} operation. \\
  $G_{callstipend}$ & 2300 & A stipend for the called contract subtracted from $G_{callvalue}$ for a non-zero value transfer. \\
  $G_{newaccount}$ & 25000 & Paid for a {\tiny CALL} or {\tiny SELFDESTRUCT} operation which creates an account. \\
  $G_{exp}$ & 10 & Partial payment for an {\tiny EXP} operation. \\
  $G_{expbyte}$ & 50 & Partial payment when multiplied by $\lceil\log_{256}(exponent)\rceil$ for the {\tiny EXP} operation. \\
  $G_{memory}$ & 3 & Paid for every additional word when expanding memory. \\
  $G_\text{txcreate}$ & 32000 & Paid by all contract-creating transactions after the {\it Homestead transition}.\\
  $G_{txdatazero}$ & 4 & Paid for every zero byte of data or code for a transaction. \\
  $G_{txdatanonzero}$ & 68 & Paid for every non-zero byte of data or code for a transaction. \\
  $G_{transaction}$ & 21000 & Paid for every transaction. \\
  $G_{log}$ & 375 & Partial payment for a {\tiny LOG} operation. \\
  $G_{logdata}$ & 8 & Paid for each byte in a {\tiny LOG} operation's data. \\
  $G_{logtopic}$ & 375 & Paid for each topic of a {\tiny LOG} operation. \\
  $G_{sha3}$ & 30 & Paid for each {\tiny SHA3} operation. \\
  $G_{sha3word}$ & 6 & Paid for each word (rounded up) for input data to a {\tiny SHA3} operation. \\
  $G_{copy}$ & 3 & Partial payment for {\tiny *COPY} operations, multiplied by words copied, rounded up. \\
  $G_{blockhash}$ & 20 & Payment for {\tiny BLOCKHASH} operation. \\

  %extern u256 const c_copyGas;			///< Multiplied by the number of 32-byte words that are copied (round up) for any *COPY operation and added.
  \bottomrule
  \end{longtabu}
  % \begin{itemize} % [label=,leftmargin=0mm,topsep=0mm]
  \begin{multicols}{2}
    \begin{itemize}[label=,leftmargin=0mm,topsep=0mm,itemsep=0em]
      \scriptsize
      \item $W_{zero}$ = \tiny\{{STOP}, {RETURN}\}
      \scriptsize
      \item $W_{low}$ = \tiny\{{MUL}, {DIV}, {SDIV}, {MOD}, {SMOD}, {SIGNEXTEND}\}
      \scriptsize
      \item $W_{mid}$ = \tiny\{{ADDMOD}, {MULMOD}, {JUMP}\}
      \scriptsize
      \item $W_{high}$ = \tiny\{{JUMPI}\}
      \scriptsize
      \item $W_{extcode}$ = \tiny\{{EXTCODESIZE}\}
      \scriptsize
      \item $W_{base}$ = \tiny\{{ADDRESS}, {ORIGIN}, {CALLER}, {CALLVALUE},
        {CALLDATASIZE}, {CODESIZE}, {GASPRICE}, {COINBASE} {TIMESTAMP},
        {NUMBER}, {DIFFICULTY}, {GASLIMIT}, {POP}, {PC}, {MSIZE}, {GAS}\}
      \scriptsize
      \item $W_{verylow}$ = \tiny\{{ADD}, {SUB}, {NOT}, {LT}, {GT}, {SLT},
        {SGT}, {EQ}, {ISZERO}, {AND}, {OR}, {XOR}, {BYTE}, {CALLDATALOAD},
        {MLOAD}, {MSTORE}, {MSTORE8}, {PUSH*}, {DUP*}, {SWAP*}\}
    \end{itemize}
  \end{multicols}
\end{center}

% \paragraph{Transaction Examples}
% \subparagraph{Example 1: Alice to Bob}
% \subparagraph{Example 2: Alice to Contract}
% \subparagraph{Example 3: Contract to Contract}

