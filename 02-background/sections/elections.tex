\section{Elections}\label{sec:elections}
Elections are perhaps the most obvious and intuitive means of allowing members
of a population to express their will and take part in their governance process.
An election is the generalized process of allowing individuals to express their
preference, typically by way of vote, as a means to come to consensus as a
group. The generalized goal of elections is to reach a consensus which
accurately and fairly reflects the preferences of the participating voters.
This, at first glance, seems like a trivial problem, and generally is in
circumstances where the number of voters and choices is small in number.
However, elections become complex as the number of actors, choices, or election
cycles increase. There are social, mathematical, and practical engineering
constraints which all voting systems are bound by, subjected to, and forced to
address.

% If we exclude singular autocratic and chaotic anarchic forms of governance then
% a system usually arises where voting is necessary.

\subsection*{Electoral Systems}
An electoral system, is the combination of rules, norms, and procedures which
define how a final result, and ultimately consensus, will be determined during
an election. Electoral systems can considered as a composition of three
components: ballot, choices, and tallying algorithm. These three components can
be woven together to produce a wide spectrum of electoral systems with varying
characteristics and properties. The choice and implementation of electoral
system has a direct and profound impact on the ways in which democratic systems
can and will operate as well as on the perceived legitimacy of the governance
model. The decisions concerning the implementation details of electoral systems
are among the most important decisions that any democratic organization will
make; the choice of design impacts all future decision making processes and
shapes the future of the governance model itself. A poorly designed electoral
system can have disastrous effects on the health and perceived legitimacy of a
democratic organization both in the immediate and long-term future. Further,
once chosen, an electoral system can be difficult to amend as political
interests respond to and solidify around the incentives presented to them which
are inherent to the electoral system chosen.\cite{electoral-handbook}

\paragraph{Choices}
A electoral system's choices are the set of options that a voting actor can
select from; they are the \emph{who} or \emph{what} being decided in an
election. The choices available may be determined by primary vote, polling,
write-in, debate, etc.\ or some other methodology or combination of
methodologies.

\subparagraph{District Magnitude}
The number of choices that will be selected as winners is of great importance in
an election. In representative democracies the measurement of seats (choices)
available is known as district magnitude. An election where the district
magnitude is one, i.e., where a single candidate or choice is to be elected or
selected, is known as a single-member district (SMD) or single-winner district.
An election where the district magnitude is greater than one, i.e., multiple
candidates are to be elected, is known as a multi-member district (MMD) or
multi-winner district.% \footnotemark{}

% \footnotetext{
%   % Much of the literature surrounding electoral systems presumes democracy via
%   % representation; this has had an influence on the language and terminology
%   % used to describe electoral systems. For example, when evaluating district
%   % magnitude terms such as ``single-member district'' and ``multi-member
%   % district'' are used --- implying that the seats available for candidacy is
%   % the metric being measured --- however, the significance of these concepts
%   % applies just the same to a vote being used to determine which pizza or
%   % pizzas should be ordered for a birthday party.
%   %
%   Much of the literature surrounding electoral systems presumes democracy via
%   representation; this has had an influence on the language and terminology used
%   to describe electoral systems. For example, a term such as ``district
%   magnitude'' implies that the seats available for candidacy is the metric being
%   measured; however, the significance of this concept applies just the same when
%   considering a vote to determine which pizza or pizzas should be ordered for a
%   birthday party.
% }

% \paragraph{Indirect Election}
% An \textbf{indirect election} is a vote where actors vote for persons who then
% elect a candidate themselves. The electoral college used in the United States
% is an example of this. When voting actors cast their vote for a particular
% candidate they're actually voting for \textbf{electors}, members of the
% electoral college who are typically aligned with the candidate's party. These
% electors are expected to vote for the candidate who wins the majority vote in a
% their state. In all but two states the electors are ``winner-take-all'', that
% is, if a candidate wins the popular vote in a state, e.g., by 50.1\%, then all
% of the electors selected for that state will be electors aligned with the
% winning candidate.
%
% \paragraph{Direct Election}
% A \textbf{direct election}, in contrast to an indirect election, is a vote
% where a voting actor's votes are cast directly for a candidate.

\paragraph{Ballot}
A ballot is the means through which voting actors express their vote. The
structure of the ballot influences exactly \emph{how} the voter can express
their preferences for a choice or choices, e.g., how many votes an individual is
able to cast. This is directly influenced by the tallying method. Ballots, in
some texts, are regarded as a process of voting; here the term is used to
describe both the medium through which a voting actor marks their choice (e.g.,
paper, punch card, or electronic machine), as well as the rules regarding how
they can mark said medium. The ballot might be thought of as the data structure
used to support the tallying algorithm.

\subparagraph{Preference Marking}
Preference marking is a ballot marking process where the electorate is given an
opportunity to rank choices: cardinally, ordinally, or by approval. This
expression of preference offers greater insights and details into the will and
desires of the electorate and can be used by the tallying method and election
officials to more optimally determine winners in an election. Preference marking
may be mandatory or optional depending on the design of the electoral system.

\paragraph{Electoral Formula}
The electoral formula is the process used to translate ballots into winning
decisions. The electoral formula is best characterized by its tallying method,
and in multi-member districts, its proportionality.

\subparagraph{Tallying Method}
The tallying method, or tallying algorithm, is closely tied to how a voter is
allowed to mark a ballot and affects how a marked ballot is ultimately counted
in the final tally of an election. Different processes of varying complexities
exist to count votes. Naturally, the process through which votes are tallied
will have a direct and significant influence on the outcome of an election.

\subparagraph{Proportionality}
Proportionality characterizes how closely votes cast translate into choices won.
The significance of proportionality becomes especially relevant as the district
magnitude increases. The closer in proportion that winning choices are to the
votes cast for them, the greater the system expresses proportionality. Analyzing
the votes cast and how they map to winning choices and ``wasted votes'' is the
easiest method to determine the proportionality of an electoral formula and is
commonly expressed as the index of disproportionality.

% \subsubsection{Conclusion}
% The choices affect \emph{who/what} you vote for, the ballot affects \emph{how}
% you express your vote, and the tallying method affects the way your expressed
% vote impacts the final results.

% \paragraph{Plurality Voting}
% A \textbf{plurality vote} system is one where

% Sub-Section: Evaluating Electoral Systems
\subsection{A Taxonomy of Electoral Systems}
Electoral systems are generally broken into three broad families based primarily
on the properties described above. The three families are plurality/majority,
proportional representation, and mixed. An overview various electoral systems is
offered in Table~\ref{tab:electoral-systems} to help visualize the landscape of
electoral systems available and where they fall.

\begin{center}
  \setstretch{1.5}
    \begin{table}[H]
      \caption{Electoral systems organized by families and characteristics.\cite{electoral-handbook}}\label{tab:electoral-systems}
        \scriptsize
        \begin{tabu} to \textwidth{@{} X[0.022l] X[1.4l] X[c] X[0.33c] X[0.33c] X[0.33c] X[c] @{}}
            \toprule
            \multicolumn{2}{l}{Electoral System}   & {Proportional Representation}     & \multicolumn{3}{c}{Plurjority}                & {Mixed} \\
                                                                                      \cmidrule{4-6}
                  &                             &                                   & Plurality     & Majority      & MMD           &  \\
            \midrule
            \taburowcolors{white..gray!15}
            STV   & Single Transferable Vote    & \textbullet{}                     &               &               &               & \\
                  & List PR                     & \textbullet{}                     &               &               &               & \\
            \addlinespace[-0.4ex]
            \cmidrule(r){1-2} \cmidrule(lr){3-3} \cmidrule(lr){4-6} \cmidrule(lr){7-7}
            \addlinespace[-0.80ex]
            FPTP  & First Past The Post         &                                   & \textbullet{} &               &               & \\
            BV    & Block Vote                  &                                   & \textbullet{} &               & \textbullet{} & \\
            PBV   & Party Block Vote            &                                   & \textbullet{} &               & \textbullet{} & \\
                  & Preferential Block Vote     &                                   &               & \textbullet{} & \textbullet{} & \\
            AV    & Alternative Vote            &                                   &               & \textbullet{} &               & \\
            RV    & Range Vote                  &                                   &               & \textbullet{} &               & \\
            TRS   & Two-Round System            &                                   & \textbullet{} & \textbullet{} &               & \\
            \addlinespace[-0.4ex]
            \cmidrule(r){1-2} \cmidrule(lr){3-3} \cmidrule(lr){4-6} \cmidrule(lr){7-7}
            \addlinespace[-0.80ex]
                  & Parallel                    & \textbullet{}                     & \textbullet{} & \textbullet{} & \textbullet{} & \textbullet{} \\
            MMP   & Mixed-Member Proportional   & \textbullet{}                     & \textbullet{} & \textbullet{} & \textbullet{} & \textbullet{} \\
            \bottomrule
            \addlinespace[\belowrulesep]
        \end{tabu}
        MMD: Multi-Member District, a district where multiple seats are
        available for candidates to hold or where multiple choices will be
        determined winners.
    \end{table}
\end{center}

\subsubsection{Plurality/Majority}
Plurality and majority electoral systems are simple in principle although not
necessarily in design. After votes have been tallied the choices with the
most resulting votes are the decided winners.

\paragraph{Plurality Voting}
A plurality voting system is one where the winning choices are the ones with the
greatest number of votes \emph{but not necessarily an absolute majority}. The
United States uses plurality electoral systems almost exclusively.

\subparagraph{First-Past-The-Post}
First-Past-the-Post (FPTP) is a plurality single-member district electoral
system used widely across the world and almost exclusively in the United States
of America. FPTP is one of the simplest electoral systems to understand; the
voter is presented with a set of choices on a ballot and is able to select
\emph{one and only one} of the choices. The ballots are then collected and
tallied by counting the number of votes cast for each choice; the choice with
the most votes wins.

% NOTE: Removed for brevity.
% Proponents of FPTP argue that it enjoys advantages which include:
% \begin{itemize}
%     \item It offers a simple design that is easy for voters to understand.
%     \item It provides a simple choice between two major parties.
%     \item It gives rise to broad encompassing parties and candidates.
%     \item It excludes extremist parties from representation.
%     \item It promotes strong geographical representation.
%     \item It allows voters to vote for individuals instead of parties.
%     \item Its design gives rise to single-party governments.
% \end{itemize}
%
% Opponents of FPTP criticize FPTP on the basis of:
% \begin{itemize}
%     \item It excludes smaller parties from offering representation.
%     \item It makes running as a minority more difficult.
%     \item It encourages political parties to form on the basis of ethnicity,
%         clan, or religion.
%     \item It makes regional fiefdoms more likely and reinforces the perception
%         of politics being defined by where one lives instead of one's ideology.
%     \item It creates a large number of wasted votes, which often disenfranchises
%         voters and candidates.
%     \item It is very dependent on the drawing of electoral boundaries.
%     \item The system can be unresponsive to changes in political opinion.
%     \item It can lead to vote-splitting, which causes two popular but similar
%         candidates to lose to a third much less popular candidate.
%     \item It forces voters to express their will in the least substantial way
%         possible – approval of just a single option while expressing nothing
%         about how much the voter likes or dislikes any choice.
%     \item It rewards the electorate for not voting honestly.
%     \item Over time, it tends towards two or fewer parties.
%     \item FPTP ballots are easy to invalidate by overvoting.
% \end{itemize}


% NOTE: Removed for brevity.
% \subparagraph{Block Vote}
% Block Vote (BV) is a plurality multi-member district voting system that operates
% much like FPTP. Voters are given as many votes as there are winning choices to
% be selected and are able to vote for as many options as they have votes.
% The elector can only submit one vote per candidate and most BV systems allow
% voters to forgo using all of their available votes. After the voting period
% ballots are collected and tallied by counting the total number of votes for each
% candidate; the choices with the most votes cast in their favor are selected as
% winners.
%
% Proponents of BV argue that it is simple and that voters retain the ability to
% vote for individual candidates. However, BV also exaggerates most of the
% disadvantages of FPTP. Further, BV can encourage party infighting by forcing
% candidates to compete against one another.
%
% \subparagraph{Party Block Vote}
% Party Block Vote (PBV) is a plurality multi-member district voting system that
% operates much like FPTP; each voter is given a single vote that can be cast for
% a party which puts forth a list of candidate options. If the party wins the
% election then the party wins all available seats for the district. The
% candidates from the candidate list presented on the ballot earn the seats.
%
% Proponents of PBV argue that it is simple while also providing a fairer and more
% diverse slate of candidates for representation. PBV still inherits most of the
% disadvantages of FPTP.

\paragraph{Majority Voting}
A majority voting system is characterized such that a choice will only be
considered the winner with an absolute majority of votes, that is, \emph{greater
than 50\% of the votes}. Most majority voting systems take advantage of
preference marking or multiple election cycles, where less-popular choices are
removed in each cycle, to form a majority.

% NOTE: Removed for brevity.
% \subparagraph{Alternative Vote}
% Alternative Vote (AV) --- also known as Instant-Runoff Voting (IRV), Ranked
% Choice Voting (RCV), transferable vote, or preferential vote --- is a majority
% single-member district voting system that depends on preference marked ballots
% to determine a majority choice. Voters mark their ballot by ranking options
% based on their preference; e.g., 1 for first choice, 2 for second choice, 3 for
% third choice, etc. Votes are summed up for each elector's first choice. If an
% option wins a majority, i.e., greater than 50\%, then that option immediately
% wins. Otherwise, a second election is simulated by determining the least popular
% option and eliminating it. Votes which were cast for the eliminated option are
% moved to the voter's second choice. This process continues until one choice has
% an absolute majority. Some AV systems require voters to rank all choices (full
% preferential voting), while others allow a partial ranking (optional
% preferential voting). One consequence of optional preferential voting is that an
% election can result in ballot exhaustion whereby no candidate can win with an
% absolute majority. Electoral system designers must decide how such edge cases
% will be handled.
%
% Proponents of AV argue that it is more likely to elect centrist options, results
% in much fewer wasted votes than minority vote alternatives, forces candidates to
% appeal to larger groups and make broader appeals, and increases the perceived
% legitimacy of the winning choice. Critics of AV point out that it is complex,
% requires an understanding of numeracy and literacy, which may disenfranchise
% electorate, and encourages dishonest voting techniques, casting ballots with
% dishonest choice-preferences to increase the likelihood of a preferred candidate
% winning.


\subparagraph{Range Vote}
Range Vote (RV), also known as Score Voting (SV), is a single-member district
majority electoral system. It takes advantage of cardinally preference marked
ballots to determine a majority choice. This form of voting allows voters to
express a preference between choices and also the degree of preference between
choices.

A range voting ballot allows voters to select a non-negative integer, up to some
maximum (usually 9 or 99), which expresses some degree of approval for a
particular option, e.g., 0 for least preferred or 9 for most preferred. Some
range voting systems also allow for disapproval voting, down to some minimum, to
express disapproval for a particular option, but this is less common. The system
may also support a ``No Opinion'' mark that can be cast if a voter is ignorant
or indifferent to some choice; this is usually considered the default mark if no
mark is made for a particular option and does not count for or against the
option. Once all ballots are cast and collected they are tallied. The tallying
process is as follows: votes for an option are summed together and divided by
the number of ballots that actively voted for that option (marked anything other
than ``No Opinion''). The option with the highest average score wins.

An approval voting system is the simplest and most restricted kind of range
voting, the range of votes is restricted to either 0 or 1. A voter can approve
as many options as they want.

% NOTE: Removed for brevity.
% Proponents of range voting argue that it has the following advantages:
%
% \begin{itemize}
%     \item It is expressive; the will and preferences of the electorate are
%           collected at a very granular and quantitative level by being able
%           to cast votes for multiple candidates, indicate a preference between
%           candidates, and express the degree of preference between those
%           candidates.
%     \item It avoids the spoiler effect, voting for candidate C does not affect a
%           battle between A and B.
%     \item There is no risk of vote-splitting.
%     \item It is less likely to produce a two-party or single-party environment
%           and encourages smaller parties to run.
%     \item Current voting machines can accommodate RV with little to no
%           modification.
%     \item There are fewer spoiled ballots because it is difficult to incorrectly
%           mark a ballot.
%     \item It is simple to understand and tally.
% \end{itemize}
%
% Critics of RV oppose it for being vulnerable to strategic voting and because
% there exists a possibility for a winner to be selected that was no voter's first
% choice.
% cite: section evaluating electoral systems (strategic voting)

% \blockquote{%
%     By avoiding ranking (and its implication of a monotonic approval reduction
%     from most- to least-preferred candidate) cardinal voting methods may solve
%     a very difficult problem:
%
%     A foundational result in social choice theory (the study of voting
%     methods) is Arrow's impossibility theorem, which states that no method can
%     comply with all of a simple set of desirable criteria. However, since one
%     of these criteria (called "universality") implicitly requires that a
%     method be ordinal, not cardinal, Arrow's theorem does not apply to
%     cardinal methods.[5][6][7][8]
%
%     Others, however, argue that this is not true, for instance because
%     interpersonal comparisons of cardinal measures are impossible.[9] If that
%     is the case, then cardinal methods do indeed fail to escape Arrow's
%     result.
%
%     Psychological research has shown that cardinal ratings are more valid and
%     convey more information than ordinal rankings in measuring human
%     opinion.[10][11][12][13]
%
%     In any case, cardinal methods do fall under the Gibbard–Satterthwaite
%     theorem, and therefore any such method must be subject to strategic voting
%     in some instances.[14][15][dubious – discuss][16]
% }

% NOTE: Removed for brevity.
% \paragraph{Two-Round System}
% The Two-Round System (TRS) is not actually a single election, but two. TRS does
% not fit cleanly into either majority or minority systems. It can be both or
% either depending on the implementation. Typically TRS elections are
% plurality-majority systems, but their format can vary widely; the first
% election, a plurality election, captures the two most popular choices, then a
% majority vote is conducted to determine the most popular choice from the
% remaining options.
%
% TRS advantages include giving voters a second chance to express their will and
% encouraging alternative candidates or proponents of other choices to put their
% weight behind a second choice, thus lessening the effect of vote-splitting.
% Unfortunately TRS elections can be prohibitively expensive and difficult to
% administer on a large scale. Further, TRS still shares many of the disadvantages
% of FPTP.


\subsubsection{Proportional Representation}
The objective of a Proportional Representation (PR) electoral system is to
produce winners from an election that accurately reflect the will of the people
and the votes they cast by reducing the disparity between shares of votes cast
for choices and shares of winning choices; this is especially relevant when
considering seats won in a representative governance models. PR operates by
providing a cross section of winners in an election which map proportionally to
the votes cast for choices. For example, if a quarter of voters in an election
desire some outcome then the election results should reflect that by producing
winners in proportion to those voters' desires; i.e., a quarter of the winners
of the election should be that of the voters' desired outcome.

% NOTE: Removed for brevity.
% A simple example, which illustrates the utility of PR electoral systems over
% plurality and majority systems, is ordering food for a group of people. For
% example, suppose it is necessary to order 10 pizzas to feed a group of people.
% If 30\% of the individuals in the group wanted pepperoni pizza and the other
% 70\% was split into 7 different groups of 10\% wanting different pizzas, it
% would not be fair to order 10 pepperoni pizzas. Likewise, even if 60\%, a
% majority, wanted pepperoni, it still would not be very fair to order 10
% pepperoni pizzas. PR systems try to address this by instead ordering 6 pepperoni
% pizzas and 4 mushroom or, in the former case, 3 pepperoni pizzas and 7 different
% pizzas that represent the desires of the people.

%%
% TODO: Insert bar graphs of the above example.
%       - Preferences
%       - Unfavorable outcome
%       - Favorable outcome
%%

% \subparagraph{Advantages}
There are a number of benefits associated with PR systems:
\begin{itemize}
    \item Votes translate into choices won with greater proportionality.
    \item Results in fewer wasted votes.
    \item Offers minority groups greater representation.
    \item Restricts regional fiefdoms.
    \item Promotes greater long-term political health and negotiation.
    \item Results in greater voter participation and perceived legitimacy.
    \item Supports a more inclusive cross-section of representatives.
\end{itemize}

% \subparagraph{Disadvantages}
There are also several disadvantages associated with PR systems:
\begin{itemize}
    \item Quick and coherent policies can be difficult to pass.
    \item Legislative gridlock can occur when factions are formed.
    \item Fragmentation of strong parties can occur in cases of extreme
        pluralism.
    \item Minority parties can hold parties ransom in coalition negotiation,
        thus offering smaller parties greater authority than perhaps deserved.
    \item It is difficult to enforce accountability by throwing out parties or
        candidates.
    \item Potentially difficult for electorate to understand or administrators
        to implement.
\end{itemize}

% \paragraph{List Proportional Representation}
% List Proportional Representation (List PR) is a party-based form of PR. Parties
% present a list of candidates and voters vote for a single party. The parties win
% some number of seats in proportion to the number of votes received and
% candidates are elected as representatives in the order they are presented on the
% ballot.
%
% Benefits of List PR include greater representation for and by minority groups
% and women.
%
% A disadvantage of List PR is that weaker links tend to be formed between
% individuals and their representatives. Further, over time, power tends to vest
% towards party leaders who can become entrenched as it is the party leaders who
% are generally responsible for selecting the candidates included in the party
% list and also the order with which candidates are to be selected for winning
% seats. Also, closed list of representatives may be put forth by the party which
% prevents the electorate from expressing their desire for a particular
% representative, who they might favor more, but does not belong to the party they
% wish to win or is simply lower in the party's candidate list. Finally, List PR,
% as a prerequisite to usage, requires that parties already exist in order to
% provide candidate lists.

\paragraph{Single Transferable Vote}
Single Transferable Vote (STV) is a PR multi-member electoral system that
operates similar to the Alternative Vote (AV). Like AV, STV leverages preference
marked ballots. STV operates as follows.
\begin{itemize*}[label={}, itemjoin={}]
  \item Electorate ordinally rank options on their ballots.
  \item Ballots are collected and tallied.
  \item While tallying, if any option reaches the quota, i.e., receives the
    minimum number of votes required, they are immediately declared a winner in
    the election and awarded a seat for the district.
  \item Surplus votes for a winning candidate are redistributed to other
    candidates based on ballot-preference.
  \item If the quota cannot be reached by any candidate then the least popular
    candidate is eliminated and votes are redistributed based on
    ballot-preference.
\end{itemize*}\cite{stv}

% \subparagraph{Quota}
There are many quota algorithm implementations which can be used in STV
elections. The quota algorithm used will determine the minimum number of votes
required to win a seat in an election. One such quota algorithm, perhaps the
most popular, is the Droop quota:\cite{electoral-handbook}
%
\begin{center}
    $Quota_{Droop} = \frac{votes}{seats+1}+1$
\end{center}
%
% NOTE: Removed for brevity.
% Another popular implementation, which produces a slightly lower threshold to
% win, is the Hare quota:
%
% \begin{center}
%     $Quota_{Hare} = \frac{votes}{seats}$
% \end{center}
%
% A less-popular quota is the Hagenbach-Bischoff quota:
%
% \begin{center}
%     $Quota_{HB} = \frac{votes}{seats+1}$
% \end{center}
%
%
% \subparagraph{Surplus Vote Allocation}
If the quota for winning is met surplus vote allocation occurs, surplus votes
are redistributed to second choices. STV implementations can differ here. It
would not necessarily be fair to select only some people to have their votes
redistributed, since voters can have different second choices. However, it might
be acceptable if surplus ballots were randomly selected for second choices. In
large enough elections with large enough surpluses the second choices should be
statistically proportional. However, non-deterministic winners are not ideal and
can lead to infinite recursion in some cases. Instead, everyone's votes (from
ballots whose candidate won) can be redistributed to their second choices (or
third, fourth, etc.\ if their second has already won) as a fraction of a vote;
this algorithm will usually be based on surplus votes, total votes, previous
votes, etc. There are many surplus allocation algorithms.\cite{electoral-handbook}
% NOTE: Removed for brevity.
% \begin{itemize}
%     \item Hare method
%     \item Cincinnati method
%     \item Hare-Clark method
%     \item Gregory method
% \end{itemize}
There are also surplus vote allocation algorithms for subsequent surplus vote
allocation, for when a second choice has already won.\cite{electoral-handbook,stv}
% NOTE: Removed for brevity.
% \begin{itemize}
%     \item Meek's method
%     \item Warren's method
%     \item The Wright System
% \end{itemize}

% \subparagraph{Excluded Candidate Vote Allocation}
When no option can win excluded candidate vote allocation occurs, the weakest
options are eliminated and their votes reallocated. There are also a number of
algorithms available when deciding how votes should be redistributed which
options are eliminated.\cite{electoral-handbook}

% NOTE: Removed for brevity.
% \begin{itemize}
%     \item Single transaction
%     \item Segmented distribution
%     \begin{itemize}
%         \item Value based segmentation
%         \item Aggregated primary vote and value segmentation
%         \item FIFO (First In First Out - Last bundle)
%     \end{itemize}
%     \item Reiterative count
% \end{itemize}

% NOTE: Removed for brevity.
% \subparagraph{Advantages}
% STV is argued to be one of the most attractive and sophisticated electoral
% systems by political scientists. STV allows for voters to choose between voting
% for parties and voting for candidates themselves. In addition, there are very
% few wasted votes, voters can maintain a geographical link to their
% representatives, and representatives are elected in proportion to the votes
% received.
%
% \subparagraph{Disadvantages}
% STV is criticized for being difficult to administrate and complex for voters.
% It is vulnerable under some implementations to tactical voting. Further, vote
% tallying must be performed at a centralized location due to constant
% recalculations and simulated elections; this centralization is concerning for
% reasons regarding election integrity. STV may cause parties to fragment
% internally due to party members running against each other.


% NOTE: Removed for brevity.
% \subsubsection{Mixed}
% The last major family of electoral systems are mixed voting systems. A mixed
% voting system is one that combines components of majority/plurality voting
% systems and proportional representation voting systems. There are two major
% forms of mixed electoral systems: parallel and mixed-member proportional. They
% might use any of the aforementioned electoral systems.

% NOTE: Removed for brevity.
% \paragraph {Parallel}
% A parallel system is one where the outcomes of the majority/plurality voting
% system and PR voting system are independent from one another.

% NOTE: Removed for brevity.
% \paragraph {Mixed-Member Proportional}
% A Mixed-Member Proportional (MMP) voting system is one where the
% majority/plurality voting system and PR voting system are dependent on another.
% Typically the PR system has some number of seats reserved which are used to
% offset any perceived unproportional election results from the majority/plurality
% election.

% \subsubsection{Summary}
% The reviewed electoral systems include only a few of the many different
% electoral systems. We have seen that there are a wide range of electoral
% systems, each of which include their own set of advantages and disadvantages and
% implementation details which can be tweaked in major or minor ways to produce a
% variety of characteristics. There are many other electoral system modalities not
% covered here which can be used to reach decisions in different contexts.
% For example, consider what kinds of electoral systems might be appropriate to
% use when deciding on solutions to complex problems, e.g., where multiple choices
% need to work in conjunction to produce a sane result. Or, where choices cannot
% easily be grouped into categories, e.g., room-temperature or tax rates.



% \subsubsection{Other Voting Systems}
% \paragraph{Borda}
% \paragraph{Limited Vote}
% \paragraph{Single Non-Transferable Vote}
% \paragraph{Comparison of Pairs of Outcomes by the Single Transferable Vote}
% \paragraph{Condorcet Method}
%
% \blockquote{%
%     Preferential voting or rank voting describes certain voting systems in which
%     voters rank outcomes in a hierarchy on the ordinal scale (ordinal voting
%     systems). When choosing between more than two options, preferential voting
%     systems provide a number of advantages over first-past-the-post voting (also
%     called plurality voting). This does not mean that preferential voting is the
%     best system; Arrow's impossibility theorem proves that no preferential
%     method can simultaneously obtain all properties desirable in a voting
%     system.[Mankiw 1][1] There is likewise no consensus among academics or
%     public servants as to the best electoral system.
% }
%
% \subsection{Other Voting System Concepts}
% \subparagraph{Ordinal Voting}
% An \textbf{ordinal vote} system is one where
%
%
% \subparagraph{Cardinal Voting}
% A \textbf{cardinal vote} system is one where
%
% \subparagraph{Proxy Voting}
% A \textbf{proxy vote} system is one where


% Sub-Section: Evaluating Electoral Systems
\subsection{Evaluating Electoral Systems}\label{sec:electoral-criteria}
Given the wide range of electoral systems that exist, or could conceivably
exist, it becomes necessary to introduce techniques that can be used to
objectively analyze and evaluate the various characteristics of electoral
systems. Social choice theory provides tools which one can use to examine and
categorize voting systems: by their advantages, their disadvantages, and
caveats.
% Voting theory demonstrates interesting and often unexpected results that can
% occur while using various electoral systems.
% What follows is an exploration into some of the voting criteria, theorems, and
% paradoxes introduced through social choice theory.

% TODO: Add this into an appendix.
% \paragraph{Sets}
% \subparagraph{Smith Set}
% \subparagraph{Landau set}

\subsubsection{Criteria}
It can be difficult to objectively judge and select an electoral system; the
choice of electoral system will have impacts on the groups, ideologies, and
candidates that are likely to succeed in a given governance model, and the
consequences of an electoral system are not always immediately obvious. In order
to compare and contrast electoral systems more objectively, there are criteria
that exist to express and describe some characteristic of an electoral system.
Table~\ref{tab:criteria-compliance} provides a summary of criteria compliance
across various electoral systems to help visualize the unique range of
compliance across various electoral systems. The criteria fall into four major
categories:

\begin{enumerate}
  \item \emph{Absolute Result Criteria},
  \item \emph{Relative Result Criteria},
  \item \emph{Ballot Counting Criteria}, and
  \item \emph{Strategy Criteria}.
\end{enumerate}


\begin{center}
  \setstretch{1.5}
    \scriptsize
    \begin{longtabu} to \textwidth{@{} X[5.85,l] X[c] X[c] X[c] X[c] X[c] X[c] X[c] X[c] X[c] X[c] X[c] X[c] X[c] X[c] @{}}
      \caption{Electoral Criteria Compliance\cite{electoral-system-analysis,electoral-handbook,electoral-system-comparison,stv-algorithms,favorite-betrayal-comparison}} \\
        \toprule
        \taburowcolors{white..white}
        Electoral Criterion    & \multicolumn{14}{c}{Electoral System} \\
                                 \cmidrule{2-15}
                               & \rot{Approval}  & \rot{Borda Count} & \rot{Copeland}           & \rot{IRV (AV)}  & \rot{Kemeny-Young} & \rot{FPTP}         & \rot{Range Voting} & \rot{Ranked Pairs} & \rot{Runoff Voting} & \rot{Schulze}   & \rot{Sortition} & \rot{Random Ballot} \\
        \midrule
        \taburowcolors{white..gray!15}
        % \endfirsthead%
        % \toprule
        % \taburowcolors{white..white}
        % Electoral Criterion & \multicolumn{14}{c}{Electoral System} \\
        %                    \cmidrule{2-15}
        % & \rot{Approval} & \rot{Borda Count} & \rot{Copeland} & \rot{IRV (AV)} & \rot{Kemeny-Young} & \rot{Majority Judgement} & \rot{Minimax} & \rot{Plurality} & \rot{Range Voting} & \rot{Ranked Pairs} & \rot{Runoff Voting} & \rot{Schulze} & \rot{Sortition} & \rot{Random Ballot} \\
        % \midrule
        % \taburowcolors{white..gray!15}
        \endhead\label{tab:criteria-compliance}
        {Majority}             & \bad[Rated]     & \vbad             & \vgood                   & \vgood          & \vgood             & \vgood             & \vbad              & \vgood             & \vgood              & \vgood          & \vbad           & \vbad \\
        {Majority Loser}       & \vbad           & \vgood            & \vgood                   & \vgood          & \vgood             & \vbad              & \vbad              & \vgood             & \vgood              & \vgood          & \vbad           & \vbad \\
        {Mutual Majority}      & \vbad           & \vbad             & \vgood                   & \vgood          & \vgood             & \vbad              & \vbad              & \vgood             & \vbad               & \vgood          & \vbad           & \vbad \\
        {Condorcet}            & \bad            & \vbad             & \vgood                   & \vbad           & \vgood             & \vbad              & \bad               & \vgood             & \vbad               & \vgood          & \vbad           & \vbad \\
        {Condorcet Loser}      & \vbad           & \vgood            & \vgood                   & \vgood          & \vgood             & \vbad              & \vbad              & \vgood             & \vgood              & \vgood          & \vbad           & \vbad \\
        {Smith/IDSA}           & \vbad           & \vbad             & \vgood                   & \vbad           & \vgood             & \vbad              & \vbad              & \vgood             & \vbad               & \vgood          & \vbad           & \vbad \\
        {LIIA}                 & \good           & \vbad             & \vbad                    & \vbad           & \vgood             & \vbad              & \good              & \vgood             & \vbad               & \vbad           & \vgood          & \vgood \\
        {IIA}                  & \good           & \vbad             & \vbad                    & \vbad           & \vbad              & \vbad              & \good              & \vbad              & \vbad               & \vbad           & \vgood          & \vgood \\
        {Cloneproof}           & \good           & \vbad[\teams]     & \vbad[\crowds{}\teams{}] & \vgood          & \vbad[\spoilers]   & \vbad[\spoilers]   & \vgood             & \vgood             & \vbad[\spoilers]    & \vgood          & \vbad           & \vgood \\
                                 \addlinespace[-0.4ex]
                                 \cmidrule(l){2-15}
                                 \addlinespace[-0.80ex]
        {Monotone}             & \vgood          & \vgood            & \vgood                   & \vbad           & \vgood             & \vgood             & \vgood             & \vgood             & \vbad               & \vgood          & \vgood          & \vgood \\
        {Consistency}          & \good           & \vgood            & \vbad                    & \vbad           & \bad               & \vgood             & \good              & \vbad              & \vbad               & \vbad           & \vgood          & \vgood \\
        {Participation}        & \good           & \vgood            & \vbad                    & \vbad           & \vbad              & \vgood             & \good              & \bad               & \vbad               & \bad            & \vgood          & \vgood \\
        {Reversal Symmetry}    & \vgood          & \vgood            & \vgood                   & \vbad           & \vgood             & \vbad              & \vgood             & \vgood             & \vbad               & \vgood          & \vgood          & \vgood \\
                                 \addlinespace[-0.4ex]
                                 \cmidrule(l){2-15}
                                 \addlinespace[-0.80ex]
        {Polytime}             & \vgood[$O(n)$]  & \vgood[$O(n)$]    & \good[$O(n^2)$]          & \good[$O(n^2)$] & \vbad[$O(n!)$]     & \vgood[$O(n)$]     & \vgood[$O(n)$]     & \good[$O(n^3)$]    & \vgood[$O(n)$]      & \good[$O(n^3)$] & \vgood[$O(1)$]  & \vgood[$O(n)$] \\
        {Resolvable}           & \vgood          & \vgood            & \vbad                    & \good           & \vgood             & \vgood             & \vgood             & \vgood             & \vgood              & \vgood          & \vbad           & \vbad \\
        {Summable}             & \vgood[$O(n)$]  & \vgood[$O(n)$]    & \good[$O(n^2)$]          & \vbad[$O(n!)$]  & \good[$O(n^2)$]    & \vgood[$O(n)$]     & \vgood[$O(n)$]     & \good[$O(n^2)$]    & \good[$O(n)$]       & \good[$O(n^2)$] & \vgood[$O(1)$]  & \vgood[$O(n)$] \\
                                 \addlinespace[-0.4ex]
                                 \cmidrule(l){2-15}
                                 \addlinespace[-0.80ex]
        {Later-no-Harm}        & \vbad           & \vbad             & \vbad                    & \vgood          & \vbad              & \meh               & \vbad              & \vbad              & \good               & \vbad           & \vgood          & \vgood \\
        {Later-no-Help}        & \good           & \vgood            & \vbad                    & \vgood          & \vbad              & \meh               & \vgood             & \vbad              & \good               & \vbad           & \vgood          & \vgood \\
        {No Favorite Betrayal} & \vgood          & \vbad             & \vbad                    & \vbad           & \vbad              & \vbad              & \vgood             & \bad               & \vbad               & \bad            & \vgood          & \vgood \\
                                 \addlinespace[-0.4ex]
                                 \cmidrule(l){2-15}
                                 \addlinespace[-0.80ex]
        {Ballot Type}          & \bad[\approval] & \good[\ranked]    & \good[\ranked]           & \good[\ranked]  & \good[\ranked]     & \vbad[\singleMark] & \vgood[\range]     & \good[\ranked]     & \vbad[\singleMark]  & \good[\ranked] & \meh             & \vbad[\singleMark] \\
        {Ranks =}              & \vgood          & \vbad             & \vgood                   & \vbad           & \vgood             & \meh               & \vgood             & \vgood             & \meh                & \vgood         & \meh             & \meh \\
        {Ranks >2}             & \vbad           & \vgood            & \vgood                   & \vgood          & \vgood             & \vbad              & \vgood             & \vgood             & \bad                & \vgood         & \meh             & \vbad \\
        \bottomrule
    \end{longtabu}
    \begin{multicols}{2}
      \begin{itemize}[label=,leftmargin=0mm,topsep=0mm]
        \item \cmark{} Criterion is supported.
        \item \xmark{} Criterion is unsupported.
        \item \notApplicable{} Criterion is not applicable.
        \item \qmark{} Criterion is supported under some conditions.
        \item \range{} Choices are cardinally ranked by preference.
        \item \ranked{} Choices are ordinally ranked by preference.

        \item \spoilers{} Vulnerable to spoilers.
        \item \teams{} Vulnerable to teams.
        \item \crowds{} Vulnerable to crowds.
        \item \approval{} Multiple yes/no choice-selections supported.
        \item \singleMark{} Single yes/no choice-selection supported.
        \item
      \end{itemize}
    \end{multicols}
\end{center}

\paragraph{Absolute Result Criteria}
The absolute criteria express whether a candidate must or must not win given the
state of some ballots.\cite{range-voting-criteria,electoral-system-comparison,electoral-system-comparison-archive}

\subparagraph{Majority Criterion}
The Majority Criterion (MC) states that a candidate who is preferred by a
majority of voters must win. This is expressed in two flavors:
\begin{enumerate}
    \item \emph{Ranked}, when a choice is preferred by a majority of voters,
        the choice must win.
    \item \emph{Rated}, when a choice is given a perfect score by a majority of
        voters, the choice must win.
\end{enumerate}
In a ranked electoral systems these two majority criteria are identical in
nature.

\subparagraph{Mutual Majority Criterion}
The Mutual Majority Criterion (MMC) states that if a subset, \verb|S|, of
candidates is strictly preferred over every candidate in the absolute complement
of subset \verb|S|, then the winner must come from subset \verb|S|.

\subparagraph{Condorcet Criterion}
The Condorcet criterion states that a choice who beats out every other choice in
a pairwise comparison will win.

\subparagraph{Condorcet Loser Criterion}
The Condorcet loser criterion states that a choice who loses to every other
choice in a pairwise comparison will always lose.

\paragraph{Relative Result Criteria}
The relative result criteria express when a candidate should or should not win
given a win in a similar circumstance.\cite{electoral-system-comparison,electoral-system-comparison-archive,range-voting-criteria}

\subparagraph{Independence of Smith-dominated Alternatives}
Independence of Smith-dominated Alternatives (ISDA) states that an added or
removed Smith-dominated choice, one which would lose in direct pairwise
competition with every other choice, will not affect the result of the contest.

% Does the outcome never change if a Smith-dominated candidate is added or
% removed (assuming votes regarding the other candidates are unchanged)?
% Candidate C is Smith-dominated if there is some other candidate A such that C
% is beaten by A and every candidate B that is not beaten by A etc. Note that
% although this criterion is classed here as nominee-relative, it has a strong
% absolute component in excluding Smith-dominated candidates from winning. In
% fact, it implies all of the absolute criteria above.

\subparagraph{Independence of Irrelevant Alternatives}
Independence of Irrelevant Alternatives (IIA) is a criterion which states that
adding or removing a non-winning candidate should not impact the end result.

\subparagraph{Local Independence of Irrelevant Alternatives}
Local Independence of Irrelevant Alternatives (LIIA) is a criterion which states
that removing a candidate will not disrupt the transitive ordering.

\subparagraph{Independence of Clone Alternatives}
The independence of clone alternatives, cloneproof, states that the outcome will
not change if non-winning candidates similar to an existing candidate are added
as choices. There are three flavors which fail independence of clones:
\begin{enumerate}
    \item \emph{Spoilers}, which are clone negative choices that decrease the
      chance of a similar choice winning by spreading votes across multiple
      choices.
    \item \emph{Teams}, which are clone positive similar choices that together
        increase the chance of their winning.
    \item \emph{Crowds}, which are non-winning choices that when cloned change
        the winner without themselves becoming the winner.
\end{enumerate}

\subparagraph{Monotonicity Criterion}
The monotonicity criterion, monotone, states that ranking a winning choice
higher will not impact the end result.

\subparagraph{Consistency Criterion}
The Consistency Criterion (CC) states that a winning choice in two complement
sets of ballots should remain the winner in a final tally combining the two
sets.

\subparagraph{Participation Criterion}
The Participation Criterion (PC) states that voting honestly will always produce
better results than not voting at all.

\subparagraph{Reversal Symmetry}
The reversal symmetry criterion states that when individual voter preferences
are universally inverted the original winner will never win.

\paragraph{Ballot Counting Criteria}
Ballot counting criteria concern the process of vote tallying and winner
determination. These criteria are especially important with respect to the
practical implementation of an electoral system.\cite{electoral-system-analysis,electoral-handbook,pr-computability}

\subparagraph{Polynomial Time}
The polynomial time criterion, polytime, states that the electoral system can
calculate the winner in linear time with respect to the number of voters and in
polynomial time with respect to the number of candidates.

\subparagraph{Resolvable}
A resolvable electoral system is one where determining a winner should be
entirely deterministic, i.e., the electoral system should not depend on random
processes such as coin flipping. This criterion is less important in large
elections where ties are unlikely to occur.

\subparagraph{Summability}
Summability is a criterion used to express how computationally complex it is to
store vote data in a compressed format, and consequentially, how difficult it is
to pre-tally votes at individual polling stations and transmit those tallied
results to a central counting authority for final counting. Votes are expected
to be mapped to a summable array which can be used to determine the winner. The
summability criterion is considered $k$\textsuperscript{th}-order summable if we
can map $n$ candidates to a matrix of size $n^k$. If no $k$ exists the electoral
system is considered non-summable.

\paragraph{Strategy Criteria}
The strategy criteria relate to the incentives and ability for voters to vote
using some strategy to produce a desired election result.\cite{electoral-system-analysis,stv-strategic-voting,electoral-handbook}

\subparagraph{Later-no-Harm Criterion}
This criterion states that ranking a preference later on a ballot will not harm
a choice already listed.

\subparagraph{Later-no-Help Criterion}
This criterion states that ranking a preference later on a ballot will not help
a choice already listed.

\subparagraph{No Favorite Betrayal}
No Favorite Betrayal (NFB) states that ranking a choice above your preferred
choice will not produce a more desirable or preferred result.

\paragraph{Ballot Format}
The ballot formats define how a voter is able to express themselves on a ballot.\cite{electoral-handbook}

\subparagraph{Ballot Type}
The ballot type defines how a voter is permitted to mark their ballot. Popular
ballot types include single mark, approval, ranked (ordinal), and scored
(cardinal).

\subparagraph{Equal Ranks}
A ballot that allows a voter to express equal support for multiple candidates is
said to support equal ranks.

\subparagraph{Over 2 Ranks}
A ballot which allows a voter to express interest for a choice in non-binary
terms is said to support over 2 ranks, e.g., ordinal and cardinal ballots.

% Sub-Sub-Section: Theorems and Paradoxes
% \subsubsection{Theorems and Paradoxes}
There are a number of interesting theorems and paradoxes regarding voting. These
theorems illustrate some of the practical limitations of what electoral systems
can accomplish and are capable of.

\paragraph{Condorcet's Paradox}
When considering governance and voting, Condorcet's paradox is a good place to
start; it grounds one to the difficulties of voting. Condorcet's paradox
demonstrates that the collective preferences of voting actors can be cyclical
despite the individual preferences of the voting actor's choices being
transitive and non-cyclical. The following example illustrates Condorcet's
paradox:

Suppose we have 3 voting actors: $A$, $B$, and $C$ voting for choices $X$, $Y$,
and $Z$. They rank their votes sequentially from 1 to 3 (lower numbers
indicating a more favorable choice).

\begin{center}
    \begin{table}[H]
        \centering\scriptsize
        \caption{A cyclic voter preference profile.} \label{tab:condorcet-paradox}
        \begin{tabu} to 30mm{@{} X[c] | X[c] X[c] X[c] @{}}
            \toprule
            & $X$ & $Y$ & $Z$ \\
            \midrule
            $A$ & 1 & 2 & 3 \\
            $B$ & 2 & 3 & 1 \\
            $C$ & 3 & 1 & 2 \\
            \bottomrule
        \end{tabu} \\[2mm]
        \begin{varwidth}{\textwidth}
            \begin{itemize}[label=,leftmargin=0mm,topsep=5mm]
                \item $A$ favors $X > Y > Z$
                \item $B$ favors $Z > X > Y$
                \item $C$ favors $Y > Z > X$
            \end{itemize}
        \end{varwidth}
    \end{table}
\end{center}


When measuring the preferences of the voters collectively we observe that choice
$X$ is preferred to $Y$, that choice $Y$ is collectively preferred to $Z$, and
that choice $Z$ is collectively preferred to $X$; a seemingly paradoxical result
($X > Y$ and $Y > Z$ and $Z > X$).

\paragraph{Arrow's Impossibility Theorem}
Arrow's impossibility theorem, also known as Arrow's paradox, states that no
rank-order voting method can satisfy the following ``fairness'' conditions:

\begin{enumerate}
    \item There is no dictator.
    \item If every voter prefers $X$ to $Y$ then so does the group.
    \item The relative positions of $X$ and $Y$ in the group ranking depend on
        their relative positions in the individual rankings, but do not
        depend on the individual rankings of any irrelevant alternative $Z$.
\end{enumerate}

The following example demonstrates Arrow's paradox:

\begin{center}
    \begin{table}[H]
        \centering\scriptsize
        \caption{A voter preference profile demonstrating Arrow's paradox.}
        \label{tab:arrows-paradox}
        \begin{tabu} to 30mm{@{} X[c] | X[c] X[c] X[c] X[c] @{}}
            \toprule
            & $X$ & $Y$ & $Z$ \\
            \midrule
            A & 1 & 2 & 3 \\
            B & 1 & 3 & 2 \\
            C & 3 & 1 & 2 \\
            D & 3 & 1 & 2 \\
            E & 3 & 2 & 1 \\
            \bottomrule
        \end{tabu} \\[2mm]
        \begin{varwidth}{\textwidth}
            \begin{itemize}[label=,leftmargin=0mm,topsep=5mm]
                \item A favors $X > Y > Z$
                \item B favors $X > Z > Y$
                \item C favors $Y > Z > X$
                \item D favors $Y > Z > X$
                \item E favors $Z > Y > X$
            \end{itemize}
        \end{varwidth}
    \end{table}
\end{center}

After tallying votes, summing each voter's first-preference, we see that $X$ has
2 votes, $Y$ has 2 votes, and $Z$ has 1 vote. $X$ and $Y$ are tied as first
choice while $Z$ is an irrelevant alternative. One might expect that if $Z$ were
eliminated from the ballot that the election result would not be affected.
However, once done we observe that the election's outcome changes:

\begin{center}
    \begin{table}[H]
        \centering\scriptsize
        \caption{Table \ref{tab:arrows-paradox} with the irrelevant alternative removed.}
        \label{tab:arrows-paradox-remove-alt}
        \begin{tabu} to 30mm{@{} X[c] | X[c] X[c] @{}}
            \toprule
            & $X$ & $Y$ \\
            \midrule
            A & 1 & 2 \\
            B & 1 & 3 \\
            C & 3 & 1 \\
            D & 3 & 1 \\
            E & 3 & 2 \\
            \bottomrule
        \end{tabu} \\[2mm]
        \begin{varwidth}{\textwidth}
            \begin{itemize}[label=,leftmargin=0mm,topsep=5mm]
                \item A favors $X > Y$
                \item B favors $X > Y$
                \item C favors $Y > X$
                \item D favors $Y > X$
                \item E favors $Y > X$
            \end{itemize}
        \end{varwidth}
    \end{table}
\end{center}

$Y$ is now favored over $X$ with 3 votes to 2 votes.

\begin{displayquote}[Kenneth Arrow]
  Most systems are not going to work badly all of the time. All I proved is that
  all can work badly at times.
\end{displayquote}

\paragraph{Gibbard–Satterthwaite Theorem}
The Gibbard–Satterthwaite Theorem is related to Arrow's Impossibility Theorem
and states the following:

\begin{displayquote}
    No single-winner voting method exists (using rank-order ballots) satisfying
    all of the following short list of conditions:

    \begin{enumerate}
        \item There is no ``dictator.''
        \item If every voter ranks $X$ top, then $X$ wins the election.
        \item The voting system is deterministic, i.e. its decision about who
            wins is based purely on the votes, not on random chance.
        \item There are at least three candidates running.
        \item Honest and strategic voting are the same thing, i.e. it never
            ``pays for a voter to lie,'' i.e. (more precisely) there is no
            election situation in which a voter, by submitting a dishonest vote
            claiming $X > Y$ when really she does not agree that $X$ is a better
            candidate than $Y$, can make the election result come out better
            (from her point of view) than if she had voted honestly.
    \end{enumerate}
\end{displayquote}

In other words, any reasonable voting system which leverages rank-order ballots
will always be susceptible to manipulation and strategic voting. Consider the
following example:

\begin{center}
    \begin{table}[H]
        \centering\scriptsize
        \caption{A preference profile demonstrating the Gibbard-Satterthwaite theorem.}
        \label{tab:gh-theorem}
        \begin{tabu} to 30mm{@{} X[c] | X[c] X[c] X[c] X[c] @{}}
            \toprule
            & $X$ & $Y$ & $Z$ \\
            \midrule
            5 & 1 & 2 & 3 \\
            6 & 2 & 3 & 1 \\
            8 & 3 & 1 & 2 \\
            \bottomrule
        \end{tabu} \\[2mm]
        \begin{varwidth}{\textwidth}
            \begin{itemize}[label=,leftmargin=0mm,topsep=5mm]
                \item 5 voters favor $X > Y > Z$
                \item 6 voters favor $Z > X > Y$
                \item 8 voters favor $Y > Z > X$
            \end{itemize}
        \end{varwidth}
    \end{table}
\end{center}

Under most electoral systems choice $Y$ would be the winner. However, if the 6
voters who favored $Z$ changed their votes to $X > Z > Y$, betraying $Z$, then a
choice they may consider a lesser evil, $X$, will win.



% Sub-Section: Processes, Procedures, Components, and Risks
% \subsection{Processes, Procedures, Components, and Risks}
\todo{Are these just the non-functional requirements?}
In practice implementing a voting system can be a massive undertaking with
tremendous consequences. For example, elections in the US dictate how and what
policies will be implemented over the next several years. As such, there are
serious risks and concerns that must be considered when attempting to build such
a system. First and foremost, an implemented voting system should protect the
voting actors as much as possible: from physical and social repercussions,
bribery, coercion, intimidation, Second, electoral systems should ensure that
the voting process itself is secure, i.e., resistant to corruption and
manipulation. Finally, an implemented voting system must be scalable
(financially, temporally, geographically, etc.) in order to be effective in
large-scale elections. This material offers background on:

\begin{itemize}
  \item \emph{Voting Processes},
  \item \emph{Physical Components}, and
  \item \emph{Risks}.
\end{itemize}

\subsubsection{Voting Processes and Procedures}
In practice, voting systems require \emph{processes and procedures} which enable
voting actors to cast votes and express their preference. The voting process can
generally be understood as being comprised of four distinct components:

\begin{itemize}
  \item \emph{Registration},
  \item \emph{Verification, Authentication, and Authorization},
  \item \emph{Vote Casting}, and
  \item \emph{Collection and Processing}.
\end{itemize}

\paragraph{Registration}
The registration component of the voting process is how the collection
of eligible voters is established. The registration process varies from state to
state in the US.\ Some states allow \emph{election day registration}, a process
whereby one can register to vote in major federal elections on the same day you
vote; however, most states require their electorate to register themselves
before the election in order to have enough time to disseminate the collection
of eligible voters to polling station officials. Several states require that the
voting actor register up to 30 days before the day of the vote. Currently only
five states in the US have automatic voter registration, or ``opt-out''
registration, a process by which citizens of that state are automatically
granted voting rights in upcoming elections if they meet eligible voter
requirements, e.g., are over the age of 18.

\paragraph{Verification, Authentication, and Authorization}
The \emph{verification} component of a vote is the process by which a voting
actor provides some form of identification in order to prove they are who they
claim to be. The identity is then \emph{authenticated} via some data source by a
polling station. After the polling station has confirmed that the voting actor
is who they claim to be and is also registered to vote, the polling place will
\emph{authorize} the voting actor to cast their vote. These combined processes
prevent bad actors from manipulating the electoral system and affecting the
outcome of the election.

% These steps combined with the registration process prevent what is known as
% Sybil attacks in peer-to-peer networks, an attack on a system by creating
% false identities to gain influence over a system.

\paragraph{Vote Casting}
Once a voter has moved through the verification, authentication, and
authorization processes they are eligible to cast their vote. The voting actor
is provided a ballot to mark their preferences then able to submit their vote.
The US does not have a consistent vote casting process across its states; some
states use paper ballots, others use mechanical voting machines, and others use
electronic machines. Special precautions must be taken here to ensure that votes
are cast securely and privately.

\paragraph{Collection and Processing}
The final step of the voting process is the collection and processing component.
This step requires that the polling places aggregate and tally all of the votes
to determine a final result. Records must be kept that would provide a means to
audit and recount votes if necessary. In electronic machines the collection and
tallying process would be done by reading a memory. Mechanical machines via a
mechanical tabulation device. Paper ballots, perhaps the most primitive, require
either scanning technology or hand counting.


%%
% TODO: This section is WEAK.
%%
\subsubsection{Fundamental Concepts and Components}
The physical components leveraged within the voting process are critical to
maintaining election security, privacy, reliability, and faith in the system.

\paragraph{Ballot}
The ballot is the physical mechanism presented to the voter to cast their vote.
How the ballot is presented and cast is an important part of securing the voting
system and protecting the voter.

\subparagraph{Secret Ballot}
A secret ballot is a ballot which anonymizes a voting actor's choice. This is
done for several reasons, namely to prevent voter intimidation and vote buying.
Secret ballots are a right provided by several treaties: the \emph{Universal
Declaration of Human Rights}, the \emph{American Convention on Human Rights},
and the \emph{Convention on the Standards of Democratic Elections, Electoral
Rights and Freedoms}. Secret ballots are one of the most important rights
offered by democratic societies. Secret ballots allow voters to vote on topics
without fear of retaliation or outcast by the rest of society. However, the term
secret ballot is somewhat of a misnomer; a secret ballot is generally a normal
ballot that is cast in such a way that it would not be possible to know who cast
the ballot. For example, dropping a paper ballot into a sealed guarded box.
Voting booths, small rooms with curtains or desks with privacy barriers in place
to protect a voter's privacy, are another example of privacy maintain devices
and used almost ubiquitously around the world.

While there are many benefits to secret ballots secret ballots are not always
necessary or useful to have during voting processes. For example, if a unanimous
vote were required on a topic it might be helpful for individuals to make their
opinions public to promote discussion. Another example would be if a
representative were voting on a citizen's behalf, for the sake of transparency
an open ballot would be more appropriate.

\paragraph{Electronic Voting Machine}
An electronic voting machine is an electronic device which presents voters with
their ballot.

\paragraph{Voting Booth}
A voting booth is a safe, private, protected place for voters to mark and cast
their ballot.

\paragraph{Tallying Machines}
Tallying machines offer fast and efficient mechanism for tallying the results of
an election.

\paragraph{Ballot Box}
A secure location for voters to cast their vote after marking their ballot.

% \begin{displayquote}
%   These universal democratic principles can be summarized as a list of
%   fundamental requirements, or ``six commandments,'' for electronic voting
%   systems[citation needed]:
%
%   \begin{enumerate}
%     \item Keep each voter's choice an inviolable secret.
%     \item Allow each eligible voter to vote only once, and only for those
%           offices for which he/she is authorized to cast a vote.
%     \item Do not permit tampering with the voting systems operations, nor
%           allow voters to sell their votes.
%     \item Report all votes accurately
%     \item The voting system shall remain operable throughout each election.
%     \item Keep an audit trail to detect any breach of [2] and [4] but
%           without violating [1].
%   \end{enumerate}
% \end{displayquote}

\subsubsection{Risks}
There are a number of risks which must be managed when conducting elections:

\begin{itemize}
  \item election fraud: machine rigging, bribing officials, etc.;
  \item voter fraud: vote buying, ballot stuffing, intimidation, etc.
\end{itemize}

These risks are addressed in greater detail in chapter \ref{chap:literature},
\emph{\nameref{chap:literature}}.


% Sub-Sub-Section: Voting Systems
% \subsection{Voting Systems}
\todo{Requirements}
\todo{Procedures}
\todo{Processes}

