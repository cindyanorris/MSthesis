\subsection{Ethereum}

Ethereum is a blockchain-based system initially proposed in late 2013,
post-Bitcoin, and released in 2015.\cite{ethereum-white} Ethereum introduced a
few novel features and functionalities with its blockchain network which are not
available in the Bitcoin network. Most notably Ethereum offers a distributed
decentralized Turing-complete computing platform via the Ethereum Virtual
Machine (EVM), which aims to provide an application layer which ``[runs] exactly
as programmed without any possibility of downtime, censorship, fraud or third
party interference.''\cite{ethereum.org,ethereum-yellow}

\subsubsection{Fundamental Concepts}

\paragraph{Currency}
Like Bitcoin, Ethereum's blockchain defines and leverages its own token,
\solt{ether} --- also referred to as \solt{eth} or ETH, and sometimes denoted
using the Greek symbol Xi, $\Xi$, the uppercase Old English letter Eth, \DH, or,
more rarely, $\blacklozenge$ --- which acts as the underlying currency of its
blockchain protocol. The smallest unit of \solt{ether} is
\emph{wei}.~\cite{mastering-ethereum,ethereum-yellow} The
denominations of \solt{ether} are broken down as follows:
% \ref{book:mastering-ethereum}
% \ref{paper:ethereum}

% \todo{Add denomination table here}

% \ref{book:mastering-ethereum}
% \ref{paper:ethereum}
%
% See page 2 of Ethereum yellow paper.
%
% Table 2-1 from page 14 of Mastering Ethereum.
%
% | Value (in wei) | Exponent | Common Name | SI Name               |
% |----------------|----------|-------------|-----------------------|
% | 1              | 1        | wei         | Wei                   |
% | 1000           | 10^3     | Babbage     | Kilowei or femroether |
% |                |          |             |                       |
% |                |          |             |                       |

\begin{center}
    \setstretch{1.5}
    \footnotesize
    \begin{longtabu} to \textwidth{@{} X[3] X[] X[] X[2] @{}}
    \caption{Ether Denominations~\cite{mastering-ethereum}}\label{tab:eth-denominations} \\
    \toprule
    Value (in wei)                    & Exponent & Common Name & SI Name \\
    \midrule
    \endfirsthead
    \toprule
    Value (in wei)                    & Exponent & Common Name & SI Name \\
    \midrule
    \endhead
    \taburowcolors{white..gray!15}
    1                                 & $10^{0}$    & wei       & Wei \\
    1,000                             & $10^{3}$    & Babbage   & Kilowei or femtoether \\
    1,000,000                         & $10^{6}$    & Lovelace  & Megawei or picoether \\
    1,000,000,000                     & $10^{9}$    & Shannon   & Gigawei or nanoether \\
    1,000,000,000,000                 & $10^{12}$   & Szabo     & Microether or micro \\
    1,000,000,000,000,000             & $10^{15}$   & Finney    & Milliether or milli \\
    1,000,000,000,000,000,000         & $10^{18}$   & Ether     & Ether \\
    1,000,000,000,000,000,000,000     & $10^{21}$   & Grand     & Kiloether \\
    1,000,000,000,000,000,000,000,000 & $10^{24}$   &           & Megaether \\

    %extern u256 const c_copyGas;			///< Multiplied by the number of 32-byte words that are copied (round up) for any *COPY operation and added.
    \bottomrule
    % \begin{varwidth}{\textwidth}
    %   These names pay homage to some of the great minds of cryptography and
    %   computer science.
    % \end{varwidth}
  \end{longtabu}
\end{center}

\paragraph{Accounts}
Accounts are an Ethereum primitive which provide an abstraction over the Bitcoin
equivalent signature chain process; this abstraction helps to both simplify the
concept of token ownership as well as extend the idea of what a token is, what a
token can be, and how blockchain state can be managed and organized.

Accounts are identified by a 160-bit code, their \solt{address}, and are
internally represented by four properties:

\begin{enumerate}
  \item \sol{nonce}, a monotonically increasing counter which represents the
    number of transactions sent from the account.
  \item \sol{balance}, the amount of \solt{ether}, expressed in wei, which is
    owned by the account.
  \item \sol{storageRoot}, a 256-bit hash of the root node of a Merkle Patricia
    tree which encodes the storage contents of the account.

    % RLP - Recursive Length Prefix
    % https://eth.wiki/en/fundamentals/rlp

    % https://blog.ethereum.org/2015/11/15/merkling-in-ethereum/
    %
    % From the yellow paper:
    % \displayquote{
    %   A 256-bit hash of the root node of a Merkle Patricia tree that encodes
    %   the storage contents of the account (a mapping between 256-bit integer
    %   values), encoded into the trie as a mapping from the Keccak 256-bit hash
    %   of the 256-bit integer keys to the RLP-encoded 256-bit integer values.
    %   This tree encodes the hash of the storage contents of this account, and
    %   is empty by default.
    % }


  \item \sol{codeHash}, an immutable hash of the EVM code corresponding to the
    account.
\end{enumerate}

Although all accounts are structurally identical it is useful to distinguish
between the two practical kinds of accounts which one is likely to encounter and
interact with on the Ethereum blockchain, \emph{external accounts} and
\emph{internal accounts}:
%
\begin{description}[font=\normalfont\emph]
  % \subsubparagraph{External Accounts}
  \item[External Accounts] --- also referred to as simple accounts, non-contract
    accounts, externally owned accounts (EOA), and sometimes user accounts ---
    are defined as accounts whose \sol{codeHash} value is the Keccak-256 hash of
    an empty string; i.e., the account contains no code.

  %\subsubparagraph{Internal Accounts}
  \item[Internal Accounts] --- also referred to as contract accounts --- are
    those accounts which are not external accounts; i.e., the account contains
    code.
\end{description}
%
Both kinds of accounts have the ability to send and receive \solt{ether} as well
as interact with \solt{contract}[s] which have been deployed to the Ethereum
network. However, there are some key differences between the two kinds of
accounts which are worth highlighting:\cite{ethereum:accounts}

\begin{itemize}[label=]
  \item External Accounts
    \begin{itemize}[label=$\bullet$]
      \item External accounts are managed by public-key cryptography.
      \item External account creation costs no \solt{ether}.
      \item Only external accounts can \emph{initiate} transactions.
      \item Transactions between external accounts can only transact
        \solt{ether}.
    \end{itemize}
  \item Internal Accounts
    \begin{itemize}[label=$\bullet$]
      \item Internal accounts are managed by code.
      \item Internal account creation costs \solt{ether} which reflects the cost
        of storing code on the Ethereum network.
      \item Internal accounts can execute code via the EVM upon receiving
        transactions, enabling a wide range network functionalities.
      \item Internal accounts can only send transactions in response to
        receiving transactions. \todo{This will have a major impact on the
        methodologies.}
    \end{itemize}
\end{itemize}

\paragraph{Transactions}
A transaction is a cryptographically-signed instruction constructed by an
external actor and submitted to the Ethereum network. There are two kinds of
transactions worth distinguishing, contract creation transactions and message
call transactions; both kinds of transactions share the following common
properties:

\begin{enumerate}
  \item \sol{to}, a 160-bit \solt{address} representing the \emph{recipient's}
    account. This value is omitted when building a contract creation
    transaction.

  \item \sol{from}, a signature which identifies the \emph{sender} of the
    transaction by account \solt{address}.\footnotemark{}

    \footnotetext{
      This field does not \emph{technically} exist, in actuality the signature
      is represented as three distinct fields (\emph{v}, \emph{r}, \emph{s}),
      which can be used to determine the \solt{address} representing the sender
      of the transaction.
    }

  \item \sol{value}, the amount of \solt{ether}, expressed in wei, which is
    to be transferred to the recipient.

  \item \sol{nonce}, a value equal to the number of transactions which have been
    sent by the \emph{sender}.

  \item \sol{gasPrice}, a value representing the amount of wei to be paid per
    unit of gas (expanded on below).

  \item \sol{gasLimit}, a value representing the maximum amount of gas which
    should be used executing the transaction.
\end{enumerate}

Contract creation transactions include the following additional property in the
transaction:

\begin{enumerate}
  \item \sol{init}, an EVM-code fragment which is executed only once and
    discarded thereafter; it returns the \sol{body}, a second fragment of code
    that executes each time the account receives a message call, which can occur
    either by transaction or internal execution of code.
\end{enumerate}

In contrast, a message call transaction includes the following additional
property:

\begin{enumerate}
  \item \sol{data}, an unlimited size byte array which contains the input data
    of the message call.
\end{enumerate}

\paragraph{Ethereum Virtual Machine}\label{eth:evm}
The Ethereum Virtual Machine (EVM) is the execution environment which Ethereum
code is processed in. The EVM processes low-level bytecode and takes actions
against the state of the Ethereum blockchain in response: reading, processing,
and writing data. The Ethereum Virtual Machine Specification introduces a
low-level instruction set which defines the available operations which an EVM
implementation should support: the opcodes, their inputs, outputs, and various
other implementation details. The EVM can be described as a \emph{state
transition function}, $Y(S, T) = S'$; given a set of transactions, $T$, and an
initial state, $S$, the state transition function, $Y(S, T)$, will produce a new
output state, $S'$.\cite{ethereum-yellow} Several EVM implementations exist
which have been written in various languages, e.g., Go, Python, and C++.

\subparagraph{Language Support}\label{eth:languages}
Several higher-level languages exist which target the Ethereum Virtual Machine
and can be used to build ``smart contracts;'' e.g., Solidity which draws
inspiration from C++ and JavaScript, and Vyper which describes itself as a
``Pythonic Smart Contract Language.'' These languages ship with compilers which
can be used to translate their code into low-level EVM bytecode; Solidity, for
example, is compiled using \solt{solc}, the Solidity compiler.

% \subsubsection{Network Topology}
\subsubsection{Network Topology}
Like Bitcoin, Ethereum exists as a network of nodes, each node supporting
differing functionalities, which work collectively to construct the Ethereum
blockchain and support the surrounding ecosystem. These nodes can be classified
by the functionalities which they support.

\paragraph{Node Types}
There exists several implementations of the Ethereum protocol, i.e., Ethereum
clients, which have been written in various languages, e.g., Geth in Go, Parity
in Rust, and pyethereum in Python. Some of these clients support running in
different modes.

\begin{itemize}
  \item A \emph{full node} is a complete implementation of the Ethereum
    protocol. A full nodes processes and validates all transactions which have
    been added to the Ethereum blockchain, thus helping to support the
    resiliency and reliability of the network. To support this functionality, a
    full node must maintain a complete copy of the blockchain. Full nodes are
    also capable of deploying and interacting with contracts, support mining and
    wallet functionality, and are able to route transactions throughout the
    network.

  \item A \emph{remote client} supports a subset of the functionality that a
    full node supports, generally wallet functionality and the ability to
    broadcast transactions. Other more complex functionalities generally require
    interacting with with a full node or other remote services which are capable
    of fulfilling requests on a remote client's behalf.
\end{itemize}

\paragraph{The Blockchain}
Like Bitcoin, the Ethereum blockchain is constructed by leveraging a
Proof-of-Work (PoW) algorithm to reach consensus throughout the network. The
proof-of-work algorithm leveraged by Ethereum, Ethash, helps to build trust and
reliability throughout the network while also securing the blockchain and
ensuring that EVM code execution has been processed and that the results
produced from said execution are as expected. Ethereum offers cryptoeconomic
incentivization in the form of \solt{ether} to promote participation in the
proof-of-work process.\cite{ethereum-yellow}

\subparagraph{Blocks}
The Ethereum protocol groups collections of transactions into \emph{blocks}.
Blocks are linked to previous blocks, via cryptographic hash, which reflect the
prior states of the blockchain. When processed collectively these blocks reflect
the current state of the Ethereum blockchain.

\subparagraph{Proof-of-Work}
Ethereum's proof-of-work algorithm, Ethash, is a proof-of-work algorithm which
was initially inspired by the Hashimoto and Dagger algorithms. The primary
motivation behind the Ethash algorithm was to produce a PoW algorithm which
would be resistant to application-specific integrated circuits (ASICs). The
primary mechanism leveraged to achieve ASIC-resistance lies in the algorithm's
memory-bound nature: a significant amount of memory, in addition to computation,
is required to correctly compute a proof-of-work solution. By requiring large
amounts of memory-bound operations the algorithm makes itself resistant to most
kinds of specialized memories and caches. Additionally, the memory requirements
are designed to grow and shift over time such that building rapid static caches
would become prohibitively expensive. In some sense the Ethash algorithm might
be better described as Proof-of-Memory.\cite{dagger-hashimoto,hashimoto,dagger}

The Ethereum network has been attempting to migrate away from Ethash to a
lower-cost consensus algorithm operating through Proof-of-Stake (PoS) but has
yet to complete the transition.

% Talk about the Greedy Heaviest Observed Subtree (GHOST) protocol.

\subparagraph{Gas}
In order to validate that EVM code has been executed, and executed as expected,
each full node on the network must recompute all transactions and whatever EVM
code those transactions have triggered when validating blocks. It is not
difficult to imagine how this might cause serious problems and introduce room
for exploitation within the network, \sol{while true { expensiveOperation() }}.
In order to address this the Ethereum specification introduces an abstraction,
\emph{gas}, which has a market-based value and must be paid by the
transaction-sender up-front when generating a transaction. Each operation
computed by the EVM has an associated gas-price and that gas price is ``paid''
to the node who successfully mints a block. If an insufficient amount of gas is
provided by the transaction-sender then the EVM will throw an out-of-gas (OOG)
exception: execution halts, the blockchain state is restored, and all gas
submitted is forfeited to the node. If an excess of gas is provided then any gas
remaining after transaction execution is refunded to the sender. When generating
a transaction on the Ethereum network the creator of the transaction has the
choice of defining how much wei they are willing to pay per unit of gas. Nodes
are incentivized to process transactions which offer more wei per unit of gas
relative to the rest of the transactions available for processing on the
network. Table~\ref{tab:gas-costs} introduces gas costs as defined by the
Ethereum Yellow Paper.


% Stuff about gas. What follows is a table of some gas costs as defined in the Ethereum
% yellow paper.

\begin{center}
  \setstretch{1.5}
  \scriptsize
  \begin{longtabu} to \textwidth{@{} X[1.1,l] X[0.5,r] X[7,l] @{}}
  \caption{Fee Schedule\cite{ethereum-yellow}}\label{tab:gas-costs} \\
  \toprule
  Name & Value & Description* \\
  \midrule
  \endfirsthead
  \toprule
  Name & Value & Description* \\
  \midrule
  \endhead
  \taburowcolors{white..gray!15}
  $G_{zero}$ & 0 & Nothing paid for operations of the set {\tiny $W_{zero}$}. \\
  $G_{base}$ & 2 & Amount of gas to pay for operations of the set {\tiny $W_{base}$}. \\
  $G_{verylow}$ & 3 & Amount of gas to pay for operations of the set {\tiny $W_{verylow}$}. \\
  $G_{low}$ & 5 & Amount of gas to pay for operations of the set {\tiny $W_{low}$}. \\
  $G_{mid}$ & 8 & Amount of gas to pay for operations of the set {\tiny $W_{mid}$}. \\
  $G_{high}$ & 10 & Amount of gas to pay for operations of the set {\tiny $W_{high}$}. \\
  $G_{extcode}$ & 700 & Amount of gas to pay for operations of the set {\tiny $W_{extcode}$}. \\
  $G_{balance}$ & 400 & Amount of gas to pay for a {\tiny BALANCE} operation. \\
  $G_{sload}$ & 200 & Paid for a {\tiny SLOAD} operation. \\
  $G_{jumpdest}$ & 1 & Paid for a {\tiny JUMPDEST} operation. \\
  $G_{sset}$ & 20000 & Paid for an {\tiny SSTORE} operation when the storage value is set to non-zero from zero. \\
  $G_{sreset}$ & 5000 & Paid for an {\tiny SSTORE} operation when the storage value's zeroness remains unchanged or is set to zero. \\
  $R_{sclear}$ & 15000 & Refund given (added into refund counter) when the storage value is set to zero from non-zero. \\
  $R_{selfdestruct}$ & 24000 & Refund given (added into refund counter) for self-destructing an account. \\
  $G_{selfdestruct}$ & 5000 & Amount of gas to pay for a {\tiny SELFDESTRUCT} operation. \\
  $G_{create}$ & 32000 & Paid for a {\tiny CREATE} operation. \\
  $G_{codedeposit}$ & 200 & Paid per byte for a {\tiny CREATE} operation to succeed in placing code into state. \\
  $G_{call}$ & 700 & Paid for a {\tiny CALL} operation. \\
  $G_{callvalue}$ & 9000 & Paid for a non-zero value transfer as part of the {\tiny CALL} operation. \\
  $G_{callstipend}$ & 2300 & A stipend for the called contract subtracted from $G_{callvalue}$ for a non-zero value transfer. \\
  $G_{newaccount}$ & 25000 & Paid for a {\tiny CALL} or {\tiny SELFDESTRUCT} operation which creates an account. \\
  $G_{exp}$ & 10 & Partial payment for an {\tiny EXP} operation. \\
  $G_{expbyte}$ & 50 & Partial payment when multiplied by $\lceil\log_{256}(exponent)\rceil$ for the {\tiny EXP} operation. \\
  $G_{memory}$ & 3 & Paid for every additional word when expanding memory. \\
  $G_\text{txcreate}$ & 32000 & Paid by all contract-creating transactions after the {\it Homestead transition}.\\
  $G_{txdatazero}$ & 4 & Paid for every zero byte of data or code for a transaction. \\
  $G_{txdatanonzero}$ & 68 & Paid for every non-zero byte of data or code for a transaction. \\
  $G_{transaction}$ & 21000 & Paid for every transaction. \\
  $G_{log}$ & 375 & Partial payment for a {\tiny LOG} operation. \\
  $G_{logdata}$ & 8 & Paid for each byte in a {\tiny LOG} operation's data. \\
  $G_{logtopic}$ & 375 & Paid for each topic of a {\tiny LOG} operation. \\
  $G_{sha3}$ & 30 & Paid for each {\tiny SHA3} operation. \\
  $G_{sha3word}$ & 6 & Paid for each word (rounded up) for input data to a {\tiny SHA3} operation. \\
  $G_{copy}$ & 3 & Partial payment for {\tiny *COPY} operations, multiplied by words copied, rounded up. \\
  $G_{blockhash}$ & 20 & Payment for {\tiny BLOCKHASH} operation. \\

  %extern u256 const c_copyGas;			///< Multiplied by the number of 32-byte words that are copied (round up) for any *COPY operation and added.
  \bottomrule
  \end{longtabu}
  % \begin{itemize} % [label=,leftmargin=0mm,topsep=0mm]
  \begin{multicols}{2}
    \begin{itemize}[label=,leftmargin=0mm,topsep=0mm,itemsep=0em]
      \scriptsize
      \item $W_{zero}$ = \tiny\{{STOP}, {RETURN}\}
      \scriptsize
      \item $W_{low}$ = \tiny\{{MUL}, {DIV}, {SDIV}, {MOD}, {SMOD}, {SIGNEXTEND}\}
      \scriptsize
      \item $W_{mid}$ = \tiny\{{ADDMOD}, {MULMOD}, {JUMP}\}
      \scriptsize
      \item $W_{high}$ = \tiny\{{JUMPI}\}
      \scriptsize
      \item $W_{extcode}$ = \tiny\{{EXTCODESIZE}\}
      \scriptsize
      \item $W_{base}$ = \tiny\{{ADDRESS}, {ORIGIN}, {CALLER}, {CALLVALUE},
        {CALLDATASIZE}, {CODESIZE}, {GASPRICE}, {COINBASE} {TIMESTAMP},
        {NUMBER}, {DIFFICULTY}, {GASLIMIT}, {POP}, {PC}, {MSIZE}, {GAS}\}
      \scriptsize
      \item $W_{verylow}$ = \tiny\{{ADD}, {SUB}, {NOT}, {LT}, {GT}, {SLT},
        {SGT}, {EQ}, {ISZERO}, {AND}, {OR}, {XOR}, {BYTE}, {CALLDATALOAD},
        {MLOAD}, {MSTORE}, {MSTORE8}, {PUSH*}, {DUP*}, {SWAP*}\}
    \end{itemize}
  \end{multicols}
\end{center}

% \paragraph{Transaction Examples}
% \subparagraph{Example 1: Alice to Bob}
% \subparagraph{Example 2: Alice to Contract}
% \subparagraph{Example 3: Contract to Contract}
