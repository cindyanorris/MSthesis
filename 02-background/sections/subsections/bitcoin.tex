\subsection{Bitcoin}
In 2008 the seminal white paper, \emph{Bitcoin: A Peer-to-Peer Electronic Cash
System}, was published under the moniker Satoshi Nakamoto.\cite{bitcoin} This
white paper outlined ideas for a new form of currency, Bitcoin. Bitcoin promised
to be the first of its kind; it would become the world's first decentralized
digital currency that would require no trust to authenticate timestamped
transactions. It would do this by combining cryptography, a proof-of-work
system, and ``miners'' to create a revolutionary new concept that is now known
as blockchain technology. In short, blockchain technology enables individuals
who do not trust one another to reach consensus via a trustless platform. More
concretely, Bitcoin can be used by Alice and Bob to send money from one to the
other over a decentralized network.

% \subsubsection{Network Topology}
\subsubsection{Network Topology}
The Bitcoin network is structured as a peer-to-peer network composed of a
variety of node types. Each node supports varying functionalities and features
based on their use case. Common functionalities and features required
include:\cite{mastering-bitcoin}
\begin{itemize}
    \item Mining functionality, to support creating new blocks by solving
      proof-of-work problems.
    \item A wallet, to offer users a way to manage their keys plus send and
      receive transactions.
    \item A copy of the full blockchain, which allows nodes to verify
      transactions independent of other nodes in the network.
    \item Network routing capabilities, which allows nodes to propagate
      transactions and discover new nodes.
\end{itemize}

\paragraph{Node Types} The most common node types, classified by their
corresponding functionalities, are as follows:\cite{mastering-bitcoin}
\begin{itemize}
    \item A \emph{reference client} contains wallet, miner, a full copy of the
        blockchain, and network routing functionality.
    \item A \emph{full node} contains a full copy of the blockchain and
        network routing functionality.
    \item A \emph{mining node} contains a miner, a full copy of the
        blockchain, and network routing functionality.
    \item A \emph{lightweight wallet} contains a wallet and network routing
        functionality. These nodes depend on \emph{full nodes} to verify
        transactions for them. These nodes are usually found on mobile devices
        where storage is limited and the size of the blockchain is inhibitive.
\end{itemize}

\subsubsection{Fundamental Concepts}
The Bitcoin blockchain is structured as a linked list of blocks, as seen in
Figure~\ref{fig:blockchain}. Each block contains a set of transactions and a
reference to the previous block in the chain. Blocks are identified by the
SHA-256 hash of its header. It is helpful to imagine the blockchain as being
blocks stacked vertically, each additional block helping to secure the previous
blocks laid before it.

\begin{figure}[H]
    \centering
    \figurepdf[width=\textwidth]{blockchain}
    \caption{A chain of linked Bitcoin blocks.~\cite{bitcoin}}\label{fig:blockchain}
    % \includestandalone[width=\textwidth]{\fig{blockchain}}
\end{figure}

\paragraph{Blocks}
The block is the Bitcoin blockchain primitive. Blocks serve as a time-stamping
tool for transactions within the network and also prove that some amount of work
(computation) occurred. The contents of a block are listed in
Table~\ref{tab:block}. Figure~\ref{fig:block} illustrates the layout of a block.

\begin{figure}[H]
    \centering
    \figurepdf{block}
    \caption{A single Bitcoin block and its internal representation.~\cite{bitcoin}}\label{fig:block}
    % \includestandalone{\fig{block}}
\end{figure}

\begin{table}
    \caption{The contents of a block \protect\cite{mastering-bitcoin}.}\label{tab:block}
    \resizebox{\columnwidth}{!}{%
        \begin{tabu} to \textwidth{@{} l l l @{}}
            \toprule
            Size                    & Field                 & Description \\
            \midrule
            4 bytes                 & Block Size            & The size of the block, in bytes, following this field \\
            80 bytes                & Block Header          & Several fields form the block header \\
            1--9 bytes (VarInt)     & Transaction Counter   & How many transactions follow \\
            Variable                & Transactions          & The transactions recorded in this block \\
            \bottomrule
        \end{tabu}
    }
\end{table}

\subparagraph{Block Header}
The block header is responsible for tracking the metadata of the Bitcoin block.
It is also used to represent the block as a whole by SHA-256 hashing the
contents of it. The contents of the block header are as seen in
Table~\ref{tab:block-header}.

\begin{table}[H]
  \setstretch{1.5}
  \caption{The contents of a block header \protect\cite{mastering-bitcoin}.}\label{tab:block-header}
  \resizebox{\columnwidth}{!}{%
    \begin{tabu} to \textwidth{@{} l l l @{}}
      \toprule
      Size            & Field                 & Description \\
      \midrule
      4 bytes         & Version               & A version number to track software/protocol upgrades \\
      32 bytes        & Previous Block Hash   & A reference to the hash of the previous (parent) block in the chain \\
      32 bytes        & Merkle Root           & A hash of the root of the Merkle tree of this block's transactions \\
      4 bytes         & Timestamp             & The approximate creation time of this block (seconds from Unix Epoch) \\
      4 bytes         & Difficulty Target     & The proof-of-work algorithm difficulty target for this block \\
      4 bytes         & Nonce                 & A counter used for the proof-of-work algorithm \\
      \bottomrule
    \end{tabu}
  }
\end{table}

\subparagraph{Merkle Trees}
A Merkle tree is a data structure that allows one to efficiently verify the
contents of a large amount of data. Merkle trees are used extensively in
peer-to-peer networks to ensure that blocks of data arrive unaltered and
undamaged. The root of a Merkle tree is called a Merkle root. Merkle trees are
composed of nodes of hashes. They have the unique property of allowing the
verification of the existence of a hash in the tree in \solt{O(log(n))} time.

% [!Merkle Tree Image]()

\paragraph{Transactions}
In Satoshi's original white paper, a coin was defined as tokens transferred via
a chain of digital signatures.

\begin{displayquote}
  ``We define an electronic coin as a chain of digital signatures. Each owner
  transfers the coin to the next by digitally signing a hash of the previous
  transaction and the public key of the next owner and adding these to the end
  of the coin. A payee can verify the signatures to verify the chain of
  ownership.''\cite{bitcoin}
\end{displayquote}

\begin{figure}[H]
    \centering
    \figurepdf{transactions}
    \caption{A chain of signatures representing an electronic coin.~\cite{bitcoin}}\label{fig:coin}
    % \includestandalone{\fig{transactions}}
\end{figure}

Figure~\ref{fig:coin} demonstrates Owner 1 hashing the transaction that gave her
ownership of the coin with the public key of the new owner (Owner 2), then
signing that hash with her private key and publishing that as a new
transaction. Owner 2 repeats the process to send his coins to Owner 3.


% \textbf{Blockchain Stack}
% \begin{figure}[H]
%     \caption{A blockchain stack with 10 blocks.}\label{tab:blockchain-stack}
%     \begin{tabu} to \textwidth{@{} l l l @{}}
%         \toprule
%         Block \# \\
%         \midrule
%         Block 9 \\
%         Block 8 \\
%         $\cdots$ \\
%         Block 1 \\
%         Block 0 \\
%         \bottomrule
%     \end{tabu}
% \end{figure}

% |  Block #  |
% |:---------:|
% |  Block 9  |
% |  Block 8  |
% |    ...    |
% |  Block 1  |
% |  Block 0  |



% | Size               | Field               | Description                                           |
% |--------------------|---------------------|-------------------------------------------------------|
% | 4 bytes            | Block Size          | The size of the block, in bytes, following this field |
% | 80 bytes           | Block Header        | Several fields form the block header                  |
% | 1-9 bytes (VarInt) | Transaction Counter | How many transactions follow                          |
% | Variable           | Transactions        | The transactions recorded in this block               |


% | Size     | Field               | Description                                                           |
% |----------|---------------------|-----------------------------------------------------------------------|
% | 4 bytes  | Version             | A version number to track software/protocol upgrades                  |
% | 32 bytes | Previous Block Hash | A reference to the hash of the previous (parent) block in the chain   |
% | 32 bytes | Merkle Root         | A hash of the root of the Merkle tree of this block's transactions    |
% | 4 bytes  | Timestamp           | The approximate creation time of this block (seconds from Unix Epoch) |
% | 4 bytes  | Difficulty Target   | The proof-of-work algorithm difficulty target for this block          |
% | 4 bytes  | Nonce               | A counter used for the proof-of-work algorithm                        |
%Source: Mastering Bitcoin

\paragraph{Proof-of-Work}
The bitcoin protocol uses a \emph{Proof-Of-Work (POW)} algorithm similar to
hashcash.\cite{hashcash,bitcoin} The goal of this proof-of-work algorithm is to
create a problem that is easy to verify for correctness but difficult to solve
for. The proof-of-work algorithm provides a means for mining nodes to be
pseudorandomly selected to build a block of transactions. The probability that a
miner will be selected is directly tied to the amount of \emph{work} a miner
does.
 % (read: NP-complete (?)).
% cite: hashcash

\subparagraph{Algorithm}
The proof-of-work algorithm depends on the \emph{SHA-256} cryptographic hashing
function, a member of the \emph{Secure Hash Algorithm 2 (SHA-2)} family, which
produces a 256-bit, 32-byte, hash result.

The primary requirements which a \emph{cryptographic hash function} must fulfill
are that it be:

\begin{itemize}
  \item deterministic, i.e., the same input will always produce the same output,
  \item quick to compute,
  \item infeasible to determine the input from the output, i.e., a small change
    in the input will produce a major, seemingly random, change to the output,
    and
  \item infeasible to find a collision in resulting hashes.
\end{itemize}

These properties of cryptographic hash functions provide \emph{collision
resistance}, meaning it is computationally infeasible to find an input that
produces a randomly selected hash output. This property of collision resistance
is leveraged to build the pseudorandom selection process that determines which
node is able to build the next block of transactions.

\subparagraph{Difficulty}
The Bitcoin network has a global \emph{difficulty} --- a 256-bit,
32-byte, value --- that is recalculated every 2016 blocks. The value is
recalculated such that the pseudorandom ``mining'' process to mint a new block
will take approximately 10 minutes to complete for each block minted.

While it is helpful to describe miners as being ``selected'' to build, or mint,
new blocks, it is also inaccurate; in reality miners have a chance of building a
correct block which can be measured as a probability, i.e., the miner's hash
rate relative to the total network hash rate:
\(\frac{miner\_hashrate}{network\_hashrate}\).

Miners repeatedly compute SHA-256 hashes, combining the previous block's header
hash, the current block data, and a \emph{nonce} (a pseudorandom shifting value
selected by the miner). The miner's objective is to find a nonce which produces
a hash less than the current difficulty. Due to the properties of the SHA-256
hash this process is, practically speaking, a brute-force processes of trial and
error, (the ``work'' in Proof-of-Work). Once a miner has discovered a valid
hash they have generated a valid block.

\subparagraph{Network}
Once a miner generates a valid block it will propagate its solution into the
network. Other nodes will then verify that the block is correct and will append
it to their chain. One of the transactions which miners will include in their
block is a coinbase transaction, a transaction containing reward paid to the
miner in the form of newly minted bitcoin.

\subparagraph{Security}
These properties, combined with the incentive of coinbase rewards, provide
security in the form of cryptography, electricity, and hardware. Attacks that
would threaten this security depend on breaking the cryptographic primitives in
action, finding ways of reducing electricity/hardware costs to outperform the
rest of the network in the PoW algorithm, attacking nodes in the network, or
colluding with other nodes in the network to outperform the remainder of the
network.

\subsubsection{Example Transaction: Alice to Bob}

As an example, we might imagine a circumstance where \emph{Alice wants to send
Bob 0.100 BTC.} Assume Alice controls two addresses:

\begin{enumerate}
  \item \sol{1PwDUn9Vxn9CyaRkZfJrTzYRg6QNCygALY}, which contains 6900000
    satoshis (0.069 BTC).
  \item \sol{1Ah1CWQ1Zxax2yzg3EZBmSmZTKDoHTuUAi}, which contains 7900000
    satoshis (0.079 BTC).
\end{enumerate}

Alice does not have enough bitcoin in either address independently to send Bob
the 0.100 BTC.\ To resolve this Alice creates a new transaction which takes as
an input both of her addresses. Alice proves she controls both addresses by
signing each transaction address and publishing her public key for each
respectively. This process allows a node on the bitcoin network to verify both
that public key is associated with the address (through a series of
cryptographic hashes) and that the person publishing this transaction controls
the private keys (by verifying the signature). Next Alice sets an output address
to Bob's address, an amount to send to Bob (0.100 BTC), and the rules Bob must
fulfill to access the bitcoin she is sending to him (typically signing the
transaction and publishing the associated public key). Alice could now publish
this transaction to the bitcoin network; however, in doing so she would
sacrifice 0.048 BTC, the remainder from her two addresses, which the miners
would claim as a transaction fee. To avoid this Alice can set a second
transaction output address to an address she controls, sending any additional
coins there that she does not want to offer as a transaction fee.
