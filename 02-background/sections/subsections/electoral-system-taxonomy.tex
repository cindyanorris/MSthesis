\subsection{A Taxonomy of Electoral Systems}
Electoral systems are generally broken into three broad families based primarily
on the properties described above. The three families are plurality/majority,
proportional representation, and mixed. An overview various electoral systems is
offered in Table~\ref{tab:electoral-systems} to help visualize the landscape of
electoral systems available and where they fall.

\begin{center}
  \setstretch{1.5}
    \begin{table}[H]
      \caption{Electoral systems organized by families and characteristics.\cite{electoral-handbook}}\label{tab:electoral-systems}
        \scriptsize
        \begin{tabu} to \textwidth{@{} X[0.022l] X[1.4l] X[c] X[0.33c] X[0.33c] X[0.33c] X[c] @{}}
            \toprule
            \multicolumn{2}{l}{Electoral System}   & {Proportional Representation}     & \multicolumn{3}{c}{Plurjority}                & {Mixed} \\
                                                                                      \cmidrule{4-6}
                  &                             &                                   & Plurality     & Majority      & MMD           &  \\
            \midrule
            \taburowcolors{white..gray!15}
            STV   & Single Transferable Vote    & \textbullet{}                     &               &               &               & \\
                  & List PR                     & \textbullet{}                     &               &               &               & \\
            \addlinespace[-0.4ex]
            \cmidrule(r){1-2} \cmidrule(lr){3-3} \cmidrule(lr){4-6} \cmidrule(lr){7-7}
            \addlinespace[-0.80ex]
            FPTP  & First Past The Post         &                                   & \textbullet{} &               &               & \\
            BV    & Block Vote                  &                                   & \textbullet{} &               & \textbullet{} & \\
            PBV   & Party Block Vote            &                                   & \textbullet{} &               & \textbullet{} & \\
                  & Preferential Block Vote     &                                   &               & \textbullet{} & \textbullet{} & \\
            AV    & Alternative Vote            &                                   &               & \textbullet{} &               & \\
            RV    & Range Vote                  &                                   &               & \textbullet{} &               & \\
            TRS   & Two-Round System            &                                   & \textbullet{} & \textbullet{} &               & \\
            \addlinespace[-0.4ex]
            \cmidrule(r){1-2} \cmidrule(lr){3-3} \cmidrule(lr){4-6} \cmidrule(lr){7-7}
            \addlinespace[-0.80ex]
                  & Parallel                    & \textbullet{}                     & \textbullet{} & \textbullet{} & \textbullet{} & \textbullet{} \\
            MMP   & Mixed-Member Proportional   & \textbullet{}                     & \textbullet{} & \textbullet{} & \textbullet{} & \textbullet{} \\
            \bottomrule
            \addlinespace[\belowrulesep]
        \end{tabu}
        MMD: Multi-Member District, a district where multiple seats are
        available for candidates to hold or where multiple choices will be
        determined winners.
    \end{table}
\end{center}

\subsubsection{Plurality/Majority}
Plurality and majority electoral systems are simple in principle although not
necessarily in design. After votes have been tallied the choices with the
most resulting votes are the decided winners.

\paragraph{Plurality Voting}
A plurality voting system is one where the winning choices are the ones with the
greatest number of votes \emph{but not necessarily an absolute majority}. The
United States uses plurality electoral systems almost exclusively.

\subparagraph{First-Past-The-Post}
First-Past-the-Post (FPTP) is a plurality single-member district electoral
system used widely across the world and almost exclusively in the United States
of America. FPTP is one of the simplest electoral systems to understand; the
voter is presented with a set of choices on a ballot and is able to select
\emph{one and only one} of the choices. The ballots are then collected and
tallied by counting the number of votes cast for each choice; the choice with
the most votes wins.

% NOTE: Removed for brevity.
% Proponents of FPTP argue that it enjoys advantages which include:
% \begin{itemize}
%     \item It offers a simple design that is easy for voters to understand.
%     \item It provides a simple choice between two major parties.
%     \item It gives rise to broad encompassing parties and candidates.
%     \item It excludes extremist parties from representation.
%     \item It promotes strong geographical representation.
%     \item It allows voters to vote for individuals instead of parties.
%     \item Its design gives rise to single-party governments.
% \end{itemize}
%
% Opponents of FPTP criticize FPTP on the basis of:
% \begin{itemize}
%     \item It excludes smaller parties from offering representation.
%     \item It makes running as a minority more difficult.
%     \item It encourages political parties to form on the basis of ethnicity,
%         clan, or religion.
%     \item It makes regional fiefdoms more likely and reinforces the perception
%         of politics being defined by where one lives instead of one's ideology.
%     \item It creates a large number of wasted votes, which often disenfranchises
%         voters and candidates.
%     \item It is very dependent on the drawing of electoral boundaries.
%     \item The system can be unresponsive to changes in political opinion.
%     \item It can lead to vote-splitting, which causes two popular but similar
%         candidates to lose to a third much less popular candidate.
%     \item It forces voters to express their will in the least substantial way
%         possible – approval of just a single option while expressing nothing
%         about how much the voter likes or dislikes any choice.
%     \item It rewards the electorate for not voting honestly.
%     \item Over time, it tends towards two or fewer parties.
%     \item FPTP ballots are easy to invalidate by overvoting.
% \end{itemize}


% NOTE: Removed for brevity.
% \subparagraph{Block Vote}
% Block Vote (BV) is a plurality multi-member district voting system that operates
% much like FPTP. Voters are given as many votes as there are winning choices to
% be selected and are able to vote for as many options as they have votes.
% The elector can only submit one vote per candidate and most BV systems allow
% voters to forgo using all of their available votes. After the voting period
% ballots are collected and tallied by counting the total number of votes for each
% candidate; the choices with the most votes cast in their favor are selected as
% winners.
%
% Proponents of BV argue that it is simple and that voters retain the ability to
% vote for individual candidates. However, BV also exaggerates most of the
% disadvantages of FPTP. Further, BV can encourage party infighting by forcing
% candidates to compete against one another.
%
% \subparagraph{Party Block Vote}
% Party Block Vote (PBV) is a plurality multi-member district voting system that
% operates much like FPTP; each voter is given a single vote that can be cast for
% a party which puts forth a list of candidate options. If the party wins the
% election then the party wins all available seats for the district. The
% candidates from the candidate list presented on the ballot earn the seats.
%
% Proponents of PBV argue that it is simple while also providing a fairer and more
% diverse slate of candidates for representation. PBV still inherits most of the
% disadvantages of FPTP.

\paragraph{Majority Voting}
A majority voting system is characterized such that a choice will only be
considered the winner with an absolute majority of votes, that is, \emph{greater
than 50\% of the votes}. Most majority voting systems take advantage of
preference marking or multiple election cycles, where less-popular choices are
removed in each cycle, to form a majority.

% NOTE: Removed for brevity.
% \subparagraph{Alternative Vote}
% Alternative Vote (AV) --- also known as Instant-Runoff Voting (IRV), Ranked
% Choice Voting (RCV), transferable vote, or preferential vote --- is a majority
% single-member district voting system that depends on preference marked ballots
% to determine a majority choice. Voters mark their ballot by ranking options
% based on their preference; e.g., 1 for first choice, 2 for second choice, 3 for
% third choice, etc. Votes are summed up for each elector's first choice. If an
% option wins a majority, i.e., greater than 50\%, then that option immediately
% wins. Otherwise, a second election is simulated by determining the least popular
% option and eliminating it. Votes which were cast for the eliminated option are
% moved to the voter's second choice. This process continues until one choice has
% an absolute majority. Some AV systems require voters to rank all choices (full
% preferential voting), while others allow a partial ranking (optional
% preferential voting). One consequence of optional preferential voting is that an
% election can result in ballot exhaustion whereby no candidate can win with an
% absolute majority. Electoral system designers must decide how such edge cases
% will be handled.
%
% Proponents of AV argue that it is more likely to elect centrist options, results
% in much fewer wasted votes than minority vote alternatives, forces candidates to
% appeal to larger groups and make broader appeals, and increases the perceived
% legitimacy of the winning choice. Critics of AV point out that it is complex,
% requires an understanding of numeracy and literacy, which may disenfranchise
% electorate, and encourages dishonest voting techniques, casting ballots with
% dishonest choice-preferences to increase the likelihood of a preferred candidate
% winning.


\subparagraph{Range Vote}
Range Vote (RV), also known as Score Voting (SV), is a single-member district
majority electoral system. It takes advantage of cardinally preference marked
ballots to determine a majority choice. This form of voting allows voters to
express a preference between choices and also the degree of preference between
choices.

A range voting ballot allows voters to select a non-negative integer, up to some
maximum (usually 9 or 99), which expresses some degree of approval for a
particular option, e.g., 0 for least preferred or 9 for most preferred. Some
range voting systems also allow for disapproval voting, down to some minimum, to
express disapproval for a particular option, but this is less common. The system
may also support a ``No Opinion'' mark that can be cast if a voter is ignorant
or indifferent to some choice; this is usually considered the default mark if no
mark is made for a particular option and does not count for or against the
option. Once all ballots are cast and collected they are tallied. The tallying
process is as follows: votes for an option are summed together and divided by
the number of ballots that actively voted for that option (marked anything other
than ``No Opinion''). The option with the highest average score wins.

An approval voting system is the simplest and most restricted kind of range
voting, the range of votes is restricted to either 0 or 1. A voter can approve
as many options as they want.

% NOTE: Removed for brevity.
% Proponents of range voting argue that it has the following advantages:
%
% \begin{itemize}
%     \item It is expressive; the will and preferences of the electorate are
%           collected at a very granular and quantitative level by being able
%           to cast votes for multiple candidates, indicate a preference between
%           candidates, and express the degree of preference between those
%           candidates.
%     \item It avoids the spoiler effect, voting for candidate C does not affect a
%           battle between A and B.
%     \item There is no risk of vote-splitting.
%     \item It is less likely to produce a two-party or single-party environment
%           and encourages smaller parties to run.
%     \item Current voting machines can accommodate RV with little to no
%           modification.
%     \item There are fewer spoiled ballots because it is difficult to incorrectly
%           mark a ballot.
%     \item It is simple to understand and tally.
% \end{itemize}
%
% Critics of RV oppose it for being vulnerable to strategic voting and because
% there exists a possibility for a winner to be selected that was no voter's first
% choice.
% cite: section evaluating electoral systems (strategic voting)

% \blockquote{%
%     By avoiding ranking (and its implication of a monotonic approval reduction
%     from most- to least-preferred candidate) cardinal voting methods may solve
%     a very difficult problem:
%
%     A foundational result in social choice theory (the study of voting
%     methods) is Arrow's impossibility theorem, which states that no method can
%     comply with all of a simple set of desirable criteria. However, since one
%     of these criteria (called "universality") implicitly requires that a
%     method be ordinal, not cardinal, Arrow's theorem does not apply to
%     cardinal methods.[5][6][7][8]
%
%     Others, however, argue that this is not true, for instance because
%     interpersonal comparisons of cardinal measures are impossible.[9] If that
%     is the case, then cardinal methods do indeed fail to escape Arrow's
%     result.
%
%     Psychological research has shown that cardinal ratings are more valid and
%     convey more information than ordinal rankings in measuring human
%     opinion.[10][11][12][13]
%
%     In any case, cardinal methods do fall under the Gibbard–Satterthwaite
%     theorem, and therefore any such method must be subject to strategic voting
%     in some instances.[14][15][dubious – discuss][16]
% }

% NOTE: Removed for brevity.
% \paragraph{Two-Round System}
% The Two-Round System (TRS) is not actually a single election, but two. TRS does
% not fit cleanly into either majority or minority systems. It can be both or
% either depending on the implementation. Typically TRS elections are
% plurality-majority systems, but their format can vary widely; the first
% election, a plurality election, captures the two most popular choices, then a
% majority vote is conducted to determine the most popular choice from the
% remaining options.
%
% TRS advantages include giving voters a second chance to express their will and
% encouraging alternative candidates or proponents of other choices to put their
% weight behind a second choice, thus lessening the effect of vote-splitting.
% Unfortunately TRS elections can be prohibitively expensive and difficult to
% administer on a large scale. Further, TRS still shares many of the disadvantages
% of FPTP.


\subsubsection{Proportional Representation}
The objective of a Proportional Representation (PR) electoral system is to
produce winners from an election that accurately reflect the will of the people
and the votes they cast by reducing the disparity between shares of votes cast
for choices and shares of winning choices; this is especially relevant when
considering seats won in a representative governance models. PR operates by
providing a cross section of winners in an election which map proportionally to
the votes cast for choices. For example, if a quarter of voters in an election
desire some outcome then the election results should reflect that by producing
winners in proportion to those voters' desires; i.e., a quarter of the winners
of the election should be that of the voters' desired outcome.

% NOTE: Removed for brevity.
% A simple example, which illustrates the utility of PR electoral systems over
% plurality and majority systems, is ordering food for a group of people. For
% example, suppose it is necessary to order 10 pizzas to feed a group of people.
% If 30\% of the individuals in the group wanted pepperoni pizza and the other
% 70\% was split into 7 different groups of 10\% wanting different pizzas, it
% would not be fair to order 10 pepperoni pizzas. Likewise, even if 60\%, a
% majority, wanted pepperoni, it still would not be very fair to order 10
% pepperoni pizzas. PR systems try to address this by instead ordering 6 pepperoni
% pizzas and 4 mushroom or, in the former case, 3 pepperoni pizzas and 7 different
% pizzas that represent the desires of the people.

%%
% TODO: Insert bar graphs of the above example.
%       - Preferences
%       - Unfavorable outcome
%       - Favorable outcome
%%

% \subparagraph{Advantages}
There are a number of benefits associated with PR systems:
\begin{itemize}
    \item Votes translate into choices won with greater proportionality.
    \item Results in fewer wasted votes.
    \item Offers minority groups greater representation.
    \item Restricts regional fiefdoms.
    \item Promotes greater long-term political health and negotiation.
    \item Results in greater voter participation and perceived legitimacy.
    \item Supports a more inclusive cross-section of representatives.
\end{itemize}

% \subparagraph{Disadvantages}
There are also several disadvantages associated with PR systems:
\begin{itemize}
    \item Quick and coherent policies can be difficult to pass.
    \item Legislative gridlock can occur when factions are formed.
    \item Fragmentation of strong parties can occur in cases of extreme
        pluralism.
    \item Minority parties can hold parties ransom in coalition negotiation,
        thus offering smaller parties greater authority than perhaps deserved.
    \item It is difficult to enforce accountability by throwing out parties or
        candidates.
    \item Potentially difficult for electorate to understand or administrators
        to implement.
\end{itemize}

% \paragraph{List Proportional Representation}
% List Proportional Representation (List PR) is a party-based form of PR. Parties
% present a list of candidates and voters vote for a single party. The parties win
% some number of seats in proportion to the number of votes received and
% candidates are elected as representatives in the order they are presented on the
% ballot.
%
% Benefits of List PR include greater representation for and by minority groups
% and women.
%
% A disadvantage of List PR is that weaker links tend to be formed between
% individuals and their representatives. Further, over time, power tends to vest
% towards party leaders who can become entrenched as it is the party leaders who
% are generally responsible for selecting the candidates included in the party
% list and also the order with which candidates are to be selected for winning
% seats. Also, closed list of representatives may be put forth by the party which
% prevents the electorate from expressing their desire for a particular
% representative, who they might favor more, but does not belong to the party they
% wish to win or is simply lower in the party's candidate list. Finally, List PR,
% as a prerequisite to usage, requires that parties already exist in order to
% provide candidate lists.

\paragraph{Single Transferable Vote}
Single Transferable Vote (STV) is a PR multi-member electoral system that
operates similar to the Alternative Vote (AV). Like AV, STV leverages preference
marked ballots. STV operates as follows.
\begin{itemize*}[label={}, itemjoin={}]
  \item Electorate ordinally rank options on their ballots.
  \item Ballots are collected and tallied.
  \item While tallying, if any option reaches the quota, i.e., receives the
    minimum number of votes required, they are immediately declared a winner in
    the election and awarded a seat for the district.
  \item Surplus votes for a winning candidate are redistributed to other
    candidates based on ballot-preference.
  \item If the quota cannot be reached by any candidate then the least popular
    candidate is eliminated and votes are redistributed based on
    ballot-preference.
\end{itemize*}\cite{stv}

% \subparagraph{Quota}
There are many quota algorithm implementations which can be used in STV
elections. The quota algorithm used will determine the minimum number of votes
required to win a seat in an election. One such quota algorithm, perhaps the
most popular, is the Droop quota:\cite{electoral-handbook}
%
\begin{center}
    $Quota_{Droop} = \frac{votes}{seats+1}+1$
\end{center}
%
% NOTE: Removed for brevity.
% Another popular implementation, which produces a slightly lower threshold to
% win, is the Hare quota:
%
% \begin{center}
%     $Quota_{Hare} = \frac{votes}{seats}$
% \end{center}
%
% A less-popular quota is the Hagenbach-Bischoff quota:
%
% \begin{center}
%     $Quota_{HB} = \frac{votes}{seats+1}$
% \end{center}
%
%
% \subparagraph{Surplus Vote Allocation}
If the quota for winning is met surplus vote allocation occurs, surplus votes
are redistributed to second choices. STV implementations can differ here. It
would not necessarily be fair to select only some people to have their votes
redistributed, since voters can have different second choices. However, it might
be acceptable if surplus ballots were randomly selected for second choices. In
large enough elections with large enough surpluses the second choices should be
statistically proportional. However, non-deterministic winners are not ideal and
can lead to infinite recursion in some cases. Instead, everyone's votes (from
ballots whose candidate won) can be redistributed to their second choices (or
third, fourth, etc.\ if their second has already won) as a fraction of a vote;
this algorithm will usually be based on surplus votes, total votes, previous
votes, etc. There are many surplus allocation algorithms.\cite{electoral-handbook}
% NOTE: Removed for brevity.
% \begin{itemize}
%     \item Hare method
%     \item Cincinnati method
%     \item Hare-Clark method
%     \item Gregory method
% \end{itemize}
There are also surplus vote allocation algorithms for subsequent surplus vote
allocation, for when a second choice has already won.\cite{electoral-handbook,stv}
% NOTE: Removed for brevity.
% \begin{itemize}
%     \item Meek's method
%     \item Warren's method
%     \item The Wright System
% \end{itemize}

% \subparagraph{Excluded Candidate Vote Allocation}
When no option can win excluded candidate vote allocation occurs, the weakest
options are eliminated and their votes reallocated. There are also a number of
algorithms available when deciding how votes should be redistributed which
options are eliminated.\cite{electoral-handbook}

% NOTE: Removed for brevity.
% \begin{itemize}
%     \item Single transaction
%     \item Segmented distribution
%     \begin{itemize}
%         \item Value based segmentation
%         \item Aggregated primary vote and value segmentation
%         \item FIFO (First In First Out - Last bundle)
%     \end{itemize}
%     \item Reiterative count
% \end{itemize}

% NOTE: Removed for brevity.
% \subparagraph{Advantages}
% STV is argued to be one of the most attractive and sophisticated electoral
% systems by political scientists. STV allows for voters to choose between voting
% for parties and voting for candidates themselves. In addition, there are very
% few wasted votes, voters can maintain a geographical link to their
% representatives, and representatives are elected in proportion to the votes
% received.
%
% \subparagraph{Disadvantages}
% STV is criticized for being difficult to administrate and complex for voters.
% It is vulnerable under some implementations to tactical voting. Further, vote
% tallying must be performed at a centralized location due to constant
% recalculations and simulated elections; this centralization is concerning for
% reasons regarding election integrity. STV may cause parties to fragment
% internally due to party members running against each other.


% NOTE: Removed for brevity.
% \subsubsection{Mixed}
% The last major family of electoral systems are mixed voting systems. A mixed
% voting system is one that combines components of majority/plurality voting
% systems and proportional representation voting systems. There are two major
% forms of mixed electoral systems: parallel and mixed-member proportional. They
% might use any of the aforementioned electoral systems.

% NOTE: Removed for brevity.
% \paragraph {Parallel}
% A parallel system is one where the outcomes of the majority/plurality voting
% system and PR voting system are independent from one another.

% NOTE: Removed for brevity.
% \paragraph {Mixed-Member Proportional}
% A Mixed-Member Proportional (MMP) voting system is one where the
% majority/plurality voting system and PR voting system are dependent on another.
% Typically the PR system has some number of seats reserved which are used to
% offset any perceived unproportional election results from the majority/plurality
% election.

% \subsubsection{Summary}
% The reviewed electoral systems include only a few of the many different
% electoral systems. We have seen that there are a wide range of electoral
% systems, each of which include their own set of advantages and disadvantages and
% implementation details which can be tweaked in major or minor ways to produce a
% variety of characteristics. There are many other electoral system modalities not
% covered here which can be used to reach decisions in different contexts.
% For example, consider what kinds of electoral systems might be appropriate to
% use when deciding on solutions to complex problems, e.g., where multiple choices
% need to work in conjunction to produce a sane result. Or, where choices cannot
% easily be grouped into categories, e.g., room-temperature or tax rates.



% \subsubsection{Other Voting Systems}
% \paragraph{Borda}
% \paragraph{Limited Vote}
% \paragraph{Single Non-Transferable Vote}
% \paragraph{Comparison of Pairs of Outcomes by the Single Transferable Vote}
% \paragraph{Condorcet Method}
%
% \blockquote{%
%     Preferential voting or rank voting describes certain voting systems in which
%     voters rank outcomes in a hierarchy on the ordinal scale (ordinal voting
%     systems). When choosing between more than two options, preferential voting
%     systems provide a number of advantages over first-past-the-post voting (also
%     called plurality voting). This does not mean that preferential voting is the
%     best system; Arrow's impossibility theorem proves that no preferential
%     method can simultaneously obtain all properties desirable in a voting
%     system.[Mankiw 1][1] There is likewise no consensus among academics or
%     public servants as to the best electoral system.
% }
%
% \subsection{Other Voting System Concepts}
% \subparagraph{Ordinal Voting}
% An \textbf{ordinal vote} system is one where
%
%
% \subparagraph{Cardinal Voting}
% A \textbf{cardinal vote} system is one where
%
% \subparagraph{Proxy Voting}
% A \textbf{proxy vote} system is one where
