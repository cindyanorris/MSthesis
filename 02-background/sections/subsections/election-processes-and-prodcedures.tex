\subsection{Processes, Procedures, Components, and Risks}
\todo{Are these just the non-functional requirements?}
In practice implementing a voting system can be a massive undertaking with
tremendous consequences. For example, elections in the US dictate how and what
policies will be implemented over the next several years. As such, there are
serious risks and concerns that must be considered when attempting to build such
a system. First and foremost, an implemented voting system should protect the
voting actors as much as possible: from physical and social repercussions,
bribery, coercion, intimidation, Second, electoral systems should ensure that
the voting process itself is secure, i.e., resistant to corruption and
manipulation. Finally, an implemented voting system must be scalable
(financially, temporally, geographically, etc.) in order to be effective in
large-scale elections. This material offers background on:

\begin{itemize}
  \item \emph{Voting Processes},
  \item \emph{Physical Components}, and
  \item \emph{Risks}.
\end{itemize}

\subsubsection{Voting Processes and Procedures}
In practice, voting systems require \emph{processes and procedures} which enable
voting actors to cast votes and express their preference. The voting process can
generally be understood as being comprised of four distinct components:

\begin{itemize}
  \item \emph{Registration},
  \item \emph{Verification, Authentication, and Authorization},
  \item \emph{Vote Casting}, and
  \item \emph{Collection and Processing}.
\end{itemize}

\paragraph{Registration}
The registration component of the voting process is how the collection
of eligible voters is established. The registration process varies from state to
state in the US.\ Some states allow \emph{election day registration}, a process
whereby one can register to vote in major federal elections on the same day you
vote; however, most states require their electorate to register themselves
before the election in order to have enough time to disseminate the collection
of eligible voters to polling station officials. Several states require that the
voting actor register up to 30 days before the day of the vote. Currently only
five states in the US have automatic voter registration, or ``opt-out''
registration, a process by which citizens of that state are automatically
granted voting rights in upcoming elections if they meet eligible voter
requirements, e.g., are over the age of 18.

\paragraph{Verification, Authentication, and Authorization}
The \emph{verification} component of a vote is the process by which a voting
actor provides some form of identification in order to prove they are who they
claim to be. The identity is then \emph{authenticated} via some data source by a
polling station. After the polling station has confirmed that the voting actor
is who they claim to be and is also registered to vote, the polling place will
\emph{authorize} the voting actor to cast their vote. These combined processes
prevent bad actors from manipulating the electoral system and affecting the
outcome of the election.

% These steps combined with the registration process prevent what is known as
% Sybil attacks in peer-to-peer networks, an attack on a system by creating
% false identities to gain influence over a system.

\paragraph{Vote Casting}
Once a voter has moved through the verification, authentication, and
authorization processes they are eligible to cast their vote. The voting actor
is provided a ballot to mark their preferences then able to submit their vote.
The US does not have a consistent vote casting process across its states; some
states use paper ballots, others use mechanical voting machines, and others use
electronic machines. Special precautions must be taken here to ensure that votes
are cast securely and privately.

\paragraph{Collection and Processing}
The final step of the voting process is the collection and processing component.
This step requires that the polling places aggregate and tally all of the votes
to determine a final result. Records must be kept that would provide a means to
audit and recount votes if necessary. In electronic machines the collection and
tallying process would be done by reading a memory. Mechanical machines via a
mechanical tabulation device. Paper ballots, perhaps the most primitive, require
either scanning technology or hand counting.


%%
% TODO: This section is WEAK.
%%
\subsubsection{Fundamental Concepts and Components}
The physical components leveraged within the voting process are critical to
maintaining election security, privacy, reliability, and faith in the system.

\paragraph{Ballot}
The ballot is the physical mechanism presented to the voter to cast their vote.
How the ballot is presented and cast is an important part of securing the voting
system and protecting the voter.

\subparagraph{Secret Ballot}
A secret ballot is a ballot which anonymizes a voting actor's choice. This is
done for several reasons, namely to prevent voter intimidation and vote buying.
Secret ballots are a right provided by several treaties: the \emph{Universal
Declaration of Human Rights}, the \emph{American Convention on Human Rights},
and the \emph{Convention on the Standards of Democratic Elections, Electoral
Rights and Freedoms}. Secret ballots are one of the most important rights
offered by democratic societies. Secret ballots allow voters to vote on topics
without fear of retaliation or outcast by the rest of society. However, the term
secret ballot is somewhat of a misnomer; a secret ballot is generally a normal
ballot that is cast in such a way that it would not be possible to know who cast
the ballot. For example, dropping a paper ballot into a sealed guarded box.
Voting booths, small rooms with curtains or desks with privacy barriers in place
to protect a voter's privacy, are another example of privacy maintain devices
and used almost ubiquitously around the world.

While there are many benefits to secret ballots secret ballots are not always
necessary or useful to have during voting processes. For example, if a unanimous
vote were required on a topic it might be helpful for individuals to make their
opinions public to promote discussion. Another example would be if a
representative were voting on a citizen's behalf, for the sake of transparency
an open ballot would be more appropriate.

\paragraph{Electronic Voting Machine}
An electronic voting machine is an electronic device which presents voters with
their ballot.

\paragraph{Voting Booth}
A voting booth is a safe, private, protected place for voters to mark and cast
their ballot.

\paragraph{Tallying Machines}
Tallying machines offer fast and efficient mechanism for tallying the results of
an election.

\paragraph{Ballot Box}
A secure location for voters to cast their vote after marking their ballot.

% \begin{displayquote}
%   These universal democratic principles can be summarized as a list of
%   fundamental requirements, or ``six commandments,'' for electronic voting
%   systems[citation needed]:
%
%   \begin{enumerate}
%     \item Keep each voter's choice an inviolable secret.
%     \item Allow each eligible voter to vote only once, and only for those
%           offices for which he/she is authorized to cast a vote.
%     \item Do not permit tampering with the voting systems operations, nor
%           allow voters to sell their votes.
%     \item Report all votes accurately
%     \item The voting system shall remain operable throughout each election.
%     \item Keep an audit trail to detect any breach of [2] and [4] but
%           without violating [1].
%   \end{enumerate}
% \end{displayquote}

\subsubsection{Risks}
There are a number of risks which must be managed when conducting elections:

\begin{itemize}
  \item election fraud: machine rigging, bribing officials, etc.;
  \item voter fraud: vote buying, ballot stuffing, intimidation, etc.
\end{itemize}

These risks are addressed in greater detail in chapter \ref{chap:literature},
\emph{\nameref{chap:literature}}.
