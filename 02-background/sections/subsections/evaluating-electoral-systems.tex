\subsection{Evaluating Electoral Systems}\label{sec:electoral-criteria}
Given the wide range of electoral systems that exist, or could conceivably
exist, it becomes necessary to introduce techniques that can be used to
objectively analyze and evaluate the various characteristics of electoral
systems. Social choice theory provides tools which one can use to examine and
categorize voting systems: by their advantages, their disadvantages, and
caveats.
% Voting theory demonstrates interesting and often unexpected results that can
% occur while using various electoral systems.
% What follows is an exploration into some of the voting criteria, theorems, and
% paradoxes introduced through social choice theory.

% TODO: Add this into an appendix.
% \paragraph{Sets}
% \subparagraph{Smith Set}
% \subparagraph{Landau set}

\subsubsection{Criteria}
It can be difficult to objectively judge and select an electoral system; the
choice of electoral system will have impacts on the groups, ideologies, and
candidates that are likely to succeed in a given governance model, and the
consequences of an electoral system are not always immediately obvious. In order
to compare and contrast electoral systems more objectively, there are criteria
that exist to express and describe some characteristic of an electoral system.
Table~\ref{tab:criteria-compliance} provides a summary of criteria compliance
across various electoral systems to help visualize the unique range of
compliance across various electoral systems. The criteria fall into four major
categories:

\begin{enumerate}
  \item \emph{Absolute Result Criteria},
  \item \emph{Relative Result Criteria},
  \item \emph{Ballot Counting Criteria}, and
  \item \emph{Strategy Criteria}.
\end{enumerate}


\begin{center}
  \setstretch{1.5}
    \scriptsize
    \begin{longtabu} to \textwidth{@{} X[5.85,l] X[c] X[c] X[c] X[c] X[c] X[c] X[c] X[c] X[c] X[c] X[c] X[c] X[c] X[c] @{}}
      \caption{Electoral Criteria Compliance\cite{electoral-system-analysis,electoral-handbook,electoral-system-comparison,stv-algorithms,favorite-betrayal-comparison}} \\
        \toprule
        \taburowcolors{white..white}
        Electoral Criterion    & \multicolumn{14}{c}{Electoral System} \\
                                 \cmidrule{2-15}
                               & \rot{Approval}  & \rot{Borda Count} & \rot{Copeland}           & \rot{IRV (AV)}  & \rot{Kemeny-Young} & \rot{FPTP}         & \rot{Range Voting} & \rot{Ranked Pairs} & \rot{Runoff Voting} & \rot{Schulze}   & \rot{Sortition} & \rot{Random Ballot} \\
        \midrule
        \taburowcolors{white..gray!15}
        % \endfirsthead%
        % \toprule
        % \taburowcolors{white..white}
        % Electoral Criterion & \multicolumn{14}{c}{Electoral System} \\
        %                    \cmidrule{2-15}
        % & \rot{Approval} & \rot{Borda Count} & \rot{Copeland} & \rot{IRV (AV)} & \rot{Kemeny-Young} & \rot{Majority Judgement} & \rot{Minimax} & \rot{Plurality} & \rot{Range Voting} & \rot{Ranked Pairs} & \rot{Runoff Voting} & \rot{Schulze} & \rot{Sortition} & \rot{Random Ballot} \\
        % \midrule
        % \taburowcolors{white..gray!15}
        \endhead\label{tab:criteria-compliance}
        {Majority}             & \bad[Rated]     & \vbad             & \vgood                   & \vgood          & \vgood             & \vgood             & \vbad              & \vgood             & \vgood              & \vgood          & \vbad           & \vbad \\
        {Majority Loser}       & \vbad           & \vgood            & \vgood                   & \vgood          & \vgood             & \vbad              & \vbad              & \vgood             & \vgood              & \vgood          & \vbad           & \vbad \\
        {Mutual Majority}      & \vbad           & \vbad             & \vgood                   & \vgood          & \vgood             & \vbad              & \vbad              & \vgood             & \vbad               & \vgood          & \vbad           & \vbad \\
        {Condorcet}            & \bad            & \vbad             & \vgood                   & \vbad           & \vgood             & \vbad              & \bad               & \vgood             & \vbad               & \vgood          & \vbad           & \vbad \\
        {Condorcet Loser}      & \vbad           & \vgood            & \vgood                   & \vgood          & \vgood             & \vbad              & \vbad              & \vgood             & \vgood              & \vgood          & \vbad           & \vbad \\
        {Smith/IDSA}           & \vbad           & \vbad             & \vgood                   & \vbad           & \vgood             & \vbad              & \vbad              & \vgood             & \vbad               & \vgood          & \vbad           & \vbad \\
        {LIIA}                 & \good           & \vbad             & \vbad                    & \vbad           & \vgood             & \vbad              & \good              & \vgood             & \vbad               & \vbad           & \vgood          & \vgood \\
        {IIA}                  & \good           & \vbad             & \vbad                    & \vbad           & \vbad              & \vbad              & \good              & \vbad              & \vbad               & \vbad           & \vgood          & \vgood \\
        {Cloneproof}           & \good           & \vbad[\teams]     & \vbad[\crowds{}\teams{}] & \vgood          & \vbad[\spoilers]   & \vbad[\spoilers]   & \vgood             & \vgood             & \vbad[\spoilers]    & \vgood          & \vbad           & \vgood \\
                                 \addlinespace[-0.4ex]
                                 \cmidrule(l){2-15}
                                 \addlinespace[-0.80ex]
        {Monotone}             & \vgood          & \vgood            & \vgood                   & \vbad           & \vgood             & \vgood             & \vgood             & \vgood             & \vbad               & \vgood          & \vgood          & \vgood \\
        {Consistency}          & \good           & \vgood            & \vbad                    & \vbad           & \bad               & \vgood             & \good              & \vbad              & \vbad               & \vbad           & \vgood          & \vgood \\
        {Participation}        & \good           & \vgood            & \vbad                    & \vbad           & \vbad              & \vgood             & \good              & \bad               & \vbad               & \bad            & \vgood          & \vgood \\
        {Reversal Symmetry}    & \vgood          & \vgood            & \vgood                   & \vbad           & \vgood             & \vbad              & \vgood             & \vgood             & \vbad               & \vgood          & \vgood          & \vgood \\
                                 \addlinespace[-0.4ex]
                                 \cmidrule(l){2-15}
                                 \addlinespace[-0.80ex]
        {Polytime}             & \vgood[$O(n)$]  & \vgood[$O(n)$]    & \good[$O(n^2)$]          & \good[$O(n^2)$] & \vbad[$O(n!)$]     & \vgood[$O(n)$]     & \vgood[$O(n)$]     & \good[$O(n^3)$]    & \vgood[$O(n)$]      & \good[$O(n^3)$] & \vgood[$O(1)$]  & \vgood[$O(n)$] \\
        {Resolvable}           & \vgood          & \vgood            & \vbad                    & \good           & \vgood             & \vgood             & \vgood             & \vgood             & \vgood              & \vgood          & \vbad           & \vbad \\
        {Summable}             & \vgood[$O(n)$]  & \vgood[$O(n)$]    & \good[$O(n^2)$]          & \vbad[$O(n!)$]  & \good[$O(n^2)$]    & \vgood[$O(n)$]     & \vgood[$O(n)$]     & \good[$O(n^2)$]    & \good[$O(n)$]       & \good[$O(n^2)$] & \vgood[$O(1)$]  & \vgood[$O(n)$] \\
                                 \addlinespace[-0.4ex]
                                 \cmidrule(l){2-15}
                                 \addlinespace[-0.80ex]
        {Later-no-Harm}        & \vbad           & \vbad             & \vbad                    & \vgood          & \vbad              & \meh               & \vbad              & \vbad              & \good               & \vbad           & \vgood          & \vgood \\
        {Later-no-Help}        & \good           & \vgood            & \vbad                    & \vgood          & \vbad              & \meh               & \vgood             & \vbad              & \good               & \vbad           & \vgood          & \vgood \\
        {No Favorite Betrayal} & \vgood          & \vbad             & \vbad                    & \vbad           & \vbad              & \vbad              & \vgood             & \bad               & \vbad               & \bad            & \vgood          & \vgood \\
                                 \addlinespace[-0.4ex]
                                 \cmidrule(l){2-15}
                                 \addlinespace[-0.80ex]
        {Ballot Type}          & \bad[\approval] & \good[\ranked]    & \good[\ranked]           & \good[\ranked]  & \good[\ranked]     & \vbad[\singleMark] & \vgood[\range]     & \good[\ranked]     & \vbad[\singleMark]  & \good[\ranked] & \meh             & \vbad[\singleMark] \\
        {Ranks =}              & \vgood          & \vbad             & \vgood                   & \vbad           & \vgood             & \meh               & \vgood             & \vgood             & \meh                & \vgood         & \meh             & \meh \\
        {Ranks >2}             & \vbad           & \vgood            & \vgood                   & \vgood          & \vgood             & \vbad              & \vgood             & \vgood             & \bad                & \vgood         & \meh             & \vbad \\
        \bottomrule
    \end{longtabu}
    \begin{multicols}{2}
      \begin{itemize}[label=,leftmargin=0mm,topsep=0mm]
        \item \cmark{} Criterion is supported.
        \item \xmark{} Criterion is unsupported.
        \item \notApplicable{} Criterion is not applicable.
        \item \qmark{} Criterion is supported under some conditions.
        \item \range{} Choices are cardinally ranked by preference.
        \item \ranked{} Choices are ordinally ranked by preference.

        \item \spoilers{} Vulnerable to spoilers.
        \item \teams{} Vulnerable to teams.
        \item \crowds{} Vulnerable to crowds.
        \item \approval{} Multiple yes/no choice-selections supported.
        \item \singleMark{} Single yes/no choice-selection supported.
        \item
      \end{itemize}
    \end{multicols}
\end{center}

\paragraph{Absolute Result Criteria}
The absolute criteria express whether a candidate must or must not win given the
state of some ballots.\cite{range-voting-criteria,electoral-system-comparison,electoral-system-comparison-archive}

\subparagraph{Majority Criterion}
The Majority Criterion (MC) states that a candidate who is preferred by a
majority of voters must win. This is expressed in two flavors:
\begin{enumerate}
    \item \emph{Ranked}, when a choice is preferred by a majority of voters,
        the choice must win.
    \item \emph{Rated}, when a choice is given a perfect score by a majority of
        voters, the choice must win.
\end{enumerate}
In a ranked electoral systems these two majority criteria are identical in
nature.

\subparagraph{Mutual Majority Criterion}
The Mutual Majority Criterion (MMC) states that if a subset, \verb|S|, of
candidates is strictly preferred over every candidate in the absolute complement
of subset \verb|S|, then the winner must come from subset \verb|S|.

\subparagraph{Condorcet Criterion}
The Condorcet criterion states that a choice who beats out every other choice in
a pairwise comparison will win.

\subparagraph{Condorcet Loser Criterion}
The Condorcet loser criterion states that a choice who loses to every other
choice in a pairwise comparison will always lose.

\paragraph{Relative Result Criteria}
The relative result criteria express when a candidate should or should not win
given a win in a similar circumstance.\cite{electoral-system-comparison,electoral-system-comparison-archive,range-voting-criteria}

\subparagraph{Independence of Smith-dominated Alternatives}
Independence of Smith-dominated Alternatives (ISDA) states that an added or
removed Smith-dominated choice, one which would lose in direct pairwise
competition with every other choice, will not affect the result of the contest.

% Does the outcome never change if a Smith-dominated candidate is added or
% removed (assuming votes regarding the other candidates are unchanged)?
% Candidate C is Smith-dominated if there is some other candidate A such that C
% is beaten by A and every candidate B that is not beaten by A etc. Note that
% although this criterion is classed here as nominee-relative, it has a strong
% absolute component in excluding Smith-dominated candidates from winning. In
% fact, it implies all of the absolute criteria above.

\subparagraph{Independence of Irrelevant Alternatives}
Independence of Irrelevant Alternatives (IIA) is a criterion which states that
adding or removing a non-winning candidate should not impact the end result.

\subparagraph{Local Independence of Irrelevant Alternatives}
Local Independence of Irrelevant Alternatives (LIIA) is a criterion which states
that removing a candidate will not disrupt the transitive ordering.

\subparagraph{Independence of Clone Alternatives}
The independence of clone alternatives, cloneproof, states that the outcome will
not change if non-winning candidates similar to an existing candidate are added
as choices. There are three flavors which fail independence of clones:
\begin{enumerate}
    \item \emph{Spoilers}, which are clone negative choices that decrease the
      chance of a similar choice winning by spreading votes across multiple
      choices.
    \item \emph{Teams}, which are clone positive similar choices that together
        increase the chance of their winning.
    \item \emph{Crowds}, which are non-winning choices that when cloned change
        the winner without themselves becoming the winner.
\end{enumerate}

\subparagraph{Monotonicity Criterion}
The monotonicity criterion, monotone, states that ranking a winning choice
higher will not impact the end result.

\subparagraph{Consistency Criterion}
The Consistency Criterion (CC) states that a winning choice in two complement
sets of ballots should remain the winner in a final tally combining the two
sets.

\subparagraph{Participation Criterion}
The Participation Criterion (PC) states that voting honestly will always produce
better results than not voting at all.

\subparagraph{Reversal Symmetry}
The reversal symmetry criterion states that when individual voter preferences
are universally inverted the original winner will never win.

\paragraph{Ballot Counting Criteria}
Ballot counting criteria concern the process of vote tallying and winner
determination. These criteria are especially important with respect to the
practical implementation of an electoral system.\cite{electoral-system-analysis,electoral-handbook,pr-computability}

\subparagraph{Polynomial Time}
The polynomial time criterion, polytime, states that the electoral system can
calculate the winner in linear time with respect to the number of voters and in
polynomial time with respect to the number of candidates.

\subparagraph{Resolvable}
A resolvable electoral system is one where determining a winner should be
entirely deterministic, i.e., the electoral system should not depend on random
processes such as coin flipping. This criterion is less important in large
elections where ties are unlikely to occur.

\subparagraph{Summability}
Summability is a criterion used to express how computationally complex it is to
store vote data in a compressed format, and consequentially, how difficult it is
to pre-tally votes at individual polling stations and transmit those tallied
results to a central counting authority for final counting. Votes are expected
to be mapped to a summable array which can be used to determine the winner. The
summability criterion is considered $k$\textsuperscript{th}-order summable if we
can map $n$ candidates to a matrix of size $n^k$. If no $k$ exists the electoral
system is considered non-summable.

\paragraph{Strategy Criteria}
The strategy criteria relate to the incentives and ability for voters to vote
using some strategy to produce a desired election result.\cite{electoral-system-analysis,stv-strategic-voting,electoral-handbook}

\subparagraph{Later-no-Harm Criterion}
This criterion states that ranking a preference later on a ballot will not harm
a choice already listed.

\subparagraph{Later-no-Help Criterion}
This criterion states that ranking a preference later on a ballot will not help
a choice already listed.

\subparagraph{No Favorite Betrayal}
No Favorite Betrayal (NFB) states that ranking a choice above your preferred
choice will not produce a more desirable or preferred result.

\paragraph{Ballot Format}
The ballot formats define how a voter is able to express themselves on a ballot.\cite{electoral-handbook}

\subparagraph{Ballot Type}
The ballot type defines how a voter is permitted to mark their ballot. Popular
ballot types include single mark, approval, ranked (ordinal), and scored
(cardinal).

\subparagraph{Equal Ranks}
A ballot that allows a voter to express equal support for multiple candidates is
said to support equal ranks.

\subparagraph{Over 2 Ranks}
A ballot which allows a voter to express interest for a choice in non-binary
terms is said to support over 2 ranks, e.g., ordinal and cardinal ballots.

% Sub-Sub-Section: Theorems and Paradoxes
% \subsubsection{Theorems and Paradoxes}
There are a number of interesting theorems and paradoxes regarding voting. These
theorems illustrate some of the practical limitations of what electoral systems
can accomplish and are capable of.

\paragraph{Condorcet's Paradox}
When considering governance and voting, Condorcet's paradox is a good place to
start; it grounds one to the difficulties of voting. Condorcet's paradox
demonstrates that the collective preferences of voting actors can be cyclical
despite the individual preferences of the voting actor's choices being
transitive and non-cyclical. The following example illustrates Condorcet's
paradox:

Suppose we have 3 voting actors: $A$, $B$, and $C$ voting for choices $X$, $Y$,
and $Z$. They rank their votes sequentially from 1 to 3 (lower numbers
indicating a more favorable choice).

\begin{center}
    \begin{table}[H]
        \centering\scriptsize
        \caption{A cyclic voter preference profile.} \label{tab:condorcet-paradox}
        \begin{tabu} to 30mm{@{} X[c] | X[c] X[c] X[c] @{}}
            \toprule
            & $X$ & $Y$ & $Z$ \\
            \midrule
            $A$ & 1 & 2 & 3 \\
            $B$ & 2 & 3 & 1 \\
            $C$ & 3 & 1 & 2 \\
            \bottomrule
        \end{tabu} \\[2mm]
        \begin{varwidth}{\textwidth}
            \begin{itemize}[label=,leftmargin=0mm,topsep=5mm]
                \item $A$ favors $X > Y > Z$
                \item $B$ favors $Z > X > Y$
                \item $C$ favors $Y > Z > X$
            \end{itemize}
        \end{varwidth}
    \end{table}
\end{center}


When measuring the preferences of the voters collectively we observe that choice
$X$ is preferred to $Y$, that choice $Y$ is collectively preferred to $Z$, and
that choice $Z$ is collectively preferred to $X$; a seemingly paradoxical result
($X > Y$ and $Y > Z$ and $Z > X$).

\paragraph{Arrow's Impossibility Theorem}
Arrow's impossibility theorem, also known as Arrow's paradox, states that no
rank-order voting method can satisfy the following ``fairness'' conditions:

\begin{enumerate}
    \item There is no dictator.
    \item If every voter prefers $X$ to $Y$ then so does the group.
    \item The relative positions of $X$ and $Y$ in the group ranking depend on
        their relative positions in the individual rankings, but do not
        depend on the individual rankings of any irrelevant alternative $Z$.
\end{enumerate}

The following example demonstrates Arrow's paradox:

\begin{center}
    \begin{table}[H]
        \centering\scriptsize
        \caption{A voter preference profile demonstrating Arrow's paradox.}
        \label{tab:arrows-paradox}
        \begin{tabu} to 30mm{@{} X[c] | X[c] X[c] X[c] X[c] @{}}
            \toprule
            & $X$ & $Y$ & $Z$ \\
            \midrule
            A & 1 & 2 & 3 \\
            B & 1 & 3 & 2 \\
            C & 3 & 1 & 2 \\
            D & 3 & 1 & 2 \\
            E & 3 & 2 & 1 \\
            \bottomrule
        \end{tabu} \\[2mm]
        \begin{varwidth}{\textwidth}
            \begin{itemize}[label=,leftmargin=0mm,topsep=5mm]
                \item A favors $X > Y > Z$
                \item B favors $X > Z > Y$
                \item C favors $Y > Z > X$
                \item D favors $Y > Z > X$
                \item E favors $Z > Y > X$
            \end{itemize}
        \end{varwidth}
    \end{table}
\end{center}

After tallying votes, summing each voter's first-preference, we see that $X$ has
2 votes, $Y$ has 2 votes, and $Z$ has 1 vote. $X$ and $Y$ are tied as first
choice while $Z$ is an irrelevant alternative. One might expect that if $Z$ were
eliminated from the ballot that the election result would not be affected.
However, once done we observe that the election's outcome changes:

\begin{center}
    \begin{table}[H]
        \centering\scriptsize
        \caption{Table \ref{tab:arrows-paradox} with the irrelevant alternative removed.}
        \label{tab:arrows-paradox-remove-alt}
        \begin{tabu} to 30mm{@{} X[c] | X[c] X[c] @{}}
            \toprule
            & $X$ & $Y$ \\
            \midrule
            A & 1 & 2 \\
            B & 1 & 3 \\
            C & 3 & 1 \\
            D & 3 & 1 \\
            E & 3 & 2 \\
            \bottomrule
        \end{tabu} \\[2mm]
        \begin{varwidth}{\textwidth}
            \begin{itemize}[label=,leftmargin=0mm,topsep=5mm]
                \item A favors $X > Y$
                \item B favors $X > Y$
                \item C favors $Y > X$
                \item D favors $Y > X$
                \item E favors $Y > X$
            \end{itemize}
        \end{varwidth}
    \end{table}
\end{center}

$Y$ is now favored over $X$ with 3 votes to 2 votes.

\begin{displayquote}[Kenneth Arrow]
  Most systems are not going to work badly all of the time. All I proved is that
  all can work badly at times.
\end{displayquote}

\paragraph{Gibbard–Satterthwaite Theorem}
The Gibbard–Satterthwaite Theorem is related to Arrow's Impossibility Theorem
and states the following:

\begin{displayquote}
    No single-winner voting method exists (using rank-order ballots) satisfying
    all of the following short list of conditions:

    \begin{enumerate}
        \item There is no ``dictator.''
        \item If every voter ranks $X$ top, then $X$ wins the election.
        \item The voting system is deterministic, i.e. its decision about who
            wins is based purely on the votes, not on random chance.
        \item There are at least three candidates running.
        \item Honest and strategic voting are the same thing, i.e. it never
            ``pays for a voter to lie,'' i.e. (more precisely) there is no
            election situation in which a voter, by submitting a dishonest vote
            claiming $X > Y$ when really she does not agree that $X$ is a better
            candidate than $Y$, can make the election result come out better
            (from her point of view) than if she had voted honestly.
    \end{enumerate}
\end{displayquote}

In other words, any reasonable voting system which leverages rank-order ballots
will always be susceptible to manipulation and strategic voting. Consider the
following example:

\begin{center}
    \begin{table}[H]
        \centering\scriptsize
        \caption{A preference profile demonstrating the Gibbard-Satterthwaite theorem.}
        \label{tab:gh-theorem}
        \begin{tabu} to 30mm{@{} X[c] | X[c] X[c] X[c] X[c] @{}}
            \toprule
            & $X$ & $Y$ & $Z$ \\
            \midrule
            5 & 1 & 2 & 3 \\
            6 & 2 & 3 & 1 \\
            8 & 3 & 1 & 2 \\
            \bottomrule
        \end{tabu} \\[2mm]
        \begin{varwidth}{\textwidth}
            \begin{itemize}[label=,leftmargin=0mm,topsep=5mm]
                \item 5 voters favor $X > Y > Z$
                \item 6 voters favor $Z > X > Y$
                \item 8 voters favor $Y > Z > X$
            \end{itemize}
        \end{varwidth}
    \end{table}
\end{center}

Under most electoral systems choice $Y$ would be the winner. However, if the 6
voters who favored $Z$ changed their votes to $X > Z > Y$, betraying $Z$, then a
choice they may consider a lesser evil, $X$, will win.

