\subsubsection{Theorems and Paradoxes}
There are a number of interesting theorems and paradoxes regarding voting. These
theorems illustrate some of the practical limitations of what electoral systems
can accomplish and are capable of.

\paragraph{Condorcet's Paradox}
When considering governance and voting, Condorcet's paradox is a good place to
start; it grounds one to the difficulties of voting. Condorcet's paradox
demonstrates that the collective preferences of voting actors can be cyclical
despite the individual preferences of the voting actor's choices being
transitive and non-cyclical. The following example illustrates Condorcet's
paradox:

Suppose we have 3 voting actors: $A$, $B$, and $C$ voting for choices $X$, $Y$,
and $Z$. They rank their votes sequentially from 1 to 3 (lower numbers
indicating a more favorable choice).

\begin{center}
    \begin{table}[H]
        \centering\scriptsize
        \caption{A cyclic voter preference profile.} \label{tab:condorcet-paradox}
        \begin{tabu} to 30mm{@{} X[c] | X[c] X[c] X[c] @{}}
            \toprule
            & $X$ & $Y$ & $Z$ \\
            \midrule
            $A$ & 1 & 2 & 3 \\
            $B$ & 2 & 3 & 1 \\
            $C$ & 3 & 1 & 2 \\
            \bottomrule
        \end{tabu} \\[2mm]
        \begin{varwidth}{\textwidth}
            \begin{itemize}[label=,leftmargin=0mm,topsep=5mm]
                \item $A$ favors $X > Y > Z$
                \item $B$ favors $Z > X > Y$
                \item $C$ favors $Y > Z > X$
            \end{itemize}
        \end{varwidth}
    \end{table}
\end{center}


When measuring the preferences of the voters collectively we observe that choice
$X$ is preferred to $Y$, that choice $Y$ is collectively preferred to $Z$, and
that choice $Z$ is collectively preferred to $X$; a seemingly paradoxical result
($X > Y$ and $Y > Z$ and $Z > X$).

\paragraph{Arrow's Impossibility Theorem}
Arrow's impossibility theorem, also known as Arrow's paradox, states that no
rank-order voting method can satisfy the following ``fairness'' conditions:

\begin{enumerate}
    \item There is no dictator.
    \item If every voter prefers $X$ to $Y$ then so does the group.
    \item The relative positions of $X$ and $Y$ in the group ranking depend on
        their relative positions in the individual rankings, but do not
        depend on the individual rankings of any irrelevant alternative $Z$.
\end{enumerate}

The following example demonstrates Arrow's paradox:

\begin{center}
    \begin{table}[H]
        \centering\scriptsize
        \caption{A voter preference profile demonstrating Arrow's paradox.}
        \label{tab:arrows-paradox}
        \begin{tabu} to 30mm{@{} X[c] | X[c] X[c] X[c] X[c] @{}}
            \toprule
            & $X$ & $Y$ & $Z$ \\
            \midrule
            A & 1 & 2 & 3 \\
            B & 1 & 3 & 2 \\
            C & 3 & 1 & 2 \\
            D & 3 & 1 & 2 \\
            E & 3 & 2 & 1 \\
            \bottomrule
        \end{tabu} \\[2mm]
        \begin{varwidth}{\textwidth}
            \begin{itemize}[label=,leftmargin=0mm,topsep=5mm]
                \item A favors $X > Y > Z$
                \item B favors $X > Z > Y$
                \item C favors $Y > Z > X$
                \item D favors $Y > Z > X$
                \item E favors $Z > Y > X$
            \end{itemize}
        \end{varwidth}
    \end{table}
\end{center}

After tallying votes, summing each voter's first-preference, we see that $X$ has
2 votes, $Y$ has 2 votes, and $Z$ has 1 vote. $X$ and $Y$ are tied as first
choice while $Z$ is an irrelevant alternative. One might expect that if $Z$ were
eliminated from the ballot that the election result would not be affected.
However, once done we observe that the election's outcome changes:

\begin{center}
    \begin{table}[H]
        \centering\scriptsize
        \caption{Table \ref{tab:arrows-paradox} with the irrelevant alternative removed.}
        \label{tab:arrows-paradox-remove-alt}
        \begin{tabu} to 30mm{@{} X[c] | X[c] X[c] @{}}
            \toprule
            & $X$ & $Y$ \\
            \midrule
            A & 1 & 2 \\
            B & 1 & 3 \\
            C & 3 & 1 \\
            D & 3 & 1 \\
            E & 3 & 2 \\
            \bottomrule
        \end{tabu} \\[2mm]
        \begin{varwidth}{\textwidth}
            \begin{itemize}[label=,leftmargin=0mm,topsep=5mm]
                \item A favors $X > Y$
                \item B favors $X > Y$
                \item C favors $Y > X$
                \item D favors $Y > X$
                \item E favors $Y > X$
            \end{itemize}
        \end{varwidth}
    \end{table}
\end{center}

$Y$ is now favored over $X$ with 3 votes to 2 votes.

\begin{displayquote}[Kenneth Arrow]
  Most systems are not going to work badly all of the time. All I proved is that
  all can work badly at times.
\end{displayquote}

\paragraph{Gibbard–Satterthwaite Theorem}
The Gibbard–Satterthwaite Theorem is related to Arrow's Impossibility Theorem
and states the following:

\begin{displayquote}
    No single-winner voting method exists (using rank-order ballots) satisfying
    all of the following short list of conditions:

    \begin{enumerate}
        \item There is no ``dictator.''
        \item If every voter ranks $X$ top, then $X$ wins the election.
        \item The voting system is deterministic, i.e. its decision about who
            wins is based purely on the votes, not on random chance.
        \item There are at least three candidates running.
        \item Honest and strategic voting are the same thing, i.e. it never
            ``pays for a voter to lie,'' i.e. (more precisely) there is no
            election situation in which a voter, by submitting a dishonest vote
            claiming $X > Y$ when really she does not agree that $X$ is a better
            candidate than $Y$, can make the election result come out better
            (from her point of view) than if she had voted honestly.
    \end{enumerate}
\end{displayquote}

In other words, any reasonable voting system which leverages rank-order ballots
will always be susceptible to manipulation and strategic voting. Consider the
following example:

\begin{center}
    \begin{table}[H]
        \centering\scriptsize
        \caption{A preference profile demonstrating the Gibbard-Satterthwaite theorem.}
        \label{tab:gh-theorem}
        \begin{tabu} to 30mm{@{} X[c] | X[c] X[c] X[c] X[c] @{}}
            \toprule
            & $X$ & $Y$ & $Z$ \\
            \midrule
            5 & 1 & 2 & 3 \\
            6 & 2 & 3 & 1 \\
            8 & 3 & 1 & 2 \\
            \bottomrule
        \end{tabu} \\[2mm]
        \begin{varwidth}{\textwidth}
            \begin{itemize}[label=,leftmargin=0mm,topsep=5mm]
                \item 5 voters favor $X > Y > Z$
                \item 6 voters favor $Z > X > Y$
                \item 8 voters favor $Y > Z > X$
            \end{itemize}
        \end{varwidth}
    \end{table}
\end{center}

Under most electoral systems choice $Y$ would be the winner. However, if the 6
voters who favored $Z$ changed their votes to $X > Z > Y$, betraying $Z$, then a
choice they may consider a lesser evil, $X$, will win.
