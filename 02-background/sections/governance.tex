\section{Governance}
Governance is the means through which organizations and communities are
managed and decisions are made; it is the governance processes and the actions
produced through these processes which allow organizations and communities to
sustain themselves and achieve their goals. Governance is a process which has
been refined, tweaked, and experimented with since the dawn of civilization.
There is no perfect governance system, only systematic trade-offs that must be
made while weighing the pros, cons, and demands of each. Ultimately the best
government is going to depend on the processes and actions that are most
convenient for its constituents while still effectively achieving its goals.
Most would argue that an ideal government would achieve its goals while
providing to its constituents maximum benefit, freedom, representation, and
protections; however, these are often at odds with one another and in opposition
to the goals of the government itself.

When analyzing governance models it is useful to consider the kinds of people
who wield power and are responsible for decision making in said system. There is
a wide multidimensional spectrum under which wielders of power might exist and
decision-making responsibilities assigned. Table~\ref{tab:forms-of-rulers}
enumerates a few classes of people who might wield power in government.

\begin{table}[H]
    \scriptsize
    \caption{Forms of Rulers}\label{tab:forms-of-rulers}
    \begin{tabu} to \textwidth{@{} l l l @{}}
        \toprule
        {Term}              & {Definition} \\
        \midrule
        Kraterocracy        & Rule by the strong, ``might makes right.'' \\
        Plutocracy          & Rule by the wealthy. \\
        Partocracy          & Rule by political parties. \\
        Geniocracy          & Rule by the intelligent. \\
        Meritocracy         & Rule by the meritorious. \\
        Technocracy         & Rule by experts. \\
        Timocracy           & Rule by the honourable. \\
        Autocracy           & Rule by a single individual. \\
        Oligarchy           & Rule by a small group of powerful individuals. \\
        Monarchy            & Rule by family. \\
        Democracy           & Rule by the people, enfranchized citizens. \\
        Anarchy             & Rule by none. \\
        \bottomrule
    \end{tabu}
\end{table}

One might also consider how decision-making responsibilities are organized and
assigned. Table~\ref{tab:forms-of-governance} enumerates some more granular
governance models.

\begin{table}[H]
    \scriptsize
    \caption{Forms of Governance}\label{tab:forms-of-governance}
    \begin{tabu} to \textwidth{l X}
        \toprule
        {Term}                          & {Definition} \\
        \midrule
        Demarchy                        & Decisions are made by randomly selected citizens a pool of enfranchised citizens (sortition). \\
        Direct Democracy                & Decisions are made by citizens vote on policies directly. \\
        Representative democracy        & Decisions are made by representatives elected by enfranchised citizens. \\
        Liberal democracy               & A form of representative democracy that attempts which operates under the ideals of liberalism. \\
        Social democracy                & A socio-economic democracy which supports economic and social intervention within the framework of a capitalist society. \\
        Electocracy                     & Decisions are entirely made by a government elected by citizens. \\
        Totalitarian democracy          & Decisions are made entirely by representatives elected by citizens. \\
        Constitutional monarchy         & A monarchy where the monarch's powers are limited by a constitution. \\
        Crowned republic                & A republic with a ceremonial monarchy. \\
        Absolute monarchy               & Rule by family. \\
        \bottomrule
    \end{tabu}
\end{table}

%%
% TODO: Improve this last section...
%%

These modalities of governance are often seen operating in conjunction with one
another, especially in larger governments, e.g., nation-states, where varying
degrees of federation can be used to balance the different features and
characteristics of the governance model with the different needs and
requirements which are demanded of the organization at hand.

% Different modalities of governance may be better suited for the different
% requirements and needs of different organizations.

% This is a feature.
% Under large governments there often exists some level of federation through
% which the governments operate.

\subsection{Democratic Governance}

% All democratic organizations and communities have power structures through
% which decisions are made.

% The goal of any democracy should be to represent the will of its constituents.

This research focuses on democratic governance models. One of the primary
objectives of democratic organizations is to accurately represent the will of
its constituents; to that end, democratic governance models should strive to
reach strong consensus throughout their organizations to the maximum extent
possible, as often as reasonable. There are many opinions concerning how
democratic communities should be structured and decision-making authority
organized within these communities, but almost universally democratic governance
demands suffrage. Voting is generally regarded as one of the most fundamental
rights within democratic organizations and as such is strongly protected. Voting
exists as one of the most straightforward and effective means for individuals to
express their will. However, designing a electoral system that \emph{fairly}
represents the will of its people is a non-trivial task and presents complex and
unintuitive pitfalls.

% * active suffrage
% * universal suffrage
% * the paradox of voting

% Democracy is often considered a ``rule of the majority,'' however, a rule of
% the majority naturally implies the potential for a tyranny of the majority.
% Thus, chief among the concerns of democratic governments should be how best to
% provide protections for minority groups.

\subsubsection{Common Modalities}

Here we review some of the more common democratic organizational modalities.


\paragraph{Direct Democracy}
Direct democracy, also referred to as pure democracy, is perhaps the most
straightforward democratic modality, one where individuals vote directly on
issues and policies to influence the decision-making process. Referendums are
examples of direct democracy.

\subparagraph{Advantages} Direct democracies enable the electorate of the
government to express their will in a direct and immediate way. Direct
democracies are more effective when the number of decisions to be voted on and
the amount of foundational knowledge required to make informed decisions is low.

\subparagraph{Disadvantages} Direct democracies fail to scale well as
issues, issue complexity, or voter base increases; it is not feasible for
individuals to be constantly required to vote on issues or to stay informed
regarding complex policies. Further, the cost of conducting standard elections
across large populations is a generally expensive procedure, and doing so
regularly for all manners of issues would quickly become prohibitively
expensive (financially and temporally).


\paragraph{Demarchy}
Demarchy, otherwise referred to as allotment or sortition, is the process of
assigning decision-making responsibility to randomly selected representatives of
a larger population. Sortition dates back to Athenian democracy and has roots in
the idea that ``power corrupts.'' Modern examples of sortition can be seen in
courtrooms juries.

\subparagraph{Advantages} Sortition can be leveraged to improve the scalability
and integrity of democratic decision-making. By localizing decision-making
authority to a small group sortition can significantly decrease the burden that
is placed on society at large to stay informed on policy and participate in
decision-making. Further, where decision-making does occur, it can occur much
more quickly and on a much smaller and less expensive scale than that of a
direct democracy. When applied at regular intervals, sortition can also
significantly decreases opportunities for persistent and systemic forms of
corruption and oppression to solidify, thus improving the integrity of the
democratic process.

\subparagraph{Disadvantages} While sortition \emph{technically} resolves most of
the issues related to election scalability, it does not resolve any of them
particularly well. Straightforward implementations of sortition make no attempt
to bias decision-making authority towards individuals who are interested or
proficient in legislation, policy, or decision-making --- which in turn results
in lower participation rates --- nor does is it attempt to bias decision-making
authority towards individuals who are proficient or at all qualified to make
decisions regarding the field in question. The random and probabilistic nature
of sortition suggests that it would, at least to some degree, provide a
generally fair and representative outcome; however, in the long-term
sortition-based democracies will bias towards majority groups, and in the
short-term presents opportunities for representatives to be selected who are
ignorant, apathetic, or otherwise unlikely to represent the will of the greater
population which they are a part of.

% Sortition means lower participation rates.
% See: https://www.houstonchronicle.com/news/houston-texas/houston/article/Battling-the-jury-duty-problem-where-fewer-than-15010187.php
%
% Sortition fails to empower minority groups.
% See: https://www.americanbar.org/groups/litigation/committees/diversity-inclusion/articles/2015/lack-of-jury-diversity-national-problem-individual-consequences/

% Democratic systems where the constituents of the system must be


\paragraph{Representative Democracy}
In a representative democracy the choices in a election are candidates,
individuals who the electorate yield decision-making authority to for some
period of time. These candidates are meant to represent their constituents and
vote on their behalf on policy and legislative concerns. Most large democratic
governments are composed to some degree or another of representative
democracies. The United States government is one example of such a government.

\subparagraph{Advantages} Representative democracies are often presented as the
most efficient form of democracy for large populations; it is unquestionably
more efficient than sortition at biasing decision-making power towards
individuals with a proficiency and desire to participate in decision making. It
does a well at maintaining a relatively small number of people who need to
participate in the decision making process.

\subparagraph{Disadvantages} Not without its own set of criticisms,
representative democracies are known for falling prey to the ``iron law of
oligarchy,'' an idea which claim that representative democracies deteriorate
towards oligarchical or particrical modalities over time. Further,
representative democracies do not generally not bias decision-making power
towards individuals who are knowledgeable in the fields that decisions are being
made in. It remains impractical for \emph{any} single individual to be
knowledgeable in all fields of policy in today's world. As an example, policy
generally lags far behind modern technology not just due to slow and complex
bureaucratic processes but also due to representatives not being in positions of
understanding to draft policy.

\subsubsection{Uncommon Modalities}

The democratic modalities discussed here, and literature surrounding these
modalities, is still relatively nascent when compared to those previously
reviewed. Thus, the language used to describe these modalities is still evolving
and as such the language seen elsewhere may not align with what is written here.

\paragraph{Delegative Democracy}
Delegative democracies are generally described as lying somewhere between direct
and representative democracies. Rather than voting for representatives, the
electorate of a delegative democracy can vest, delegate, or proxy --- these
words are used interchangeably --- their vote to a delegate, another member of
the electorate. A key property of this delegation feature, which distinguishes
it from representation as understood from a representative democracy, is that it
is inherently transitive, i.e., a delegate on the receiving-end of some number
of votes can proxy those votes to a another delegate, and the delegate on the
receiving-end of that delegation can proxy those votes further still to another
delegate.

%%
% TODO: Examples IRL.
% There are examples of this with the pirate party in Sweden.
%%

Additionally electorate can take back, transfer, or override any vote they have
proxied to a delegate for any given issue being voted on. This makes direct
involvement optional. Votes can percolate naturally to delegates who are well
informed on issues or may have a strong history of policy that produces positive
outcomes. Delegates may have specific term limits or might be delegated votes in
a fluid and ongoing process. All voters are free to fluidly abstain or partake
in the process of registering and becoming a delegate. There is no requirement
of campaigning and no notion of ``winning'' an election. All votes made by
individuals acting only on behalf of themselves are kept private to prevent
voter intimidation, while delegates acting on behalf of others are required to
cast ballots publicly to ensure accountability (or at least in a form that is
viewable to the electorate that they represent). Finally, a delegative system
might be explicitly structured in such a way as to allow for specialization;
e.g., Alice delegates her vote to Bob only for issues concerning Economics, and
delegates her vote to Charlie for issues concerning Climate Change and Wildlife
Preservation.

\subparagraph{Advantages}
Thus delegative democracies have the following properties:

\begin{itemize}
    \item Choice of Role
    \item Low Barrier to Participation
    \item Delegated Authority
    \item Privacy of the Individual
    \item Accountability of the Delegates
    \item Specialization by Re-Delegation
\end{itemize}

\subparagraph{Disadvantages}
Delegative democracies have been criticized for their complexity. Computers and
Internet infrastructure would be requirement for this sort of democratic model
to function realistically. This would likely make it an infeasible governance
model for areas of the world where computer technologies are not readily
available.

\paragraph{Liquid Democracy}
Liquid democracies operate similarly to delegative democracies. Where delegative
democracies operate as push-based systems, like SMTP (email), liquid democracies
operate as pull-based systems, like HTTP.\ Ballots can be cast publicly or in a
pseudonymized way by electorate, others may then use that data to
programmatically vote on issues; e.g., Bob, Charlie, David, and Faythe all
publish their ballots publicly, Alice can have her vote automatically cast as
the majority choice tallied across Bob, Charlie, David, and Faythe or default to
Faythe's choice when no majority consensus is determined.

%%
% TODO: Finish advantages and disadvantages.
%%

\subparagraph{Advantages}
\todo{Finish documenting advantages.}

\subparagraph{Disadvantages}
Like delegative democracies, liquid democracies have been criticized for their
complexity and because computers and Internet infrastructure would likely be a
requirement for this sort of democratic model to function.

% \paragraph{Consensus Democracy}
% Add a section here!!!

%%
% \subsection{Summary}
% TODO: Consider adding a summary of electoral systems.
%%


% Demarchy
% Technocracy         % & Rule by experts. \\
% Geniocracy          % & Rule by the intelligent. \\
% Meritocracy         % & Rule by the meritorious. \\
% Futurarchy
%
% \subsection{Voting}
% Perhaps the most natural and intuitive means by which to allow a democracy's
% constituents to express their will is through the right to vote.
% * Paradox of voting
%
%
% Governments might be best grouped by three attributes: those in power, those
% whose elect, and the distribution of responsibility among those in power.
%
% Those in power:
%
% % \url{https://en.wikipedia.org/wiki/Forms_of_government#By_elements_of_where_decision-making_power_is_held}
%
%
% How decisions are made:
% \begin{table}[H]
%     \resizebox{\columnwidth}{!}{%
%         \begin{tabu} to \textwidth{@{} l l l @{}}
%             \toprule
%             {Term}                          & {Definition} \\
%             \midrule
%             Demarchy                        & Decisions are made by randomly selected citizens a pool of enfranchised citizens (sortition). \\
%             Direct Democracy                & Decisions are made by citizens vote on policies directly. \\
%             Representative democracy        & Decisions are made by representatives elected by enfranchised citizens. \\
%             Liberal democracy               & A form of representative democracy that attempts which operates under the ideals of liberalism. \\
%             Social democracy                & A socio-economic democracy which supports economic and social intervention within the framework of a capitalist society. \\
%             Electocracy                     & Decisions are entirely made by a government elected by citizens. \\
%             Totalitarian democracy          & Decisions are made entirely by representatives elected by citizens. \\
%             Constitutional monarchy         & A monarchy where the monarch's powers are limited by a constitution. \\
%             Crowned republic                & A republic with a ceremonial monarchy. \\
%             Absolute monarchy               & Rule by family. \\
%             \bottomrule
%         \end{tabu}
%     }
% \end{table}
%
% Distribution of power lies mostly along a single axis from democracy to
% autocracy. Between there usually exists some level of federation which
% governments operate under.
