\section{Requirements}\label{sec:requirements}
Borrowing from the requirements introduced and defined by, ``The Future of
Voting: End-to-End Verifiable Internet Voting''\cite{e2e-viv} --- which were
outlined in Chapter~\ref{chap:literature}, \emph{\nameref{chap:literature}}, and
grouped into two major categories: technical and non-functional --- a set of
requirements are identified and targeted for satisfaction. These requirements
provide a framework to analyze the results produced by this research.
Table~\ref{tab:research-requirements} summarizes the requirements targeted for
fulfillment.

% % \NewDocumentCommand{\vgood}{ O{\cmark} }{\cellcolor{green!40}#1}
% % \begin{minipage}[t]{1\linewidth}
%   \begin{minipage}[t]{0.50\textwidth}
%     \begin{enumerate}[label=, leftmargin=0mm, topsep=2mm, align=left]
%       % \small
%       \normalsize
%       \item \textbf{Technical Requirements}
%       \item[\faCheck]        authentication requirements
%       \item[\faCheck]        system operational requirements
%       \item[\faPlus]         reliability requirements
%       \item[\faPlus]         security requirements
%       \item[\faPlus]         auditing requirements
%       \item[\faPlus]         functional requirements
%       \item[\faMinus]        interoperability requirements
%       \item[\faMinus]        certification requirements
%       \item[\faRemove]       accessibility requirements
%       \item[\faRemove]       usability requirements
%     \end{enumerate}
%   \end{minipage}
%   \begin{minipage}[t]{0.50\textwidth}
%     \begin{enumerate}[label=, leftmargin=0mm, topsep=2mm, align=left]
%       % \small
%       \normalsize
%       \item \textbf{Non-Functional Requirements}
%       \item[\faPlus]         maintenance/evolvability requirements
%       \item[\faPlus]         assurance requirements
%       \item[\faRemove]       operational requirements
%       \item[\faRemove]       procedural requirements
%       \item[\faRemove]       legal requirements
%     \end{enumerate}
%   \end{minipage}\\
%   % \begin{multicols}{2}
%   % \begin{itemize}[label=,leftmargin=7mm,topsep=-40mm]
%   \begin{itemize}[label=, leftmargin=0mm, topsep=3mm]
%     % Reduce item separation.
%     % \setlength\itemsep{-5mm}
%     \scriptsize
%     \footnotesize
%     \item[\faCheck]        Requirements which are targeted to be \emph{fully} satisfied.
%     \item[\faPlus]         Requirements which are targeted to be \emph{mostly} satisfied.
%     \item[\faMinus]        Requirements which are targeted to be \emph{partially} satisfied.
%     \item[\faRemove]       Requirements which are expected to remain \emph{unsatisfied}.
%   \end{itemize}
%   % \end{multicols}
% % \end{minipage}

\begin{spacing}{1.1}
  \begin{longtabu} to \textwidth{X[0.1,l] X[2.3,j] X[0.8,c] X[0.8,c] X[0.8,c] X[1.1,c]}
    % \multicolumn{2}{c}{Requirements} \\
    \caption{Requirements targeted for fulfillment.} \\
    \toprule
    \multicolumn{2}{l}{{Requirements}}                                       & \multicolumn{3}{c}{Targeted}                  & {Untargeted} \\
                                                                               \cmidrule{3-5}
    \multicolumn{2}{l}{}                                                     & {Fully}       & {Mostly}       & {Partially}  &               \\
    \midrule
    \endhead
    \multicolumn{2}{l}{\emph{Technical}} \\
    % \addlinespace[-0.4ex]
    % \cmidrule(lr){1-2} \cmidrule(lr){3-5} \cmidrule(lr){6-6}
    % \addlinespace[-0.4ex]
    % \cline{1-2}
    % \midrule
    % \multirow{10}{*}{\rot[90]{Technical}}     & Authentication               & \textbullet{} &                &               &               \\
                                              & Authentication               & \textbullet{} &                &               &               \\
                                              & System Operational           & \textbullet{} &                &               &               \\
                                              & Reliability                  &               & \textbullet{}  &               &               \\
                                              & Security                     &               & \textbullet{}  &               &               \\
                                              & Auditing                     &               & \textbullet{}  &               &               \\
                                              & Functional                   &               & \textbullet{}  &               &               \\
                                              & Interoperability             &               &                & \textbullet{} &               \\
                                              & Certification                &               &                & \textbullet{} &               \\
                                              & Accessibility                &               &                &               & \textbullet{} \\
                                              & Usability                    &               &                &               & \textbullet{} \\
    % \midrule
    % \cmidrule(lr){1-2} \cmidrule(lr){3-5} \cmidrule(lr){6-6}
    \addlinespace[0.4ex]
    \cmidrule(lr){1-2} \cmidrule(lr){3-5} \cmidrule(lr){6-6}
    \multicolumn{2}{l}{\emph{Non-Functional}} \\
    % \cline{1-2}
    % \cmidrule{1-2}
    % \midrule
    % \multirow{5}{*}{\rot[90]{Non-Functional}} & Maintenance/Evolvability     &               & \textbullet{}  &               &               \\
                                              & Maintenance/Evolvability     &               &                & \textbullet{} &               \\
                                              & Assurance                    &               &                & \textbullet{} &               \\
                                              & Operational                  &               &                &               & \textbullet{} \\
                                              & Procedural                   &               &                &               & \textbullet{} \\
                                              & Legal                        &               &                &               & \textbullet{} \\
    \bottomrule\label{tab:research-requirements}
  \end{longtabu}
\end{spacing}

The requirements identified in ``The Future of Voting: End-to-End Verifiable
Internet Voting'' have rightfully demanding and difficult to meet criteria for
fulfillment; however, the expectations laid out for a system of the kind
imagined in the study are far beyond the scope of this research. Therefore, a
subset of the technical and non-functional requirements are identified which
are considered achievable, relevant, and within the scope of this research. The
requirements are individually assessed and categorized based on whether their
criteria for satisfaction are considered feasible to fully, mostly, or partially
satisfy. The remaining requirements are considered either outside of the scope
of this research or not feasible to otherwise fulfill and are therefore not
targeted. Several requirements are identified which are considered partially or
mostly satisfied by virtue of intrinsic properties made available through the
underlying blockchain technology.

% \begin{center}
% \begin{spacing}{1.5}
%   \begin{longtabu} to \textwidth{ X[1] X[1] }
%     % \multicolumn{2}{c}{Requirements} \\
%     \toprule
%     {Technical Requirements}        & {Non-Functional Requirements}        \\
%     \midrule
%     \endhead
%     \faCheck{}  authentication      & \faPlus{}   maintenance/evolvability \\
%     \faCheck{}  system operational  & \faPlus{}   assurance                \\
%     \faPlus{}   reliability         & \faRemove{} operational              \\
%     \faPlus{}   security            & \faRemove{} procedural               \\
%     \faPlus{}   auditing            & \faRemove{} legal                    \\
%     \faPlus{}   functional          &                                      \\
%     \faMinus{}  interoperability    &                                      \\
%     \faMinus{}  certification       &                                      \\
%     \faRemove{} accessibility       &                                      \\
%     \faRemove{} usability           &                                      \\
%     \bottomrule
%   \end{longtabu}
% % \begin{multicols}{2}
%   \begin{itemize}[label=,leftmargin=0mm,topsep=0mm]
%     % Reduce item separation.
%     % \setlength\itemsep{-5mm}
%     \scriptsize
%     \footnotesize
%     \item[\faCheck]        Requirements which are targeted to be \emph{fully} satisfied.
%     \item[\faPlus]         Requirements which are targeted to be \emph{mostly} satisfied.
%     \item[\faMinus]        Requirements which are targeted to be \emph{partially} satisfied.
%     \item[\faRemove]       Requirements which are expected to remain \emph{unsatisfied}.
%   \end{itemize}
% % \end{multicols}
% \end{spacing}
% \end{center}

\subsection{Technical Requirements}
The technical requirements are requirements which the study asserts should be
fulfilled through the design and architecture of the system. This research
attempts to fully satisfy the authentication and system operational
requirements; considers the accessibility and usability requirements beyond the
scope of this research, and therefore makes no effort to satisfy them; and
targets partial satisfaction of the remaining requirements. Requirements which
are targeted to be partially satisfied are reviewed in further detail.

\paragraph{Reliability and Security Requirements}
Reliability and security requirements are considered \emph{mostly} satisfiable
due to the inherent properties made available by Ethereum and its decentralized
structure; however, there are some notable potential attacks on the network
which are worth considering; these are discussed in further detail in
Chapter~\ref{chap:results}, \emph{\nameref{chap:results}}.

\paragraph{Auditability Requirements}
Ethereum inherently offers a tremendously detailed log through its blockchain
structure which provides a wide range of opportunities for detailed auditing and
validation. Furthermore, auditing and validation functionality are inherent
features of the blockchain. However, the auditability which Ethereum provides is
not considered a privacy preserving feature and therefore falls short of the
criteria presented.

\paragraph{Functional Requirements}
A subset of the functional requirements are deemed relevant to this research:

\begin{itemize}[topsep=0pt]
  \item[\faCheck] Multi-vote functionality is targeted to be investigated and
    fully satisfied in all electoral system implementations.
    % However, the receipt freedom which it is generally meant to reflect is not
    % satisfied along with it.

  \item[\faPlus] Ensuring that ``a voter's ballot, and the act of them casting a
    ballot, is recorded and retained as expected;'' is considered \emph{mostly}
    fulfilled through intrinsic properties of the Ethereum blockchain. Where
    this property falls short is when a voter casts a ballot, broadcasts a
    transaction to the network, which miners fail to include in any blocks. This
    is discussed further in Chapter~\ref{chap:results},
    \emph{\nameref{chap:results}}.

  \item[\faMinus] Maintaining voter anonymity is considered \emph{partially}
    fulfilled through intrinsic properties of the Ethereum blockchain. While it
    is \emph{technically} possible to broadcast privacy-maintaining
    transactions, those which preserve anonymity on the Ethereum network, it is
    likely beyond the capabilities of most voting actors and would be difficult
    to guarantee in any meaningful way. Pseudo-anonymity is a better profile of
    the kind of anonymity which Ethereum offers, This is discussed further in
    Chapter~\ref{chap:results}, \emph{\nameref{chap:results}}.
\end{itemize}

\subsection{Non-Functional Requirements}
The non-functional requirements are those which the study asserts must be
fulfilled by entities external to the system itself. These requirements are
regarded as being mostly beyond the scope and context of this research, however,
the existence of this document does itself at least \emph{partially} satisfy
some of the categories of requirements, e.g., assurance, maintenance, and
evolvability. Detailed documentation regarding design, architecture and
implementation is provided in Appendix~\ref{appendix:documentation},
\emph{\nameref{appendix:documentation}}.
