\section{Assumptions and Objectives}\label{sec:objectives}
Given the nascent nature of blockchain technologies, it is difficult to imagine
the demands and needs of the organizations and communities which will come to
exist in these cyberspace environments; and therefore difficult to predict which
governance models and electoral systems would best support them. However, it
seems likely that the needs of these organizations will vary as dramatically by
organization as the organizations which have come to exist in meatspace
environments. It therefore follows that a wide range of kinds of electoral
systems should exist which would serve to support the various governance models
and decision-making processes that are likely to form over time. Given these
assumptions, and in compliance with the aforementioned requirements, \emph{the
objectives of this research are defined as follows}:

\begin{enumerate}[topsep=0mm, leftmargin=*]
  \item This research aims to explore various electoral systems and features,
    using the content reviewed in Section~\ref{sec:elections} as a basis for
    exploration.

    \begin{enumerate}
      \item Provide a range of kinds of electoral contracts using
        Table~\ref{tab:electoral-systems} and
        Table~\ref{tab:criteria-compliance} as the primary resource for
        electoral system and feature selection.

        \begin{enumerate}
          \item Support single-winner elections and provide plurality and
            majority electoral system in fulfillment of that objective.

            \begin{enumerate}
              \item Implement first-past-the-post to fulfill the targeted
                single-winner plurality electoral system, due mostly to the fact
                that it is extremely simple, well-understood, and
                widely-adopted.

              \item Implement range vote to fulfill the targeted single-winner
                majority electoral system, due to its effectiveness and
                simplicity.
            \end{enumerate}

          \item Support multi-winner elections and provide an electoral system
            which offers proportional representation; Single transferable vote
            is selected as the electoral system implementation to fulfill both
            objectives due to its wide-adoption, prevalence, and effectiveness
            as a PR system.
        \end{enumerate}

      \item Explore the design, implementation, and efficacy of various ballot
        types across the various electoral systems selected.
    \end{enumerate}

  \item This research aims to explore voter authentication, registration, and
    access control patterns for managing voter registration.

  \item This research aims to explore delegative or liquid governance models.
\end{enumerate}
