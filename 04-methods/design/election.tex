\subsection{Election Components}
The election components are responsible for constructing and operating election
processes. These responsibilities include maintaining the integrity and security
of the election, handling votes, tallying ballots, and determining election
winners.

The phases typically involved in conducting an Internet-based election,
described in Section~\ref{sec:internet-voting}, are: Setup, Distribution,
Voting, Casting, Tallying, and Auditing. Responsibilities pertaining to voter
registration, which are typically handled in the Setup phase, are expected to be
managed by Authorization components. The Distribution phase, which mostly
pertains to processes such as mailing election materials to voters, is
considered outside the scope of this research and therefore ignored. The
Auditing phase, where election integrity and results are scrutinized, is assumed
to be handled and provided by the underlying features made available by
Ethereum.

% Ideally the ballots would be recorded separately from the tallying method. If
% they were we could apply different tallying methods to the same set of ballots
% and produce different results. Unfortunately it's extremely costly to store
% ballots as part of the contract state.

% We opt for first-past-the-post and range vote for single-winner elections, the
% former for its simplicity and ubiquity, the latter for its effectiveness. This
% provides an implementation of a plurality voting system and majority voting
% system.

\subsubsection{Design}
Each electoral system implementation is unique, due mostly to the unique
requirements necessitated by their underlying tallying algorithms and ballot
structures.

There are important electoral criteria worth considering when analyzing the
implementation feasibility of these electoral systems, namely, the ballot
counting criteria: summability, polynomial time, and resolvable. These electoral
criteria are significant because they impact the gas costs associated with
operating the electoral system. Table~\ref{tab:criteria-compliance} is a useful
resource when considering these electoral criteria. Both the first-past-the-post
and range vote electoral systems are a 1\textsuperscript{st}-order summable;
i.e., ballots can be tallied using a linear amount of space with respect to the
number of choices available without losing information required to complete the
tallying processes. Both also have polytime ballot counting implementations that
are linear in the number of candidates and voters. STV is not listed in the
table; however, its single-winner equivalent (IRV) is; which implies that STV
has, at best, a tallying complexity of $O(n^2)$ and a summability complexity
of $O(n!)$ with respect to the number candidates.

Gas costs must be considered when implementing, deploying, and running contract
functions on the Ethereum Virtual Machine (EVM). If gas costs are too high then
it may become infeasible for an external account to afford to execute a contract
function, e.g., the network fees become too high to vote. Another concern to
consider is that the cost to execute a contract function may become too high to
ever complete in a single block, therefore never actually complete. The most
expensive operations, by no small margin, are operations involving storage.
Therefore it is important to minimize storage usage.

% Fortunately these EVM limitations map well
% to many voting method criteria. We can lean on these criteria and the field of
% computational social choice theory broadly to restrict the set of voting systems
% to just the ones that can be feasibly implemented on the Ethereum blockchain.

\paragraph{Finite State Machine}
Several of the voting system designs use a phased approach with timed
transitions between election states: Configuration, Frozen, Vote, and Tally.
These phases can be modeled as a finite state machine where calls to various
functions trigger transitions to different stages. A modifier function can be
used to restrict access of functions to only be callable during their
appropriate state. Most of the transitions from state to state can be designed
as timed transitions which occur lazily, i.e., the check to confirm that it is
time to transition from one state to another occurs when a function is called.
The rationale for this approach is due to the fact that Ethereum offers no
mechanism to trigger code execution \emph{from within the blockchain itself};
therefore all function calls, even if called from another function or even a
function in a different contract, must ultimately have originated from a
transaction created by an externally owned account. The product of this timed
transition implementation implies that the state restriction modifier needs to
occur after the lazily evaluated timed transition modifier; i.e., first update
the current state if it should be, then validate if the function should
executed. On the other hand, if transitions are manually updated during a
function's execution then the opposite behavior should occur; i.e., first
validate if the function should be run, then update the state as required.

\subparagraph{Configuration}
The Configuration phase, as implied by the name, exists to provide election
administrators an opportunity to configure the election contract: choices,
election start/end time, etc. After the configuration is complete the
administrators can freeze the contract, preventing it from being modified
further.

\subparagraph{Frozen}
Transitioning to the Frozen phase could simply set a boolean \sol{frozen} to
true and update the phase to Frozen. The boolean \sol{frozen} should be checked
at the start of every administrative function's execution and should cause the
immediate cease of code execution if the value is true. This serves to prevent
potentially malicious election administrators from modifying an election
contract after the voting process has started; it also serves to notify external
contracts and users that the election is configured and is ready or waiting to
move into the Vote phase.

\subparagraph{Vote}
The Vote phase is the phase during which various kinds of voting can occur for a
particular election contest. A \sol{vote} function should be defined. For
flexibility, a \sol{vote (uint8[], uint8[])} function signature is considered
which is capable of being leveraged by most kinds of electoral systems and
ballot structures. For example, as a sparse vector, where both arrays are of
equal length; values in the first array act as a choice index and values from
the second array act as a choice value. This could be used to marginally reduce
the size of transactions for many kinds electoral systems.

\subparagraph{Tally}
The Tally phase is the final phase of the election process that takes place
after the end of the Voting phase. The exact tallying process will depend on the
electoral system itself.


\paragraph{First-Past-the-Post}
The process for a first-past-the-post election is as follows:

\begin{enumerate}
  \item Configuration
  \begin{enumerate}
    \item Election administrators add each contest choice: \\
      \sol{addChoice(bytes32 _choice)}.
    \item Election administrators set the voting start time: \\
      \sol{setVoteStartTime(uint _voteStartTime)}.
    \item Election administrators set the voting end time: \\
      \sol{setVoteEndTime(uint _voteStartTime)}.
    \item Election administrators freeze the electoral contract, preventing
      further contract configuration mutations: \sol{freeze()}.
  \end{enumerate}

  \item Frozen
  \begin{enumerate}
    \item All functions are disabled until start time is met.
  \end{enumerate}

  \item Vote
  \begin{enumerate}
    \item Authorized voters cast votes by calling the \sol{vote(uint8 _choice)}
      function, where the value provided is the choice the voter supports.
      Votes are immediately summed into the struct of the appropriate choice.
      This is feasible because the first-past-the-post electoral system is
      summable in linear time and with linear space consumption.
  \end{enumerate}

  \item Tally
  \begin{enumerate}
    \item When the voting end time is reached the election moves into the
      Tally phase. At this point only the tally function can be called.
      The tally function, in this case, will check the number of votes for
      each candidate and set the winner to the choice that has received
      the most support. The election administrator is expected to run this
      contract function but any account can execute this function without
      any consequence to the result of the contract.
  \end{enumerate}
\end{enumerate}

\paragraph{Range Voting}
The process for a range vote is very similar to the process for
first-past-the-post. It is as follows:

\begin{enumerate}
  \item Configuration
  \begin{enumerate}
    \item Election administrators add each contest choice: \\
      \sol{addChoice(bytes32 _choice)}.
    \item Election administrators set the voting start time: \\
      \sol{setVoteStartTime(uint _voteStartTime)}.
    \item Election administrators set the voting end time: \\
      \sol{setVoteEndTime(uint _voteStartTime)}.
    \item Election administrators set the max score (<=100) a choice can
      receive: \\
      \sol{setMaxScore(uint8 _maxScore)}.
    \item Election administrators freeze the electoral contract, preventing
      further contract configuration mutations: \sol{freeze()}.
  \end{enumerate}

  \item Frozen
  \begin{enumerate}
    \item All functions are disabled until start time is met.
  \end{enumerate}

  \item Vote
  \begin{enumerate}
    \item Authorized voters cast votes by calling the \sol{vote(uint8[] _choices, uint8[] _scores)}
      function. The parameters act as a sparse vector where the value
      provided in \sol{_choice} indicates the choice being scored
      and the corresponding value provided in \sol{_scores}
      indicates the score of the choice. Scores are immediately summed
      into the struct of of the appropriate choice and a
      \sol{voteCount} member is incremented. This is feasible
      because the range vote electoral system is summable in linear time
      and with linear space consumption.
  \end{enumerate}

  \item Tally
  \begin{enumerate}
    \item When the voting end time is reached the election moves into the
      Tally phase. At this point only the tally function can be called.
      The tally function for this electoral system will, for each choice,
      multiply the summed score by $10^p$ where $p$ is
      \sol{precision} then divide the result by
      \sol{voteCount} to find an average score for the choice. The
      winner is set to the choice with the highest average score. The
      score is multiplied by $10^p$ because the EVM does not have floating
      point functionality and a greater precision makes ties less likely.
      The election administrator is expected to run this contract function
      but any account can execute this function without any consequence to
      the result of the contract.
  \end{enumerate}
\end{enumerate}

