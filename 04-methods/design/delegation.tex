\subsection{Delegation Components}
The delegation components offer mechanisms for the electorate to vest votes to
delegates who may vote on their behalf. A delegation hierarchy lends itself to a
graph representation.\cite{delegative-democracy} Specifically a directed
acyclic graph (DAG) forest-like structure where pendant and isolated vertices
represent voters and all other vertices represent delegates. A sink vertex
represents a delegate who has not delegated their vote further. Directed edges
represent delegations. The total weight of a voter, their voting power, can be
measured by recursively calculating the total number of incoming edges for each
vertex. We represent edges in the graph as follows:

% \begin{displayquote}
%   In a large and complex organization, the generalist role --- merely knowing
%   who knows best about something --- is often as important a function as the
%   specialist role of being able to make good decisions in any particular area.
%   In traditional large democratic structures, where elected officials in key
%   positions manage a hierarchical bureaucracy of some kind, most of those
%   elected positions are effectively generalist positions by practical necessity.
%   Ordinary voters do not have the interest level or patience required for
%   electing appropriate people to a large number of specialist positions, and
%   most would not have the knowledge or connections required to make good
%   decisions about those positions anyway. Therefore, specialists can usually
%   only participate in a large organization by being appointed or hired into a
%   position in the un-elected bureaucracy, and for this reason people are always
%   effectively subservient to the generalists who choose them. For this reason
%   the entire field of ``politics'' has effectively become a profession of
%   generalists, which very little technical knowledge in any particular field is
%   required or expected (except perhaps law), but in which the most important
%   characteristic of a candidate is the ability to look good, speak well, and
%   make the right connections. But it seems unfortunate and problematic that all
%   specialists are effectively excluded from direct participation in the
%   democratic process except in role that are strictly subservient to the
%   generalists: at the very least this structural pathology certainly contributes
%   to the frequently observed tendency of politicians to ``legislate'' highly
%   technical issues without adequate consultation with the experts in the
%   appropriate technical field --- or with a selection of ``experts'' that is
%   highly skewed for political reasons --- often with disastrous results.
%   Delegative democracy with open specialized forums and re-delegation provides
%   an alternative structure that enables both generalists and specialists to
%   participate directly in the democratic process as first-class policy-makers.
%   Widely recognized and trusted generalized are given the legitimate ability to
%   affect the balance of power in specialized forums by directing delegated votes
%   to chosen specialists, while the specialists participating directly in those
%   forums also retain the ability to build their own independent power bases and
%   thus not be completely at the mercy of generalists. At the same time, since
%   generalists and specialists alike are direct participants in the democratic
%   process, they have the same fundamental rights and responsibilities and are
%   all held accountable to all of their constitutes (i.e., those who delegate
%   votes to them) in the same way.
% \end{displayquote}

\begin{solidity}[Delegation Structure]
struct Voter {
  uint40 weight; // 5 bytes
  address delegate; // 32 bytes
}
\end{solidity}

Each time a delegation occurs a depth first traversal should be performed to
ensure that no cycles are created by the delegation; doing so maintains the
acyclic invariant required of the graph to support voter delegation. A simple
graph coloring algorithm is considered, beginning with white vertices and
coloring vertices black as they are traversed; traversal starts from the vertex
representing the voter who is delegating their vote and ends at some sink
vertex. If no cycle is detected the algorithm should check to see if the voter
has already delegated their vote; if so, it should traverse down their current
delegation path, decreasing the weight of each vertex visited by the weight
of the voter or delegate delegating their vote. Finally, the algorithm should
record the new delegate address and perform a final traversal, following the
same path originally traversed while performing cycle detection, increasing the
weight of each vertex visited along the way by the weight of the voter or
delegate delegating their vote. More briefly:
\begin{enumerate}
  \item Confirm that there are no cycles within the new delegation chain.
  \item Decrease the weight of the delegates in the old delegation chain.
  \item Increase the weights of the delegates in the new delegation chain.
\end{enumerate}
A complete implementation of this algorithm is as follows:\footnotemark{}

\footnotetext{
  Here we assume that every voter has their voting weight initialized to 1. This
  algorithm might be improved by combining the cycle detection and weight
  accumulation steps while leveraging the \sol{revert()} function to revert the
  state of the contract if a cycle is detected. We opt not to do that for now
  because return values are not currently available with the \sol{revert()}
  functionality.
}

\begin{solidity}[Vote Delegation]
mapping (address => Voter) voters;

function delegateVote (address delegate) public auth returns (bool _success) {
  // The `auth` modifier prevents this function from being
  // called until the Authority has confirmed that the
  // the message sender has the proper privileges.
  Voter cursor;
  uint40 weight = voters[msg.sender].weight;

  mapping (address => bool) visited;
  visited[msg.sender] = true;
  visited[delegate] = true;

  // Cycle Detection
  cursor = voters[delegate];
  while (cursor.delegate) {
    address newDelegate = cursor.delegate;
    if (visited[newDelegate]) return false;
    cursor = voters[newDelegate];
    visited[newDelegate] = true;
  }

  // Decrement weights of old delegate chain.
  cursor = voters[msg.sender];
  while (cursor.delegate) {
    address newDelegate = cursor.delegate;
    cursor = voters[newDelegate];
    cursor.weight -= weight;
  }

  // Increment weights of new delegate chain.
  cursor = voters[msg.sender];
  cursor.delegate = delegate;
  while (cursor.delegate) {
    address newDelegate = cursor.delegate;
    cursor = voters[newDelegate];
    cursor.weight += weight;
  }

  return true;
}
\end{solidity}
