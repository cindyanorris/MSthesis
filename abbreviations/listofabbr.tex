\abbreviations{}

% You can put here what you like, but here's an example
%Note the use of starred section commands here to produce proper division
%headers without bad '0.1' numbers or entries into the Table of Contents.
%Using the {\verb \begin{symbollist} } environment ensures that entries are
%properly spaced.

% \section*{Symbols}

% Put general notes about symbol usage in text here.  Notice this text is
% double-spaced, as required.

% \begin{symbollist}
% 	\item[$\Xi$/\DH/$\blacklozenge$] \solt{ether}, also referred to as \solt{eth} or ETH
% 	% \item[$\mathbb{X}$] A blackboard bold $X$.  Neat.
% 	% % Optional item argument makes the symbol/abbr
% 	% \item[$\mathcal{X}$] A caligraphic $X$.  Neat.
% 	% \item[$\mathfrak{X}$] A fraktur $X$.  Neat.
% 	% \item[$\mathbf{X}$] A boldface $X$.
% 	% \item[$\mathsf{X}$] A sans-serif $X$. Bad notation.
% 	% \item[$\mathrm{X}$] A roman $X$.
% \end{symbollist}

% \section*{Abbreviations}

% Long lines in the \texttt{symbollist} environment are single spaced, like in
% the other front matter tables.
%
% \begin{symbollist}
% 	\item[AR] Aqua Regia, also known as hydrocloric acid plus a splash of
% 	nitric acid.
% 	\item[SHORT] Notice the change in alignment caused by the label width
% 	between this list and the one above.  Also notice that this multiline
% 	description is properly spaced.
% 	\item[OMFGTXTMSG4ME] Abbreviations/Symbols in the item are limited to
% 	about a quarter of the textwidth, so don't pack too much in there.
% 	You'll bust the margins and it looks really bad.
% \end{symbollist}

\section*{Blockchain Technologies}
\begin{symbollist}
	\item[ASIC] Application-Specific Integrated Circuit
\end{symbollist}

\subsection*{Ethereum}
\begin{symbollist}
	\item[$\Xi$/\DH/$\blacklozenge$] \solt{ether}, also referred to as \solt{eth} or ETH
	\item[EVG] Ethereum Virtual Machine
	\item[OOG] Out-of-Gas Exception
\end{symbollist}

% Legislation
\section*{Legislation}
\begin{symbollist}
	\item[VRA] Voting Rights Act (1965)
	\item[UOCAVA] Uniformed and Overseas Citizens Absentee Voting Act (1986)
	\item[NVRA] National Voter Registration Act, also known as the Motor Voter Act (1993)
	\item[HAVA] Help America Vote Act (2002)
	\item[MOVE] Military and Overseas Voter Empowerment Act, amended UOCAVA (2009)
	\item[VEA] Voter Empowerment Act (2015)
\end{symbollist}

\subsection*{Legislative Organizations}
\begin{symbollist}
	\item[DoD] Department of Defense
	\item[EAC] Election Assistance Commission, established under HAVA (2002)
	\item[FVAP] Federal Voting Assistance Program, established under UOCAVA (1986)
	\item[NIST] National Institute of Standards and Technology
	\item[SPRG] Security Peer Review Group, technical group assembled by FVAP to review SERVE
	\item[(DC)BOEE] Washington D.C. Board of Elections and Ethics
	\item[OSDV] Open Source Digital Voting Foundation
\end{symbollist}

% Voting
\section*{Voting}
\begin{symbollist}
	\item[LEO] Local Election Official
	\item[VVPAT] Verifiable Voter Paper Audit Trail
	\item[VVSG] Voluntary Voting System Guidelines
	\item[DRE] \emph{Direct Recording Electronic} voting machines
\end{symbollist}

\subsection*{Voting Systems}
\begin{symbollist}
	\item[IVAS] Integrated Voting Alternative Site
	\item[VOI] Voting Over the Internet project (2000)
	\item[SERVE] Secure Electronic Registration and Voting Experiment (2004)
	\item[DVBM] D.C. Digital Vote-by-Mail system
\end{symbollist}

% Elections
\section*{Elections}
\begin{symbollist}
	\item[MMD] \emph{Multi-Member District}, a district with multiple seats
		available for candidates to hold or multiple choices available for
		selection.
	\item[SMD] \emph{Single-Member District}, also known as single-winner district
	\item[MMP] Mixed-Member Proportional
	\item[PR] Proportional Representation
\end{symbollist}

\subsection*{Electoral Criteria}
\begin{symbollist}
	\item[MC] \emph{Majority Criterion}
	\item[MMC] \emph{Mutual Majority Criterion}
	\item[ISDA] \emph{Independence of Smith-dominated Alternatives}
	\item[IIA] Independence of Irrelevant Alternatives (IIA)
	\item[LIIA] Local Independence of Irrelevant Alternatives (LIIA)
	\item[CC] Consistency Criterion
	\item[PC] Participation Criterion (PC)
	\item[PC] No Favorite Betrayal (NFB)
\end{symbollist}

\subsection*{Electoral Systems}
\begin{symbollist}
	\item[LPR] List Proportional Representation
	\item[STV] Single Transferable Vote
	\item[FPTP] First Past the Post
	\item[BV] Block Vote
	\item[PBV] Party Block Vote
	\item[AV/IRV/RCV] \emph{Alternative Vote}, also known as \emph{Instant-Runoff
		Voting}, \emph{Ranked Choice Voting}, transferable vote, or preferential
		vote
	\item[RV/SV] \emph{Range Vote}, also known as \emph{Score Voting}
	\item[TRS] Two-Round System
\end{symbollist}
