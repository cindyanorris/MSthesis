% \abstract{}

\begin{abstract}
Proponents of Internet-based voting systems suggest that it might offer
solutions to alleviate some of the shortcomings previous democratic institutions
have fallen victim to: gerrymandering, election fraud, malicious ballot-box
zoning, etc. This has become especially relevant as governments and societies
wrestle with the COVID-19 pandemic, state-sponsored election interference, and
claims of election fraud. However, Internet voting systems are fraught with
risks and there is a wealth of research and real-world incidents which
demonstrate the plethora of issues one would face in implementing such a system.
Voting systems are well-understood to have notoriously difficult to fulfill
requirements, which are often at odds with one another, including security,
privacy, and verifiability requirements. The rise in popularity of
blockchain-based technologies has renewed interest in such systems, and although
it is unlikely that publicly available blockchain-based solutions can meet the
requirements demanded of large-scale elections, there is potential utility in
having on-chain electoral systems available for lower-stakes decision-making.
This research investigates blockchain-based decision-making through electoral
system design and implementation using the Ethereum blockchain and Solidity
programming language. This research demonstrates that secure and verifiable
small-scale voting systems can be built using ``off-the-shelf'' blockchain
technologies when privacy constraints are loosened. This research further
demonstrates that the costs associated with operating such voting systems are
steep, likely rendering such systems impractical in most circumstances and
therefore an inappropriate foundation for large-scale elections. Finally, this
research identifies which electoral systems and features are viable within these
blockchain environments and the associated costs of supporting and operating
these electoral systems and features.
\end{abstract}
