\chapter{Discussion}\label{chap:discussion}

\section{Contracts}
\subsection{Delegation Contract \& Registration Authority}
The delegation contract, and to some degree the registration authority, suffer
from issues with respect to state management, multi-contract usage, and
concurrency generally. It would be ideal (for conceptual, storage, and
computational reasons) for multiple contracts to leverage the same registration
authority and delegation contracts. Unfortunately issues begin to arise as
electoral contracts in different phases begin querying the registration
authorities and delegation contracts. This is further complicated as malicious
actors attempt to subvert the electoral system.

To begin to understand the issue we can imagine a single FPTP election contract
\lstinline|A| which is leveraging a delegation contract \lstinline|D|. The FPTP
contract \lstinline|A| expects delegation changes to continue until it enters
its \lstinline|Tally| phase, at which point it should expect the delegation
contract to enter a read-only or write-blocking mode. This might occur through
some mutex mechanism. This locking mechanism is necessary because the FPTP
election contract must accurately determine the weight of each vote for each
delegate. If the delegation contract is not locked a delegate might transfer his
voting power to another delegate immediately after the FPTP election contract
\lstinline|A| determines his weight from delegation contract \lstinline|D| and
sums the delegate's votes to the total. The following example demonstrates this
issue:

\begin{enumerate}
    \item Voting Begins
    \begin{enumerate}[i.]
        \item Delegate X with a voting weight of 5 casts a ballot for choice A.
        \item Delegate Y with a voting weight of 1 casts a ballot for choice A.
        \item Delegate Z with a voting weight of 10 casts a ballot for choice B.
    \end{enumerate}
    \item Voting Ends
    \item Tallying Begins
    \begin{enumerate}[i.]
        \item The election contract sums delegate X's $vote*weight$ to choice A.
            Choice A now has 5 votes in its favor.
        \item Delegate X confers his votes to delegate Y. Y now has a weight of
            6 votes.
        \item The election contract sums delegate Y's $vote*weight$ to choice A.
            Choice A now has 11 votes in its favor.
        \item The election contract sums delegate Z's $vote*weight$ to choice B.
            Choice B now has 10 votes in its favor.
    \end{enumerate}
    \item Tallying Ends
    \begin{enumerate}[i.]
        \item Choice A wins by 1 vote (11 to 10) despite only having 6 votes
            cast in its favor.
    \end{enumerate}
\end{enumerate}

It is clear that \lstinline|B| should have won this contest 10 to 6. This
failure to properly tally the results is a product of the classic
readers-writers concurrency problem and represents a clear danger to any
electoral system. A common solution to the readers-writers problem is to
implement a simple mutex mechanism. For example, we could in the delegation
contract allow readers to lock writing to the delegation contract until the
readers have finished reading from the contract.  This is known as the
readers-preference solution. A simple improvement beyond that might be to cache
delegations in a separate data structure as a sort of journaling system while
the delegation contract is locked. These delegations could be combined and
applied as a single commit to the contract state as soon as the contract is
unlocked, this is the writers-preference solution. In this way the electorate
could still confer their votes freely to their delegates and have those
delegations reflected in the next election or tally.

This works well under some circumstances; e.g., when a small set of electoral
contracts interacting with a delegation contract non-maliciously; however, if
ever a malicious or broken electoral contract failed to release the lock, then
the writers would starve, essentially preventing the electorate from ever
conferring their votes to a delegate again.

We can imagine solving this starvation issue by giving the delegation contract
the ability to take back a mutex after a specified period of time, however, if
the electoral contract was not malfunctioning or malicious and simply performing
a long-running tally of votes then it's very likely that the voting actors would
have either maliciously or accidentally corrupted the tally results once the
delegation contract unlocked itself to commit its journaled delegations.

Further, even if we imagined a perfectly functional electoral contract with a
short tally process, issues still manifest themselves as multiple instances of
that electoral contract begin to request the delegation contract be locked.
There's a chance that the writers become starved if in a reader-preference
solution. This is not so serious a concern in a writer-preference solution where
mutexes expire, but a much larger concern then appears, which is that a
writers-preference solution would produce an ever-expanding queue of electoral
contracts depending on the delegation contract. This occurs because readers
would be forced to operate serially to prevent their election contract state
from becoming corrupted by trying to begin a read/tally process in the middle or
late stages of another election contract's read/tally process. This is clearly
suboptimal.

The solution to this problem would likely involve the electoral contract being
able to specify the root of the delegation contract's Merkle DAG state; this
would functionally act as a fingerprint of the state of the delegation contract
which the election contract wishes to operate against. This concept is not too
dissimilar from git commits. Unfortunately the EVM does not currently offer such
functionality.  There are several mechanisms which, if offered by Ethereum's
EVM, would support such functionality. One mechanism to support this
functionality would be via explicit contract state checkpoints, however, this
would likely be a prohibitively expensive operation for storage reasons. Another
mechanism would be to support invertible operation definitions, thus allowing a
node to rewind the state of a contract. Such functionality would only be
computationally expensive, but potentially prohibitively so if there had been
many state modifications since the commit a contract wishes to operate against.
A combination of these two mechanisms might offer the best solution.

\section{Performance}


\section{Voter Registration List Population}



\subsection{Design Principles}
We borrow design principles from ``Electoral System Design: The New
International IDEA Handbook'' which were outlined in Chapters 2 and 3.

\subparagraph{Representation} We wish to achieve fair representation. What
constitutes fair representation will largely depend on the greater democratic
framework and the constitutes of that framework. Our electoral system should
translate votes into winning choices in a way that accurately and fairly
represents the will of the people while also being flexible enough to allow for
configuration and modification appropriate for various governance structures.

\subparagraph{Transparency} Our electoral system should be as transparent as
possible, preferably end-to-end verifiable. The winners, losers, and electorate
must be able to trust that the results of an election were achieved
legitimately.

\subparagraph{Inclusiveness} Our electoral system should support full suffrage
(active and passive) as well as universal suffrage. The mechanisms of the
electoral system should not be biased such that any group is discriminated
against. Designing an electoral system with inclusiveness in mind results in
governance with a stronger sense of legitimacy and wider participation and
willingness to participate by the electorate.

\paragraph{International Standards} There is no universally agreed upon
standards, but most would agree upon the following standards.

\begin{enumerate}[label=\Large$\square$]
  \item Elections should be free, fair, and periodic.
  \item Universal adult suffrage should be supported.
  \item Ballot secrecy should be preserved and constituents should be free
    from coercion.
  \item A commitment to the principle of ``one person, one vote.''
\end{enumerate}

\paragraph{Design Checklist}
We also borrow a design checklist from the ``Electoral System Design: The New
International IDEA Handbook.''
\todo{This should be moved to the conclusion.}

\begin{enumerate}[label=\Large$\square$]
  \item Is the system clear and comprehensible?
  \item Has context been taken into account?
  \item Is the system appropriate for the time?
  \item Are the mechanisms for future reform clear?
  \item Does the system avoid underestimating the electorate?
  \item Is the system as inclusive as possible?
  \item Was the design process perceived to be legitimate?
  \item Will the election results be seen as legitimate?
  \item Are unusual contingencies taken into account?
  \item Is the system financially and administratively sustainable?
  \item Will the voters feel motivated to participate?
  \item Is a competitive party system encouraged?
  \item Does the system fit into a holistic constitutional framework?
  \item Will the system help alleviate conflict rather than exacerbate it?
\end{enumerate}
