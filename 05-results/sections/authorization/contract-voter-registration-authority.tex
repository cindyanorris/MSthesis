\subsubsection{Contract VoterRegistrationAuthority}

The \solt{contract}, \sol{contract VoterRegistrationAuthority}, is the final
component of our authorization design and is required to construct a generalized
voter registration authority capable of managing registered voters and
conducting elections.

% signatureMap['vote'] = bytes4(sha3('vote(uint8[],uint8[])'));
\begin{solidity}[contract VoterRegistrationAuthority]
contract VoterRegistrationAuthority is Owned, Registry, Authority {
  enum Permissions {
    Vote
  }

  AccessControlList acl;
  mapping (bytes32 => bytes4) resourceSignatures;

  constructor () public {
    acl = new ACLAuthority(true);

    resourceSignatures['vote'] = bytes4(sha3('vote()'));

    acl.setRequiredResourcePermission(
      resourceSignature('vote'),
      uint8(Permissions.Vote)
    );
  }

  function register (address _voter) public restrictTo(owner) returns (bool _success) {
    return acl.grantPermission(_voter, resourceSignatures['vote'], uint8(Permissions.Vote));
  }

  function unregister (address _voter) public restrictTo(owner) returns (bool _success) {
    return acl.revokePermission(_voter, resourceSignatures['vote'], uint8(Permissions.Vote));
  }

  function isRegistered (address _voter) public constant returns (bool _isRegistered) {
    return acl.hasPermission(_voter, resourceSignatures['vote'], uint8(Permissions.Vote));
  }

  function canCall (address _subject, bytes4 _resource) public constant returns (bool _canCall) {
    return acl.canCall(_subject, _resource);
  }
}
\end{solidity}

\todo{Finish documenting the VoterRegistrationAuthority implementation.}

\begin{state}
  \item \attributes

  \begin{public}
    \item \sol{AccessControlList acl} maintains the \solt{address} of an ACL
      implementation, a \solt{contract} which implements \sol{interface
      AccessControlList}.

    \item \sol{mapping (bytes32 => bytes4) resourceSignatures} maintains a
      mapping of strings, \sol{bytes32}, to \solt{function} signature,
      \sol{bytes4}.
  \end{public}
\end{state}

\begin{code}
  \item \operations

  \begin{constructor}
    \item \sol{constructor (bool _createACL)}, upon creation and initialization
      of this \solt{contract} the \solt{constructor} can deploy a
      \solt{contract}, \sol{contract BasicACL}, if the \solt{bool}, \sol{bool
      _createACL}, is set to \solt{true}.

      \begin{parameters}
      \item \sol{bool _createACL}, set to \solt{true} to deploy an ACL
        implementation, \sol{contract BasicACL}, in addition to this
        \solt{contract}; otherwise \solt{false}.
      \end{parameters}
  \end{constructor}

  \begin{functions}
    \item \sol{function setAcl (AccessControlList _acl)}, updates the
      \solt{contract} reference which is responsible for managing ACL requests.

      \begin{modifiers}
        \item \sol{modifier restrictTo (owner)}, restricts access to the
          \solt{function} such that \emph{only} the current \solt{owner} of the
          \sol{contract} can call it.
      \end{modifiers}

      \begin{parameters}
        \item \sol{AccessControlList _acl}, the ACL implementation which access
          control responsibilities are to be delegated to.
      \end{parameters}

      \begin{returns}
        \item \sol{bool _success}, resolves to \solt{true} if the operation was
          successful, otherwise \solt{false}.
      \end{returns}


    \item \sol{function setRequiredResourcePermission (bytes4 _resource, uint8 _permission)},
      updates the mapping representing the \emph{permission} required to access
      some \emph{resource}, \solt{function} signature.

      \begin{modifiers}
        \item \sol{modifier restrictTo (owner)}, restricts access to the
          \solt{function} such that \emph{only} the current \solt{owner} of the
          \sol{contract} can call it.
      \end{modifiers}

      \begin{parameters}
        \item \sol{bytes4 _resource}, the \emph{resource}, \solt{function}
          signature, which the \emph{permission} is to be modified for.

        \item \sol{uint8 _permission}, a \solt{uint8} value representing the
          \emph{permission-bit} required for the \emph{resource}.
      \end{parameters}

      \begin{returns}
        \item \sol{bool _success}, resolves to \solt{true} if the operation was
          successful, otherwise \solt{false}.
      \end{returns}


    \item \sol{function canCall (address _subject, bytes4 _resource)}, evaluates
      whether a \emph{subject} can access to some \emph{resource} by validating
      with the ACL implementation that the \emph{subject} has the required
      \emph{resource} \emph{permission}.

      \begin{parameters}
        \item \sol{address _subject}, the \emph{subject}, Ethereum account,
          whose \emph{permissions} are being evaluated.

        \item \sol{bytes4 _resource}, the \emph{resource}, \solt{contract
          function}, which the \emph{subject's} \emph{permissions} are being
          evaluated against.
      \end{parameters}

      \begin{returns}
        \item \sol{bool _canCall}, resolves to \solt{true} if the
          \emph{subject}, \sol{address _subject}, is permitted to access the
          \emph{resource}, \sol{bytes4 _resource}, otherwise \solt{false}.
      \end{returns}


    \item
      \sol{function hasPermission (address _subject, bytes4 _resource, uint8 _permission)}, \\
      \sol{function getPermissions (address _subject, bytes4 _resource)}, \\
      \sol{function setPermissions (address _subject, bytes4 _resource, uint256 _permissions)}, \\
      \sol{function grantPermission (address _subject, bytes4 _resource, uint8 _permission)}, \\
      \sol{function revokePermission (address _subject, bytes4 _resource, uint8 _permission)}, \\
      all ACL responsibilities originating from \sol{interface AccessControlList}
      are delegated to the provided ACL implementation stored in
      \sol{AccessControlList acl}.

      \begin{modifiers}
        \item \sol{modifier restrictTo (owner)}, restricts access to the
          \solt{function} such that \emph{only} the current \solt{owner} of the
          \solt{contract} can call this \solt{function}; this applies to
          \solt{function}[s] \sol{setPermissions}, \sol{grantPermission}, and
        \sol{revokePermission}.
      \end{modifiers}
  \end{functions}
\end{code}
