% \section{Novel Electoral Systems, Voting Experiments, and Technologies}
\section{Notable Voting Systems \& Experiments}
What follows is a review of other notable voting systems and experiments:

\subsection{Novel Voting Systems}

\subsubsection{Estonia}
Estonia began using Internet voting in 2005. By the 2015 Estonian parliamentary
elections 30.5\% of all voters voted over the Internet. Estonia maintains what
are probably the most advanced national identification cards in the world.
Estonian IDs are part of a \emph{Public Key Infrastructure (PKI)} where IDs
serve as smart cards which possess two RSA key pairs: one for signing and one
for authentication. Cryptographic functions are performed on the card. The
signatures produced by the IDs are used extensively throughout the country and
are legally binding. These cryptographic IDs allow Estonia to provide voter
authentication capabilities that cannot be reproduced in the US.\ Despite the
advanced authentication capabilities that Estonia offers researchers in 2014
devised a number of attacks that could be performed on the Estonian voting
system to spoil ballots, damage ballot secrecy, and steal/drop votes. The
researchers also criticized the transparency and operational security of the
system.

\subsection{Blockchain Voting Systems}
\todo{Compare and contrast other electoral systems with this research, discuss
weaknesses, holes in the research, etc.}

% \subsection{Voting Machines}
% \todo{Complete section.}
%
% % #### Diebold AccuVote-TS
% \subsubsection{Diebold AccuVote-TS}
% \todo{Complete section.}
%
% % #### AVS WINVote
% \subsubsection{AVS WINVote}
% \todo{Complete section.}

% \subsection{Voting Experiments}

% ### Demonstrated Attacks
% The following is a review of Internet voting systems that research has shown to
% be susceptible to attack.


% \subsection{Voting Technologies}

% ### Demonstrated Attacks
% The following is a review of electronic voting machines that research has
% shown to be susceptible to attack.
%
% \subsection{Compare and Contrast}
