\subsection{Design Principles}

\todo{
  This section was taken from the Methods, it should be covered in the
  Literature Review, referenced from the Methods, addressed again in the
  Discussion, and briefly mentioned in the Conclusion.
}

We borrow design principles from ``Electoral System Design: The New
International IDEA Handbook'' which were outlined in Chapters 2 and 3.

\subparagraph{Representation} We wish to achieve fair representation. What
constitutes fair representation will largely depend on the greater democratic
framework and the constitutes of that framework. Our electoral system should
translate votes into winning choices in a way that accurately and fairly
represents the will of the people while also being flexible enough to allow for
configuration and modification appropriate for various governance structures.

\subparagraph{Transparency} Our electoral system should be as transparent as
possible, preferably end-to-end verifiable. The winners, losers, and electorate
must be able to trust that the results of an election were achieved
legitimately.

\subparagraph{Inclusiveness} Our electoral system should support full suffrage
(active and passive) as well as universal suffrage. The mechanisms of the
electoral system should not be biased such that any group is discriminated
against. Designing an electoral system with inclusiveness in mind results in
governance with a stronger sense of legitimacy and wider participation and
willingness to participate by the electorate.

\paragraph{International Standards} There is no universally agreed upon
standards, but most would agree upon the following standards.

\begin{enumerate}[label=\Large$\square$]
  \item Elections should be free, fair, and periodic.
  \item Universal adult suffrage should be supported.
  \item Ballot secrecy should be preserved and constituents should be free
    from coercion.
  \item A commitment to the principle of ``one person, one vote.''
\end{enumerate}

\paragraph{Design Checklist}
We also borrow a design checklist from the ``Electoral System Design: The New
International IDEA Handbook.''

\begin{enumerate}[label=\Large$\square$]
  \item Is the system clear and comprehensible?
  \item Has context been taken into account?
  \item Is the system appropriate for the time?
  \item Are the mechanisms for future reform clear?
  \item Does the system avoid underestimating the electorate?
  \item Is the system as inclusive as possible?
  \item Was the design process perceived to be legitimate?
  \item Will the election results be seen as legitimate?
  \item Are unusual contingencies taken into account?
  \item Is the system financially and administratively sustainable?
  \item Will the voters feel motivated to participate?
  \item Is a competitive party system encouraged?
  \item Does the system fit into a holistic constitutional framework?
  \item Will the system help alleviate conflict rather than exacerbate it?
\end{enumerate}
