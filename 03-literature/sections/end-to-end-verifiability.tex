\section{End-to-End Verifiability}\label{sec:e2e-viv}
The vulnerabilities and weaknesses of electronic voting systems demonstrate a
clear need for a stronger kind of system design. \emph{End-to-End Verifiable
(E2E-V)} voting systems are systems which aim to provide such a design by
providing the following features for voters:

\begin{enumerate}
  \item allows voters to check that the system recorded their votes correctly

  \item allows voters to check that the system included their votes in the final
    tally

  \item allows voters to count the recorded votes and double-check the announced
    outcome of the election.
\end{enumerate}

% E2E-V voting systems have been a hotbed of research over the past several years.
An \emph{End-to-End Verifiable Internet Voting (E2E-VIV)} is an E2E-V voting
system that supports voting over the Internet.

% \begin{displayquote}
%    Aggressive early adoption of election technology must be tempered by a clear
%    understanding that voters’ trust in their elections is hard-won and easily
%    lost.
% \end{displayquote}


\subsection{Requirements}
In 2015, the U.S. Vote Foundation published a specification and feasibility
assessment study which laid out requirements for building an end-to-end
verifiable internet voting system. The study broke these requirements into two
categories:\cite{e2e-viv}

\begin{itemize}
  \item \emph{Technical Requirements}, the set of requirements which can be
    directly addressed by system design and architecture

  \item \emph{Non-Functional Requirements}, the set of requirements which must
    be imposed on a system by external entities
\end{itemize}

% cite: The Future of Voting: E2E-VIV
% This section reviews those requirements.

\subsubsection{Technical Requirements}
The ten categories of technical requirements, those which should be addressed
by the design and architecture of the systems itself, are:\cite{e2e-viv}

\begin{enumerate}
    \item functional requirements
    \item accessibility requirements
    \item usability requirements
    \item security requirements
    \item authentication requirements
    \item auditing requirements
    \item system operational requirements
    \item reliability requirements
    \item interoperability requirements
    \item certification requirements
\end{enumerate}

% \paragraph{Functional}
% \begin{displayquote}
%   The functional requirements of an E2E-VIV system deal primarily with the
%   casting and recording of ballots and associated voter records.
% \end{displayquote}

The \emph{functional requirements} deal primarily with casting and recording of
ballots. \emph{Receipt freedom} is one such functional requirement. An
electoral system which expresses receipt freedom is said to make it impossible
for a voter to prove to anyone how they voted.
%
% \subparagraph{Receipt freedom} is one such functional requirement, a property
% where it is impossible for a voter to prove to anybody how they voted.
%
Others functional requirements include:
\begin{itemize}
    \item ensuring that a voter cast a ballot if such an act is recorded
    \item data retention in case of failure
    \item multi-vote functionality to overwrite previous votes
    \item maintaining voter anonymity
\end{itemize}

% \paragraph{Usability}
% \begin{displayquote}
%   The usability of an E2E-VIV system is critical to its successful adoption and
%   use.
% \end{displayquote}

\emph{Usability} is mostly concerned with user experience and confirmation
guarantees. For example, voters should be confident that their vote was cast by
being provided a confirmation screen. The voting process should be both
intuitive and guide the voter through the process. Presentations such as the
butterfly ballot should be avoided at all costs.

% \paragraph{Accessibility}
% \blockquote{\emph{Accessibility} is the property of being usable by and
% useful to voters with disabilities.}

Digital voting systems have the potential to provide wider \emph{accessibility}
guarantees than traditional paper ballots for voters with disabilities. To
provide these guarantees developers must involve voters throughout the
development process to identify accessibility issues and implement solutions.

% \paragraph{Security}
\emph{Security} is an integral property and requirement which voting systems
must maintain. Included in this requirement is that:

\begin{itemize}
    \item no data can be permanently lost
    \item integrity of voters, candidates, ballot information, cast ballots, and
      other critical information must be maintained
    \item accurate timing information is critical for auditing
    \item voting equipment must be protected
    \item the system must perform regular health checks
\end{itemize}

% \paragraph{Authentication}
\emph{Authentication} is the process of ascertaining the validity of a claimed
identity. Authentication ensures that the voting system can enforce privacy and
prevent multi-voting, Sybil attacks, and vote theft. All individuals must be
identified uniquely. The system must allow access to services only to authorized
users, e.g., only allow election officials to load ballot info.

% \paragraph{Auditing}
The property of \emph{auditability} means that a voting system is capable of
comprehensive examination. Auditability must exist at all stages and levels of
the voting process. The system must keep auditable logs of all relevant activity
and the logs must be public and write only. Furthermore, the logs cannot leak
any data regarding voters or the way any ballot was cast. Privacy must always
be the foremost concern.

The auditing system must actively report issues and information in
real-time. At least the following events should be recorded:\cite{e2e-viv}
\begin{itemize}
  \item ``all voting-related information, including the number of eligible
    voters and votes cast, the number of invalid votes, count and recount
    results, etc.''
  \item ``any detected attacks on the operation of the system or its
    communication infrastructure''
  \item ``any system failures, malfunctions, or other detected threats to proper
    system operation.''
\end{itemize}
%
% \blockquote[{\cite[67]{}}]{%
%     \begin{itemize}
%         \item all voting-related information, including the number of eligible voters and
%             votes cast, the number of invalid votes, count and recount results, etc.;
%         \item any detected attacks on the operation of the system or its communication
%             infrastructure; and
%         \item any system failures, malfunctions, or other detected threats to proper
%             system operation.
%     \end{itemize}
% }
%
The system should provide auditing features which support the ability to:\cite{e2e-viv}
\begin{itemize}
  \item ``cross-check and verify the correct operation of the voting system and
    the accuracy of the election results''
  \item ``detect voter fraud''
  \item ``prove that all counted votes are legitimate and that all ballots have
    been counted''
\end{itemize}
%
% \blockquote[{\cite[67]{}}]{%
%     \begin{itemize}
%         \item cross-check and verify the correct operation of the voting system and the
%             accuracy of the election results;
%         \item detect voter fraud; and
%         \item prove that all counted votes are legitimate and that all ballots have been
%             counted.
%     \end{itemize}
% }
%
Finally, auditability must extend to the source code, actions performed, and the
documentation itself.

% \paragraph{System Operational}
\emph{System operational} requirements are those that enforce and regulate
transparency, accountability, system configuration, and updates. Logs, software,
configurations, versions, updates, etc., must all be managed and produced to
audit for tampering. Protocols should be in place to guard sensitive equipment
at all times and handle system failures. Officials managing these systems and
the procedures themselves must be scrutinized closely to prevent insider attacks
and election fraud.

% \paragraph{Reliability}
\emph{Reliability} is the property of a system behaving reasonably and as
expected under both normal conditions and while under attack. During an election
period a system should be highly available. 99.9\% availability is a minimum for
voting systems. The system must also be able to recover from any failure within
10 minutes, with the exception for failure caused by natural disaster or
malicious attack. The system should have redundant backup systems for critical
components of the system.

Internet voting systems are compelling targets for Distributed Denial of Service
(DDoS) attacks, therefore it is important that an E2E-VIV system be hardened to
such attacks and be able to continue operation with full correctness during a
sustained DDoS attack.

% \paragraph{Interoperability}
An E2E-VIV system must use open standards for \emph{interoperability} between
components, services, and other E2E-VIV systems. Logs and documentation of such
standards must be published so that anyone can download, inspect, and publish
analysis and concerns.

% \paragraph{Certification}
Finally, there should be \emph{certification} and test procedures involved for
every functional requirement; these tests should be able to be run on demand.
Formal proofs of security and correctness should be provided wherever possible,
and third-parties should be hired to conduct an independent review, audit, and
test of the system.

\subsubsection{Non-functional Requirements}
The five non-functional requirements defined, those which must be fulfilled by
external entities, e.g., operators and administrators, are:\cite{e2e-viv}

\begin{enumerate}
    \item operational requirements
    \item procedural requirements
    \item legal requirements
    \item assurance requirements
    \item maintenance/evolvability requirements
\end{enumerate}

% \paragraph{Operational}
The specification describes several \emph{operational} requirements: election
and registration timing, maintaining voter registration and candidate
nomination lists, providing receipt freedom, voter assistance, election
integrity, and openness.

% \subparagraph{Voter Assistance}
Voters must be well-informed on how to register, vote, and protect their privacy
in the voting system.
% \subparagraph{Election and Registration Timing}
Clear instruction on when voting and registration occurs should be announced far
in advance for the voter's benefit. When multiple forms of remote voting take
place, votes cast over the Internet should not be accepted after other forms of
remote voting end.
% \subparagraph{Voter Registration}
E2E-VIV systems must publish a voter register that is regularly updated. Voters
should be able to check that information in the register is accurate and request
corrections.
% \subparagraph{Candidate Nomination Lists}
The ballot presented to voters must be consistent, fair, unbiased, and free from
any superfluous information about candidates/choices.

% \subparagraph{Receipt Freedom}
Operational receipt freedom represents two different requirements depending on
whether a voter is voting from a supervised or unsupervised location. In a
supervised location receipt freedom requires that the voting terminal clear all
indication of how a ballot was cast and ensure that no paper trail representing
how the ballot was cast is able to leave the polling place (except by official
means). In an unsupervised location any visual proof of vote should not be able
to be used to determine how a vote was cast or will be tallied.

% \subparagraph{Election Integrity}
If test ballots are capable of being submitted then those ballots must be
clearly marked as a test ballot with instruction on how to cast a real ballot.
The voting system should not disclose any results to any person until after the
voting period has ended, including alternative forms of voting. Tallying should
be done as soon as possible afterwards and the tallying process should be
transparent, recorded, and be able to be replayed. Any irregularities which
affect the integrity of votes should be recorded.

% \subparagraph{Openness}
An E2E-VIV system must be open and function properly regardless of the hardware
or software being used to run the voting software. The system must be available
for auditing by external actors, especially when considering components which
are expected to be run on external systems or voter's machines.

% \paragraph{Procedural}
\emph{Procedural} requirements define the processes required to deploy and run
the E2E-VIV system.

\begin{itemize}
  \item Procedures should be published regarding provisioning, certification,
    maintenance, availability, and use. For example, when updates occur,
    election officials must call upon an independent body to perform
    verifications of performance and certification of intent.

  \item Procedures should be in place to teach voters the voting process.

  \item Election officials should have maintenance and security procedures to
    ensure that voting equipment is operating nominally and has not been
    tampered with. For example, conducting sensitive operations should require
    teams of at least two people.

  \item As much as possible there should be procedures in place to allow
    observers to watch election procedures.

  \item Procedures should be in place to update results in the event that a
    voter proves that their vote was not accurately received or counted.
\end{itemize}

% \paragraph{Legal}
The \emph{legal} requirements include national, state, and local laws that apply
to voting systems, e.g., accessibility, anonymity, and availability guarantees.
Any deployed E2E-VIV system must comply with these laws. For example, election
officials must ensure that only one ballot by each voter is tallied when
multiple means of voting exist, e.g., remote and traditional polling place.
%
% Voters must be able to restart, discard, or alter their votes at any point
% during the voting process. The system must allow the voter to participate in an
% election without marking choices, i.e., casting blank or partially blank
% ballots. The voting system must always preserve anonymity and indicate clearly
% that the voters ballot has been cast and voting procedure completed.
%
\begin{displayquote}
  ``There must be no impediments to interested parties who want to study the
  E2E-VIV system. In particular, no nondisclosure agreement or contract of any
  kind may be required for download and study of, or for building, testing and
  publishing test results for, the E2E-VIV system.''\cite{e2e-viv}
\end{displayquote}
% \blockquote{There must be no impediments to interested parties who want to study
% the E2E-VIV system. In particular, no nondisclosure agreement or contract of any
% kind may be required for download and study of, or for building, testing and
% publishing test results for, the E2E-VIV system.}

% \paragraph{Assurance}
To meet \emph{assurance} requirements, client-side software must be functional
and free of bugs across a wide range of hardware and software stack
combinations. There must be strong security with respect to authentication such
that voter credentials cannot be forged or invalidated without breaking
underlying cryptographic protocols.

The entirety of the voting system --- e.g., software, documentation, design,
architecture, algorithms, build scripts, issue tracking system, etc. --- must be
free, open, and public. All available resources should be up to date, certified,
and released under license that permits anyone to download, build, test, or
modify the source.

% \paragraph{Maintenance and Evolvability}
To meet \emph{maintenance and evolvability requirements} election officials must
have the right and ability to update the election system to conform to law,
technology, or threat independent of the original vendor.

\subsection{Architecture}
% \todo{Improve the Architecture content.}
The study, ``The Future of Voting: End-to-End Verifiable Internet Voting,''
provides an architectural feature model, seen in Figure~\ref{fig:viv-model},
which defines over 127,000 possible architectural variants.\cite{e2e-viv}

\begin{figure}[H]
  \setstretch{1}
  \begin{Verbatim}[
    label={[Architectural Feature Model]Architectural Feature Model},
    tabsize=2,
    fontsize=\scriptsize,
    frame=lines,
    framesep=5mm,
    numbers=left,
    fillcolor=\color{yellow}
  ]
  -- This diagram shows the various dimensions of an E2EVIV architecture
  static_diagram E2EVIV_Architecture_Dimensions
  component
    class E2EVIV_ARCHITECTURE
    feature
      authority_distribution: SET[VALUE]
        ensure 0 < Result.count;
          for_all v: VALUE such_that v member_of Result
            it_holds v member_of { Centralized, Distributed };
        end
      crypto_protocols: SET[VALUE]
        ensure 0 < Result.count;
          for_all v: VALUE such_that v member_of Result
            it_holds v member_of { On_Paper, Mechanized, Verified, Generated };
        end
      correctness_evidence: SET[VALUE]
        ensure 0 < Result.count;
          for_all v: VALUE such_that v member_of Result
          it_holds v member_of { Process_Based, Assertions };
        end
      implementation_type: SET[VALUE]
        ensure 0 < Result.count;
          for_all v: VALUE such_that v member_of Result
            it_holds v member_of { Golden_Implementation, Open_Protocols_and_Specs };
        end
      key_distribution_method: SET[VALUE]
        ensure 0 < Result.count;
          for_all v: VALUE such_that v member_of Result
            it_holds v member_of { Public_Ceremony, Threshold_Cryptography, PKI, Web_of_Trust };
        end
      deployment_style: SET[VALUE]
        ensure 0 < Result.count;
          for_all v: VALUE such_that v member_of Result
            it_holds v member_of { Trusted_Servers, Public_Cloud, Peer_to_Peer };
        end
      client_technology: SET[VALUE]
        ensure 0 < Result.count;
          for_all v: VALUE such_that v member_of Result
            it_holds v member_of { Custom_App, Web_Based };
        end
    end
  end
  \end{Verbatim}
  \caption{A specification of the possible variants for an E2E-VIV system.\cite{e2e-viv}}\label{fig:viv-model}
\end{figure}

Most of the variability available when constructing an E2E-VIV system stems from
the cryptographic techniques and tools available to select from when designing
the system.

\subsubsection{Cryptographic Techniques and Tools}
The cryptographic techniques and cryptosystems available for use in E2E-V
electoral systems are presented in this section; this is not a comprehensive
list of cryptographic techniques and tools available, but includes some of the
most common techniques and tools leveraged in E2E-VIV systems.

% \paragraph{Asymmetric Cryptography}
Asymmetric cryptography, also known as public-key cryptography, uses pairs of
keys to securely encrypt and decrypt messages. There are many different
public-key based cryptosystems available for use.

% \paragraph{Homomorphic Schemes}
Homomorphic cryptographic schemes allow one to perform basic arithmetic
operations on ciphertexts without requiring decryption of the ciphertext. This
property has a number of uses in E2E-VIV systems; for example, an E2E-VIV system
might leverage this property to tally a collection of encrypted ballots without
decrypting any individual's ballots.

% \subparagraph{Additive Homomorphic Encryption}
Additive homomorphic encryption schemes enable processing of ciphertexts by way
of addition. The Pallier and Benaloh cryptosystems both support additive
homomorphic encryption.\cite{helios}

% \subparagraph{Multiplicative Homomorphic Encryption}
Multiplicative homomorphic encryption schemes enable processing ciphertexts by
way of multiplication. The ElGamal cryptosystem supports multiplicative
homomorphic encryption.

% \paragraph{Append-Only Public Bulletin Board}
Most E2E-VIV systems depend on an append-only web/public bulletin
board.\cite{bulletin-board} The append-only public bulletin board is a
publicly-visible secure location where election operations and ballot data are
submitted, logged, and made available to support auditing
requirements.\cite{e2e-viv,helios,almost-correct-mixing,pretty-good-democracy,mix-networks,vector-ballot-voting}
Blockchains fulfill most of the requirements of an append-only public bulletin
board.

% \subsubsection{Secret Sharing and Threshold Schemes}
Secret sharing and threshold schemes allow a collection of actors to cooperate
to produce ``shares'' of a secret; each participant is responsible for
managing their share of the secret and some threshold of shares must come
together to recover the complete secret or perform cryptographic
operations.\cite{distributed-key-generation,large-scale-distributed-key-generation}

% \paragraph{Zero-Knowledge Proofs}
A zero-knowledge proof (ZKP) is a probabilistic method which allows one party
to prove knowledge of some secret without revealing any information about the
secret itself. A zero-knowledge proof satisfies the following properties:\cite{e2e-viv,zkproof}
%
\begin{enumerate}
  \item Completeness, an honest verifier will be convinced by an honest prover.
  \item Soundness, an honest verifier will not be convinced by a dishonest
    prover.
  \item Zero-knowledge, the verifier will not learn any information regarding
    the secret itself.
\end{enumerate}
%
% \subparagraph{Non-Interactive Zero-Knowledge Proofs}
A useful form of zero-knowledge proofs are non-interactive zero-knowledge
proofs --- also known as NIZKs, zk-SNARKs, or zkSTARKs --- which are
zero-knowledge proofs which require no interaction between the prover and
verifier for the verifier to be convinced of
correctness.\cite{helios,almost-correct-mixing,vector-ballot-voting,mix-networks}

% \paragraph{Mixnet Schemes}
Mixnet schemes are used to provide anonymity. Mix networks are operated by a
set of trusted nodes, mix-servers, which consume messages --- typically
encrypted ballot data in the case of E2E-VIV systems --- from a set of
network-participants to produce a random permutation of the input messages.
Each node performs a ``mix'' operation on the incoming messages in such a way
that the output cannot be unscrambled and tied back to a network-participant
except by the node itself which is performing the mixing operation. Therefore,
as long as any single mix-server in the mix network is acting honestly, the
anonymity of the participants will be
maintained.\cite{mix-networks,vector-ballot-voting}

% \subparagraph{Decryption Mixnet}
A decryption mixnet operates by encrypting a message in multiple layers with
each mix server's public key. To decrypt the message, the message layers are
decrypted in the opposite order they were encrypted in by each node. Each node
forwards its decrypted results to the next node in the mix network. So long as a
single node does not reveal the source of the message then the message will
become untraceable (assuming no information is leaked by the message
itself).\cite{mix-networks}

% \subparagraph{Re-encryption Mixnet}
A re-encryption mixnet works by leveraging the re-encryption properties of the
underlying encryption scheme. Certain cryptosystems make it possible to change a
ciphertext without modifying the underlying message. In this way a set of nodes
can shuffle and re-encrypt a ciphertext then pass them to the next mixnet node
to repeat the process. So long as a single node does not reveal the shuffling
process the anonymity offered by the mixnet will be
maintained.\cite{mix-networks,almost-correct-mixing}

% \paragraph{Blind Signature Schemes}
Blind signature schemes separate the voter authentication, authorization, and
signing components from the vote tallying, shuffling, and decryption components.
A voter will encrypt a ballot (blind it) then send it to a signing authority who
will blind-sign the encrypted ballot after it has verified that the voter is
qualified to vote. Once a voter has acquired a blind signed ballot they can
strip their identifying data, unblind the ballot, and submit the signed ballot
through an anonymous channel. The underlying cryptosystem makes it such that the
blind signed ballot is equivalent to a signed unblinded ballot.\cite{e2e-viv}

% cite

% \paragraph{ElGamal}
% The ElGamal cryptosystem is a
%
% \paragraph{Paillier}

% \subsection{Existing Systems}
% \todo{Complete section.}
% \todo{Do not directly discuss any systems which are strictly physical.}

% \subsubsection{Estonia}
% Estonia began using Internet voting in 2005. By the 2015 Estonian parliamentary
% elections 30.5\% of all voters voted over the Internet. Estonia maintains what
% are probably the most advanced national identification cards in the world.
% Estonian IDs are part of a \emph{Public Key Infrastructure (PKI)} where IDs
% serve as smart cards which possess two RSA key pairs: one for signing and one
% for authentication. Cryptographic functions are performed on the card. The
% signatures produced by the IDs are used extensively throughout the country and
% are legally binding. These cryptographic IDs allow Estonia to provide voter
% authentication capabilities that cannot be reproduced in the US.\ Despite the
% advanced authentication capabilities that Estonia offers researchers in 2014
% devised a number of attacks that could be performed on the Estonian voting
% system to spoil ballots, damage ballot secrecy, and steal/drop votes. The
% researchers also criticized the transparency and operational security of the
% system.

% \subsubsection{Helios}
% \todo{Review Ben Adidas' Helios system.}

% \subsubsection{RIES}
% \todo{Complete section.}

% \subsubsection{Pret A Voter}
% \todo{Special characters/spelling in pret a voter.}

% \subsubsection{Punchscan}
% \todo{Complete section.}
%
% \subsubsection{Scantegrity II}
% \todo{Complete section.}

% \subsubsection{Remotegrity}
% \todo{Complete section.}

% \subsubsection{Norwegian System}
% \todo{Complete section.}

% \subsubsection{Wombat}
% \todo{Review Wombat.}
% \todo{Complete section.}

% \subsubsection{Demos}
% \todo{Complete section.}
% \todo{Review Demos.}

% \subsubsection{vVote}
% \todo{Complete section.}
