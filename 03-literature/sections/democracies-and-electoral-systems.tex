\section{Democracies and Electoral Systems}

% Democracy is a fundamental birthright for Americans, a cherished blessing
% which serves as the foundation for our freedom. Each generation must nourish
% and foster the processes of their democracy so that future generations may too
% reap the benefits it provides.

% FIXME: All of this seems better suited for the introduction!
Key to democracy is the right to vote. By voting the citizens of a democratic
society can express their will to a governing body. The goal of any voting
system should be to produce an outcome which fairly represents the will of the
people.

Societies have practiced democracy for thousands of years: held elections,
casted votes, and executed on the results. However, technology, and its
applications with respect to voting systems, have developed markedly in that
time. Today the Internet pervades our lives in the form of websites,
applications, peripherals and more --- used for social media, banking,
e-commerce, and everything in-between --- this leads one wonder why it is that
we haven't seen more progress in the form of Internet-based voting.
Constituents, election officials, businesses, counties, states, even entire
countries have requested and experimented with such systems. Internet security
is robust enough to support a \$100 billion industry, e-commerce, which suggests
that an Internet-based voting system is at least plausible.

Contentious election results have been seen both locally and abroad, as well as
issues that cast doubt and concern on the validity of our democratic processes:
voter suppression, gerrymandering, intimidation, ballot stuffing, destruction of
audit material, etc. Internet based voting systems might help to alleviate some
of these concerns. Research on end-to-end verifiable voting has seen much
progress in the past few years. However, Internet voting brings with it an
entirely new set of security risks and concerns that must also be addressed.

The legitimacy of a government depends on public perception. Thus, election
security is a matter of national security. Voters must be certain that their
votes have not been unfairly manipulated or tallied to modify election results,
they must feel that universal suffrage is being upheld, that all those who are
eligible to vote have had the opportunity to do so, that each voter has had
precisely one vote, that each vote is of equal weight, and that the privacy of
each voter has been maintained.

Voting is the means through which we express ourselves; if voters lose faith in
their electoral system they will lose faith in their election results, their
elected leaders, and their democracy. Trustworthy democracy is an ambitious and
worthy goal that every government should strive to realize.

% What follows are a review of various election systems.

\subsection{Voter's Rights and Legislation}
American elections are a massive and complicated undertaking filled with
federal, state, and local legislation. Interestingly, neither the Bill of Rights
nor the US Constitution originally spelled out any right to vote, active
suffrage, for the citizens of its democracy. Indeed, the United States still
does not offer universal suffrage to its citizens, e.g., felons are
disenfranchised. It wasn't until the Fifteenth Amendment was ratified that the
right to vote for citizens and the protections for citizen's right to vote were
finally recognized at a federal level. In total, we have had four separate
amendments to the Constitution which concern voting rights:

\begin{itemize}
  \item Fifteenth Amendment --- Racial suffrage (1870)
    \begin{displayquote}
      The right of citizens of the United States to vote shall not be denied or
      abridged by the United States or by any state on account of race, color,
      or previous condition of servitude.
    \end{displayquote}
    % \textquote{The right of citizens of the United States to vote shall not be denied or abridged by the United States or by any state on account of race, color, or previous condition of servitude.}

  \item Nineteenth Amendment --- Sexual suffrage (1920)
    \begin{displayquote}
      The right of citizens of the United States to vote shall not be denied or
      abridged by the United States or by any on account of sex.
    \end{displayquote}
    % \textquote{The right of citizens of the United States to vote shall not be denied or abridged by the United States or by any on account of sex.}

  \item Twenty-fourth Amendment --- Financial suffrage (1964)
    \begin{displayquote}
      The right of citizens of the United States to vote in any primary or other
      election for President or Vice President, for electors for President or
      Vice President, or for Senator or Representative in Congress, shall not be
      denied or abridged by the United States or any State by reason of failure
      to pay any poll tax or other tax.
    \end{displayquote}
    % \textquote{The right of citizens of the United States to vote in any primary or other election for President or Vice President, for electors for President or Vice President, or for Senator or Representative in Congress, shall not be denied or abridged by the United States or any State by reason of failure to pay any poll tax or other tax.}

  \item Twenty-sixth Amendment --- Age suffrage (1971)
    \begin{displayquote}
      The right of citizens of the United States, who are eighteen years of age
      or older, to vote shall not be denied or abridged by the United States or
      by any State on account of age.
    \end{displayquote}
    % \textquote{The right of citizens of the United States, who are eighteen years of age or older, to vote shall not be denied or abridged by the United States or by any State on account of age.}
\end{itemize}

There are several major pieces of legislation relevant to voter's voting rights.
Many of these statutes provide basic voting rights and others go further to
ensure that these rights are upheld for various demographics via varied
enforcement policies.

\subparagraph{Voting Rights Act}
Although the Fifteenth Amendment of the Constitution was ratified in 1870 it
wasn't until the \emph{Voting Rights Act (VRA)}, passed in 1965, that the
Fifteenth Amendment was actually enforced. The act laid out a number of
provisions used to regulate election administration. Prior to the VRA southern
states would disenfranchise racial minorities via Jim Crow laws, election fraud,
voter restrictions: literacy tests, poll taxes, property-ownership requirements,
moral character tests, requirements that voter registration applicants interpret
particular documents, etc.

\subparagraph{Uniformed and Overseas Citizens Absentee Voting Act}
The \emph{Uniformed and Overseas Citizens Absentee Voting Act (UOCAVA)} was
passed in 1986 to render services to merchant marines, uniformed services, and
other overseas civilians. Specifically, UOCAVA mandates that overseas and
military voters be able to remotely register and vote in federal elections. The
\emph{Federal Voting Assistance Program (FVAP)}, established under the
\emph{Department of Defense (DOD)}, provides voter assistance, tools, and
education to overseas voters so that they are able to vote from anywhere in the
world.

\subparagraph{National Voter Registration Act}
The \emph{National Voter Registration Act (NVRA)}, also known the
\emph{Motor Voter Act}, was federal law passed in 1993 that came into effect
in 1995. The goal of NRVA was to increase voter registration, enhance voter
participation, protect election security, and ensure states maintain accurate
voter rolls. This was done by combining voter registration with obtaining a
driver's license.

\begin{displayquote}
  The NVRA effectively forced every state to offer voter registration in
  combination with the single civic act performed almost universally by American
  adults --- obtaining a driver’s license.
\end{displayquote}

% \blockquote{The NVRA effectively forced every state to offer voter
% registration in combination with the single civic act performed almost
% universally by American adults --- obtaining a driver’s license.}

\subparagraph{Help America Vote Act}
The \emph{Help America Vote Act (HAVA)}, passed in 2002, was legislation passed to
lower barriers disabled people encountered while attempting to vote. HAVA also
aimed to improve vote auditing after a large number of ballots in the 2000
election were rejected. HAVA recommended that all election systems use
\emph{Verifiable Voter Paper Audit Trail (VVPAT)} and worked to create statewide
voter registration lists and identification requirements.

HAVA was also responsible for the formation of the United States \emph{Election
Assistance Commission (EAC)}. The EAC is responsible for testing, certifying,
and decertifying voting equipment; developing voting machine standards; and
administering funds to states so that they become HAVA compliant.

The EAC requested that the \emph{National Institute of Standards and Technology
(NIST)} create a \emph{Voluntary Voting System Guidelines (VVSG)} which cover
equipment, documentation, and testing requirements of voting machines.

\begin{displayquote}
  The purpose of the Voluntary Voting System Guidelines is to provide a set of
  specifications and requirements against which voting systems can be tested to
  determine if they provide all the basic functionality, accessibility and
  security capabilities required to ensure the integrity of voting systems. The
  VVSG specifies the functional requirements, performance characteristics,
  documentation requirements, and test evaluation criteria for the national
  certification of voting systems.
\end{displayquote}

HAVA also established provisional ballots for states that don't allow same-day
registration. Provisional ballots allow voters to cast a ballot on election day
if the voter feels they're entitled to vote but are not listed as being
registered. The ballot is counted after the voter's eligibility has been
verified with the goal being that no voter is turned away who should have
otherwise been able to vote.

\subparagraph{Military and Overseas Voter Empowerment Act}
The \emph{Military and Overseas Voter Empowerment Act (MOVE)}, passed in 2009,
amended UOCAVA and other statues to provide further protections to eligible
citizens. Specifically the act aimed to reduce the number of ballots that are
not counted due to late receipt. MOVE accomplishes this by requiring that states
send absentee ballots no later that 45 days prior to election day. MOVE goes
further by requiring that all registration material and blank ballots be
available electronically as well as removes requirements regarding notarization
on voting applications and ballots.

\subparagraph{Voter Empowerment Act}
The \emph{Voter Empowerment Act (VEA)} is meant to improve antiquated voter
registration, ensure access to ballots, preserve the integrity of the voting
system, further prohibit deceptive practices, protect voting rights, and demand
accountability from election administration.

\subsection{Voting Procedures, Technologies, and Availability}
Voting procedure varies depending on state legislation. Here we review early
voting, remote voting, and Internet voting by state. Remote voting is a form of
early voting and Internet voting a form of remote voting.

\paragraph{Early Voting}
\emph{Early voting} is a service which allows voters to cast ballots before
election day during a specified time frame at a polling location. Some states
offer early voting through \emph{in-person absentee voting}: a voter receives a
ballot through mail or internet, marks said ballot, then casts their vote at an
official polling location before Election Day.

Early voting has become a popular means of voting for American citizens. In the
1980s fewer that 5\% of ballots cast in the general election were cast before
election day, by 2000 16\% percent of votes for president were cast early, and
by 2012 the number of votes casted early had risen to at least 31\%.

% ### Arguments
Proponents of early voting argue that early voting makes the voting process more
convenient for citizens, thus increasing voter turnout. Voters would have more
flexibility to work around children, jobs, doctor's appointments, out of state
trips, as well as be able to avoid long lines on election day.

Critics argue that those concerns are less important when compared to the risks
it presents.

\begin{displayquote}
   Citizens should vote with a common base of information about candidates. If
   they vote over a period of weeks before Election Day, they vote based on the
   knowledge available on a scattering of different dates.
\end{displayquote}

\subparagraph{State Breakdown of Early Voting Availability}

% ### State Breakdown
\begin{itemize}
  \item 34 states and the District of Columbia allow any qualified voter to vote
    early without excuse or justification, see
    Table~\ref{tab:pre-election-voting}.

  \item 3 states are remote/early voting exclusively (by mail), see
    Table~\ref{tab:pre-election-voting}.
\end{itemize}

\paragraph{Remote Voting}
\emph{Remote voting}, in contrast to early voting, is any form of voting where
ballots are not marked and cast at an official polling place. Remote voting is
also known as \emph{absentee voting}. The medium through which a marked absentee
ballot is returned to election officials depends on the state.

% ### Arguments
Remote voting offers flexibility above and beyond early voting. Remote voting
allows voters to vote from the comfort of their own homes, overseas, and even
from space. However, there are risks associated with remote voting:
voter intimidation, vote buying, vote manipulation, etc.

\begin{displayquote}
   Growing use of absentee voting has turned this area of voting into the most
   likely opportunity for election fraud now encountered by law enforcement
   officials. These cases are especially difficult to prosecute, since the
   misuse of a voter’s ballot or the pressure on voters occurs away from the
   polling place or any other outside scrutiny. These opportunities for abuse
   should be contained, not enlarged.
\end{displayquote}

% ### State Breakdown
% \subparagraph{Remote Voting by State}
\subparagraph{State Breakdown of Remote Voting Availability}

\begin{itemize}
  \item 27 states and the District of Columbia permit any qualified voter to
    vote absentee, without offering an excuse, via postal service.
    Table~\ref{tab:pre-election-voting}

  \item 20 states require an excuse to remote vote.
    Table~\ref{tab:pre-election-voting}

  \item 3 states are remote/early voting exclusively (by mail).
    Table~\ref{tab:pre-election-voting}
\end{itemize}

\paragraph{Voting Procedure Availability by State}

\begin{center}
    \scriptsize
    \begin{longtabu} to \textwidth{@{} X[1.1,l] X[c] X[0.01] X[c] X[c] X[c] X[c] @{}}
        \caption{Pre-Election Day Voting}\label{tab:pre-election-voting} \\
        \toprule
        \taburowcolors{white..white}
        State                       & \multicolumn{1}{c}{In-Person}         && \multicolumn{4}{c}{By Mail} \\
                                      \cmidrule{2-2}                           \cmidrule{4-7}
                                    & Early Voting                          && No-Excuse Absentee                    & Absentee; Excuse Required             & All-Mail Voting           & Permanent Absentee Status \\
        \midrule
        \taburowcolors{white..gray!15}
        \endfirsthead%

        \toprule
        \taburowcolors{white..white}
        State                       & \multicolumn{1}{c}{In-Person}         && \multicolumn{4}{c}{By Mail} \\
                                      \cmidrule{2-2}                           \cmidrule{4-7}
                                    & Early Voting                          && No-Excuse Absentee                    & Absentee; Excuse Required             & All-Mail Voting           & Permanent Absentee Status \\
        \midrule
        \taburowcolors{white..gray!15}
        \endhead%

        Alabama                     &                                       &&                                       & \textbullet{}                         &                           & \\
        Alaska                      & \textbullet{}                         && \textbullet{}                         &                                       & (1)                       & \\
        Arizona                     & \textbullet{}                         && \textbullet{}                         &                                       & (1)                       & \textbullet{} \\
        Arkansas                    & \textbullet{}                         &&                                       & \textbullet{}                         & (1)                       & \\
        California                  & \textbullet{}                         && \textbullet{}                         &                                       & (1)                       & \textbullet{} \\
        Colorado                    &                                       &&                                       &                                       & \textbullet{}             & \\
        Connecticut                 &                                       &&                                       & \textbullet{}                         &                           & \textbullet{} \\
        Delaware                    &                                       &&                                       & \textbullet{}                         &                           & \\
        D.C.                        & \textbullet{}                         && \textbullet{}                         &                                       &                           & \textbullet{} \\
        Florida                     & \textbullet{}                         && \textbullet{}                         &                                       & (1)                       & \\
        Georgia                     & \textbullet{}                         && \textbullet{}                         &                                       &                           & \\
        Hawaii                      & \textbullet{}                         && \textbullet{}                         &                                       & (1)                       & \textbullet{} \\
        Idaho                       & (2)                                   && \textbullet{}                         &                                       & (1)                       & \\
        Illinois                    & \textbullet{}                         && \textbullet{}                         &                                       &                           & \\
        Indiana                     & (2)                                   &&                                       & \textbullet{}                         &                           & \\
        Iowa                        & (2)                                   && \textbullet{}                         &                                       &                           & \\
        Kansas                      & \textbullet{}                         && \textbullet{}                         &                                       & (1)                       & \\
        Kentucky                    &                                       &&                                       & \textbullet{}                         &                           & \\
        Louisiana                   & \textbullet{}                         &&                                       & \textbullet{}                         &                           & \\
        Maine                       & (2)                                   && \textbullet{}                         &                                       &                           & \\
        Maryland                    & \textbullet{}                         && \textbullet{}                         &                                       & (1)                       & \\
        Massachusetts               & (3)                                   &&                                       & \textbullet{}                         &                           & \\
        Michigan                    &                                       &&                                       & \textbullet{}                         &                           & \\
        Minnesota                   & (2)                                   && \textbullet{}                         &                                       & (1)                       & \textbullet{} \\
        Mississippi                 &                                       &&                                       & \textbullet{}                         &                           & \\
        Missouri                    &                                       &&                                       & \textbullet{}                         & (1)                       & \\
        Montana                     & (2)                                   && \textbullet{}                         &                                       & (1)                       & \textbullet{} \\
        Nebraska                    & \textbullet{}                         && \textbullet{}                         &                                       & (1)                       & \\
        Nevada                      & \textbullet{}                         && \textbullet{}                         &                                       & (1)                       & \\
        New Hampshire               &                                       &&                                       & \textbullet{}                         &                           & \\
        New Jersey                  & (2)                                   && \textbullet{}                         &                                       & (1)                       & \textbullet{} \\
        New Mexico                  & \textbullet{}                         && \textbullet{}                         &                                       & (1)                       & \\
        New York                    &                                       &&                                       & \textbullet{}                         &                           & \\
        North Carolina              & \textbullet{}                         && \textbullet{}                         &                                       &                           & \\
        North Dakota                & \textbullet{}                         && \textbullet{}                         &                                       & (1)                       & \\
        Ohio                        & (2)                                   && \textbullet{}                         &                                       &                           & \\
        Oklahoma                    & (2)                                   && \textbullet{}                         &                                       &                           & \\
        Oregon                      &                                       &&                                       &                                       & \textbullet{}             & \\
        Pennsylvania                &                                       &&                                       & \textbullet{}                         &                           & \\
        Rhode Island                &                                       &&                                       & \textbullet{}                         &                           & \\
        South Carolina              &                                       &&                                       & \textbullet{}                         &                           & \\
        South Dakota                & (2)                                   && \textbullet{}                         &                                       &                           & \\
        Tennessee                   & \textbullet{}                         &&                                       & \textbullet{}                         &                           & \\
        Texas                       & \textbullet{}                         &&                                       & \textbullet{}                         &                           & \\
        Utah                        & \textbullet{}                         && \textbullet{}                         &                                       &                           & \textbullet{} \\
        Vermont                     & (2)                                   && \textbullet{}                         &                                       &                           & \\
        Virginia                    &                                       &&                                       & \textbullet{}                         &                           & \\
        Washington                  &                                       &&                                       &                                       & \textbullet{}             & \\
        West Virginia               & \textbullet{}                         &&                                       & \textbullet{}                         &                           & \\
        Wisconsin                   & (2)                                   && \textbullet{}                         &                                       &                           & \\
        Wyoming                     & (2)                                   && \textbullet{}                         &                                       &                           & \\

        \taburowcolors{white..white}
        \cmidrule{1-7}

        TOTAL                       & 34 states + DC                        && 27 states + DC                        & 20 states                             & 3 states                  & 8 states + DC \\
        \bottomrule
    \end{longtabu}
    \begin{enumerate}[leftmargin=5mm,topsep=0mm]
      \item Certain elections may be held entirely by mail. The circumstances
        under which all-mail elections are permitted vary from state to state.

      \item Although these states do not have Early Voting in the traditional
        sense, within a certain period of time before an election they do allow
        a voter to apply in person for an absentee ballot (without an excuse)
        and cast that ballot in one trip to an election official’s office. This
        is often known as ``in-person absentee'' voting.

      \item Massachusetts has Early Voting only during even-year November
        elections, beginning in 2016. Currently it does not permit Early Voting
        in primaries or municipal elections.
    \end{enumerate}
\end{center}

% \textbf{Source:} National Conference of State Legislatures, January 2016.
% http://www.ncsl.org/research/elections-and-campaigns/absentee-and-early-voting.aspx


\paragraph{Electronic Voting}
\emph{Electronic voting}, also referred to as \emph{e-voting}, is the use of
electronic systems to cast and/or count votes. There are two major categories of
electronic voting worth considering:

\begin{itemize}
  \item \emph{Direct Recording Electronic (DRE) voting machines}, which provide
    mechanisms for voters to digitally mark and cast their ballot; votes which
    are cast via DRE voting machine are recorded electronically in the voting
    machine's memory for later tallying.

  \item \emph{Optical scan voting systems}, which do not provide a mechanism for
    voter's to cast a ballot, unlike DRE voting machines, but instead provide an
    electronic means of counting physical ballots which have already been cast.
    This functionality is provided via optical scanning technology, much like
    Scantrons in the academic world, and allows election officials to tally
    ballots much more quickly than possible when compared to manual
    ballot-tallying procedures.
    % thus provide initial elections results earlier, and certify election
    % results more quickly
\end{itemize}

% ### Arguments
% \subparagraph{Advantages}
Proponents of electronic voting argue that e-voting provides a faster, more
transparent, more secure, and more accurate form of voting. Other argued
benefits include greater ballot language support, increased support and
independence for handicapped people, lower costs, and the ability to entirely
prevent spoiled ballots.

% \subparagraph{Disadvantages}
Critics of electronic voting argue that the promises of e-voting have repeatedly
fallen short of expectations. Security researchers have demonstrated methods of
attacking various e-voting machines to manipulate votes casted or the tallying
process itself. As a result most states no longer offer e-voting without a
\emph{Verifiable Voter Paper Audit Trail (VVPAT)}. Proponents and vendors of
electronic voting systems have argued that their systems are rigorously tested;
however, security experts have challenged not only the rigorousness of the
available testing procedures, but also the notion that testing procedures are a
sufficient measure of a system's security.

\begin{displayquote}
   Election law in most states requires that all voting systems—whether
   electronic or not—be qualified by an authorized federally licensed laboratory
   known as an Independent Testing Authority (ITA), and then submitted to the
   state for certification. The ostensible purpose of these procedures it to
   make sure that the voting system meets the voluntary federal voting system
   standards promulgated by the FEC (and in the future, by NIST), and that they
   conform to the state’s election laws. It is tempting to place a lot of faith
   in certification procedures as a means for preventing security failures. We
   believe such faith is unwarranted. We argue that even a lengthy,
   conscientious testing and examination program by the most qualified people
   cannot give us the necessary security guarantees. In fact, in general, no
   process can, since in most cases the problem of establishing that a program
   meets any particular security requirement is known to be fundamentally
   unsolvable.
\end{displayquote}

\begin{displayquote}
   There are fundamental limits to what testing can accomplish; it is a truism
   of the software world that while testing can be used to verify that bugs and
   security vulnerabilities are present, it can never prove that they are
   absent.
\end{displayquote}

\begin{displayquote}
   Contrary to many people’s intuition, it is unlikely in the extreme that
   anyone, whether on the development team or not, would detect malicious logic
   that was deliberately disguised by a clever programmer, no matter how much
   effort was put into the search. It is much easier to hide a needle in a
   haystack than to find it.
\end{displayquote}

\paragraph{Internet Voting}
\emph{Internet voting} is defined as any form of voting where a marked ballot is
transfered over a network, this includes transfer via fax, email, or web
application (fax being included because of the widespread proliferation and
usage of Internet-based fax machines).

There have been many attempts to bring about online voting in the US and abroad.
Over \$100 million in federal funding and decades of research and development
has been spent on internet voting systems.

\begin{displayquote}
   In 2000 there were several other experiments with Internet voting in U.S.
   public elections. In some cases the votes counted officially; in others they
   did not. The largest and most well-known was the Arizona Democratic
   presidential primary, conducted by election.com (whose assets were acquired
   in 2003 by Accenture) in March of that year, in which approximately 85,000
   votes were cast and counted. The Reform Party national primary was also
   conducted over the Internet that summer, as were various nonbinding Internet
   voting experiments in some counties of Washington, California, Arizona and
   elsewhere.
\end{displayquote}

% ### Arguments
Internet voting has the potential to provide all of the benefits of early,
remote, and electronic voting, but also inherits all of their risks, and some.
The major additional risk is that Internet voting exposes the entire system to
remote attack from anywhere in the world and the possibility for large-scale
attacks.

The general consensus of research initiatives and academic research is that
Internet voting is fundamentally impossible to accomplish while maintaining
voter privacy, audit trails, and preventing large scale attacks.

\begin{displayquote}
   it is currently not possible to ensure the security, privacy, auditability
   and integrity of ballots cast over the Internet.

   \dots

   federal researchers determined that secure online voting is not currently
   feasible

   \dots

   The conclusive evidence that online voting cannot yet be done securely led
   the federal government to abandon its effort to develop a secure online
   voting system for the military in 2014.
\end{displayquote}

\begin{displayquote}
   It is reasonable to assume that the shortcomings of ITAs with respect to DREs
   will carry over to their certification of Internet voting.
\end{displayquote}

``A comment on the May 2007 DoD report on Voting Technologies for UOCAVA
Citizens'' made the following arguments against such voting systems:

\begin{itemize}
  \item Paperless (non-VVPAT) voting systems have been widely criticized: closed
    source software, insufficient security, insider attack potential, and no
    VVPAT.\

  \item Cyber attack potential: insider attacks, denial of service attacks,
    spoofing, automated vote buying, viral attacks, etc.

  \item Attacks could occur at a large-scale and launched by individual,
    corporate, or state actors that may lie outside the reach of US law. Attacks
    could result in widespread or selective voter disenfranchisement, vote
    buying/selling, vote switching, etc. These attacks are capable of being
    perpetrated without detection.

  \item These vulnerabilities cannot be eliminated without a wholesale redesign
    and replacement of both the internet and PC.\

  \item Seemingly successful Internet voting systems may appear to work
    flawlessly, promoting expansion of an insecure system.

  \item Not detecting a successful attack does not mean that one has not
    occurred.

  \item Because the threat of large-scale cyber attacks is so great, ``we
    recommend against any Internet voting until both the Internet and the
    world's home computer infrastructure have been fundamentally redesigned.''
\end{itemize}

% Sub-Section: Electoral System Design
%\subsection{Contentious Elections}
Despite having had thousands of years to improve on our election systems we
continue to see contentious election results both locally and abroad. What
follows are some of the more egregious and contentious election results.

\paragraph{American Elections}
\subparagraph{Bush vs Gore (2000 --- United States)}


\subparagraph{Trump vs Clinton (2016 --- United States)}
\begin{displayquote}
  ``A 58 percent majority of Clinton supporters say they accept Trump's
  election, while 33 percent do not. Questions about Trump's victory are
  passionate --- 27 percent of Clinton supporters feel ``strongly'' he did not
  win legitimately.''
\end{displayquote}

\paragraph{Foreign Elections}
\subparagraph{Gortari vs Cárdenas (1988 --- Mexico)}

\todo{Add back the Contentious Elections sub-section once it has been improved.}

% Sub-Section: Electoral System Design
\subsection{Design Principles}

\todo{
  This section was taken from the Methods, it should be covered in the
  Literature Review, referenced from the Methods, addressed again in the
  Discussion, and briefly mentioned in the Conclusion.
}

We borrow design principles from ``Electoral System Design: The New
International IDEA Handbook'' which were outlined in Chapters 2 and 3.

\subparagraph{Representation} We wish to achieve fair representation. What
constitutes fair representation will largely depend on the greater democratic
framework and the constitutes of that framework. Our electoral system should
translate votes into winning choices in a way that accurately and fairly
represents the will of the people while also being flexible enough to allow for
configuration and modification appropriate for various governance structures.

\subparagraph{Transparency} Our electoral system should be as transparent as
possible, preferably end-to-end verifiable. The winners, losers, and electorate
must be able to trust that the results of an election were achieved
legitimately.

\subparagraph{Inclusiveness} Our electoral system should support full suffrage
(active and passive) as well as universal suffrage. The mechanisms of the
electoral system should not be biased such that any group is discriminated
against. Designing an electoral system with inclusiveness in mind results in
governance with a stronger sense of legitimacy and wider participation and
willingness to participate by the electorate.

\paragraph{International Standards} There is no universally agreed upon
standards, but most would agree upon the following standards.

\begin{enumerate}[label=\Large$\square$]
  \item Elections should be free, fair, and periodic.
  \item Universal adult suffrage should be supported.
  \item Ballot secrecy should be preserved and constituents should be free
    from coercion.
  \item A commitment to the principle of ``one person, one vote.''
\end{enumerate}

\paragraph{Design Checklist}
We also borrow a design checklist from the ``Electoral System Design: The New
International IDEA Handbook.''

\begin{enumerate}[label=\Large$\square$]
  \item Is the system clear and comprehensible?
  \item Has context been taken into account?
  \item Is the system appropriate for the time?
  \item Are the mechanisms for future reform clear?
  \item Does the system avoid underestimating the electorate?
  \item Is the system as inclusive as possible?
  \item Was the design process perceived to be legitimate?
  \item Will the election results be seen as legitimate?
  \item Are unusual contingencies taken into account?
  \item Is the system financially and administratively sustainable?
  \item Will the voters feel motivated to participate?
  \item Is a competitive party system encouraged?
  \item Does the system fit into a holistic constitutional framework?
  \item Will the system help alleviate conflict rather than exacerbate it?
\end{enumerate}

