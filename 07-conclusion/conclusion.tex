\chapter{Conclusion}\label{chap:conclusion}

The issues faced by governments and societies --- especially while wrestling
with issues such as the COVID-19 pandemic, state-sponsored election
interference, and claims of election fraud --- demonstrate a clear need for
improvements with regards to electoral systems, voting systems, and governance
models. The concept of decentralized organizations, democratically operating and
existing on-chain, offers a provocative and alluring alternative approach which
might have the potential to support more egalitarian and meritocratic social
structures and governance models. Section~\ref{sec:elections} introduced
elections, electoral systems, and the tools available for analyzing them.
Section~\ref{sec:blockchain-technologies} introduced blockchain technologies,
their design, and the properties thereof which might prove useful in a number of
systems, such as voting, where a high degree of verifiability and confidence of
correctness is required.

The literature reviewed in Section~\ref{sec:internet-voting} demonstrates the
difficulty of building Internet-based voting systems, and reviews a number of
theoretical and real-world attacks that have been demonstrated in the various
attempts to build such systems. As seen in Section~\ref{sec:e2e-viv}, the
tensions present in the security, privacy, and authentication requirements of
voting systems are challenging to resolve; however, although still not without
risks, there is promise in Internet voting if systems can be designed which
achieve end-to-end verifiability.

Section~\ref{sec:architecture-and-design} explored several election,
authorization, and delegation designs for use on the Ethereum blockchain.
However, although Chapter~\ref{chap:results} demonstrated the possibility
implementing such systems, the issues outlined in Section~\ref{sec:analysis}
make clear that the Ethereum blockchain still requires many improvements before
it could be considered capable of supporting the kind of next-generation
elections and electoral systems this research sought to investigate.

Among other things, the semantics available for building such systems are
complex and unintuitive, an inevitable source of bugs. Further, tools and
techniques, which are currently unavailable, are necessary to support
large-scale systems such as voting; for example, inter-contract communication
and inter-block computations should be possible with consistent states available
across contract boundaries from start to end of computation.

More importantly, the computation and storage requirements demanded for
electoral system operation is too high to make their use on-chain practical in
most scenarios. Measuring generously, the least expensive vote operation listed
in Table~\ref{tab:election-simulations} would cost the equivalent of \$7.87, the
average of these vote operations (across electoral systems implementations)
measures in at \$28.18 per vote, and in the worst case a single vote operation
might cost upwards of \$57.11. The cost of executing the most-expensive tallying
operation is significantly worse, costing the equivalent of over \$1,300.00 to
tally the ballots of an election simulation with just 100 voters.\footnotemark{}
These prices are clearly far too high for serious consideration, and at best
they exclude from consideration a large number of popular electoral systems
which have much higher computation and storage demands than the ones selected.

\footnotetext{The gas price at the time of writing is approximately 120 gwei.}

These costs appear even worse when viewed in light of the fact that the
measurements made in Table~\ref{tab:election-simulations} were the product of
electoral system implementations which were specifically designed to accurately
measure the raw costs of the electoral systems' computational and storage
requirements, and therefore included no security or state management systems
beyond that. Inclusion of such systems which would certainly have made these
operations more costly. Still, these results should not be viewed in vacuum;
some estimates put the cost of election administration in the US at around
\$8.00 per voter, so the lower bounds of these results suggest that there is
potential here, especially with advancements to the Ethereum network and if some
scaling issues can be resolved.

\section*{Summary of Accomplishments}

\begin{itemize}
  \item Designed authorization components for managing voter registration and
    providing access control in a generalized way.

  \item Implemented election contracts, tests, and simulations to benchmark and
    analyze electoral system performance and cost.

  \item Identified electoral system criteria relevant to Ethereum-based
    implementations of electoral systems.

  \item Identified techniques for improving some performance characteristics
    of electoral systems.

  \item Identified features which damage performance characteristics of
    electoral systems.
\end{itemize}

\section*{Future Work}

\begin{itemize}
  \item Unless scaling issues are resolved, it is likely that more reasonable
    approaches to on-chain electoral system design would involve using dedicated
    or permissioned blockchain implementations and leveraging the more
    traditional approaches to E2E-VIV security as outlined in
    Section~\ref{sec:e2e-viv} and Section~\ref{sec:analysis}.

  \item If scaling issues are resolved then the following items seem like
    worthwhile areas of research:

    \begin{itemize}
      \item A set of hardened electoral system contracts and libraries.

      \item Electoral system contracts which provide support for multi-winner
        elections and support proportional representation.

      \item A standardized election interface. This seems difficult to produce
        given the unique inputs and outputs required for different kinds of
        elections.

      \item An visual interface for marking and casting ballots to election
        contracts.

      \item Approaches and semantics for achieving consistency with
        inter-contract computation, especially with regards to inter-block
        computation.
    \end{itemize}
\end{itemize}
