\chapter{Introduction}\label{chap:intro}

% \epigraph{These are revolutionary times. All over the globe men are revolting
% against old systems of exploitation and oppression, and out of the wombs of a
% frail world, new systems of justice and equality are being born.}{Martin Luther
% King, Jr.}

% We are forced to make decisions every day. Some decisions are simple:
%
% \begin{itemize}
%     \item ``What do I eat for dinner?''
%     \item ``What do I wear to work today?''
%     \item ``Should I go out tonight?''
% \end{itemize}
%
% Other decisions are more difficult:
%
% \begin{itemize}
%     \item ``Which politician should I vote for?''
%     \item ``Which stock should I invest in?''
%     \item ``Should I negotiate for more money on that new job offer?''
% \end{itemize}

This research investigates blockchain-based decision-making through electoral
system design and implementation. Decision-making, in its idealized form, is a
process which entails the evaluation of some problem's context and environment
in order to identify relevant criteria and variables worth considering. An
analysis of these criteria and variables can be used to identify a problem's
decision-space: the individual impact of each variable, their trade-offs, and
ultimately a set of viable choices to pick from. Weights might be applied to
the criteria --- determined by the choice-maker's personal values, beliefs, and
preferences ---  which, when considered with each of the viable choices' risks
and rewards, could be used to produce a final choice with an optimal outcome.
This decision-making process is a non-trivial procedure which becomes
increasingly difficult to consider as the consequences of the decisions become
more significant; and the analysis process harder to evaluate as the variables,
factors, goals, and the general environment of the problem becomes more complex.
For example, it is much more difficult to make decisions that have long-lasting
impacts than when the impacts are strictly immediate; likewise, decision-making
which involves complex ethical concerns, e.g., life or death, are much more
difficult to consider than those which do not.

% The entirety of the decision-making process, evaluation and analysis, would be
% based on the most reliable information available, be it personal knowledge,
% sourced from a relevant expert, or discovered if necessary.  All perspectives,
% biases, values, preferences, and even the evaluation and analysis processes
% itself would be thoroughly considered, evaluated, cogitated, and reflected on.

% This may all seem unmistakably and painfully obvious, hardly worth mentioning
% at all; dealing with the decision-making process is one that all adults are
% intimately familiar with in their day-to-day lives. However, the point to be
% understood is that the realities of the decision-making process rarely live up
% to the standards required to meet that of the idealized process.

Decision-making in reality rarely lives up to the standards required to meet
that of the idealized process; full-knowledge of an environment is rare and
constraint satisfaction is often sufficient, i.e., most decisions require only a
``good enough'' solution. Heuristics are often used to trade optimality,
completeness, accuracy, or precision for speed. Tacit knowledge or intuition is
used to fill the gaps in one's explicit knowledge or substituted for reliable
information. And even thorough decision-making processes can still produce
poor results: over-analysis can cause ``analysis paralysis,'' preventing one
from arriving at any decision at all; too much information can cause information
overload, preventing one from properly assimilating knowledge, or worse,
producing an illusion of knowledge; and still, unconscious biases, instincts, and
emotions and can subvert well-intended decision-making processes.

% It is somewhat ironic and unfortunate that there seems to be a correlation
% between the importance of a decision, its complexity, and the likelihood of
% one straying from the idealized decision-making process.
% [^Citation Needed]

% \blockquote{ The trade-off criteria for deciding whether to use a heuristic
% for solving a given problem include the following:
%
%     \begin{itemize} \item Optimality: When several solutions exist for a given
%     problem, does the heuristic guarantee that the best solution will be
%     found? Is it actually necessary to find the best solution?  \item
%     Completeness: When several solutions exist for a given problem, can the
%     heuristic find them all? Do we actually need all solutions?  Many
%     heuristics are only meant to find one solution.  \item Accuracy and
%     precision: Can the heuristic provide a confidence interval for the
%     purported solution? Is the error bar on the solution unreasonably large?
%     \item Execution time: Is this the best known heuristic for solving this
%     type of problem? Some heuristics converge faster than others. Some
%     heuristics are only marginally quicker than classic methods.
%     \end{itemize}
%
%     In some cases, it may be difficult to decide whether the solution found by
%     the heuristic is good enough, because the theory underlying heuristics is
%     not very elaborate.  }

All of the complexities regarding decision-making become more complex at scale
as we attempt to manage group-based decision-making processes: the processes
themselves become fractured as they are distributed across individuals, personal
biases manifest themselves as group biases, and the various procedures and
algorithms used to collect, accumulate, and evaluate individuals' preferences
are themselves flawed and biased in their own ways.\cite{electoral-handbook}
% All organizations and communities have processes through which decisions are
% made.
Democratic communities aim to make decisions by granting and counting
individuals' votes; and although voting may seem conceptually simple, there are
practical and theoretical complexities which are difficult to overcome. The
complexities with respect to actually implementing such systems often result in
systems which are mired with imbalances, fraud, and corruption. The
disenfranchisement of citizens is one example of this and it occurs in many
ways: lobbying, vote buying, gerrymandering, malicious ballot-box zoning, and
overly aggressive voter identification requirements to name a few. Many other
forms of voter suppression and election fraud exist which provide real-world
examples of the shortcomings a democratic-process might incur and demonstrate
the complexity of designing such decision-making
systems.\cite{steal-this-vote,electoral-handbook,confidence-electoral-process}

% In theory democracies help to provide a fairer and more balanced society,
% however,

Issues such as these in conjunction with the growing usage and near-ubiquitous
nature of the Internet and personal computers has led many to consider whether
the Internet could offer a new medium through which votes could be cast.
Proponents of Internet-based voting systems suggest that it might offer
solutions to alleviate some of the shortcomings previous democratic institutions
have fallen victim to. This has become especially relevant as governments and
societies wrestle with the COVID-19 pandemic and state-sponsored election
interference. However, online voting has been shown to be fraught with risks,
and there has been a wealth of research and events which demonstrate the
plethora of issues one would face in implementing such a system.\cite{serve-analysis,dc-voting-system}

The research with respect to large-scale Internet-based voting systems is clear:
the potential rewards of such systems do not currently outweigh the
risks.\cite{comment-on-dod-report} This research does not intend to retread this
ground, nor does it set out to advocate for such a system. However, digital
organizations and communities which exist and operate entirely ``on-chain'' are
everyday springing into existence; this has become especially true since the
advent of blockchain technologies such as Ethereum, which provides decentralized
foundations for securely executing arbitrary code.\cite{ethereum.org,DAO} Like
all organizations these decentralized organizations and communities need
decision-making processes and systems to reach group consensus. To that end,
this research explores blockchain-based electoral systems: their feasibility,
scalability, security, and potential implementations. The unique constraints and
costs stemming from the underlying blockchain systems and the nascent nature of
decentralized organizations demand focus; thus, the viability and implementation
of several electoral systems is explored. Smart contract implementations,
written in Solidity, are provided which offer insights into the potential
designs, pitfalls, and costs required to support various electoral systems and
features. This research demonstrates that secure and verifiable small-scale
voting systems can be built using ``off-the-shelf'' blockchain technologies
which support code-execution; however, this is likely only possible where
privacy, anonymity, and receipt freedom constraints are loosened. This research
further demonstrates that the costs associated with operating such voting
systems are steep, likely rendering such systems impractical in most
circumstances and therefore an inappropriate foundation for large-scale
elections. Finally, this research identifies which electoral systems and
features are viable within these blockchain environments and the associated cost
of supporting and operating these electoral systems and features.

% In particular, blockchain-based communities are considered, and therefore the
% potential for blockchain-based voting systems are considered as a means to
% address some of the issues and risks that have been shown to exist in other
% voting systems. Ultimately, could a transactional, decentralized, secure,
% verifiable, and electronic voting system exist?

Some of the questions this research initially sought to answer includes:

\begin{enumerate}
  \item What modalities of governance and electoral systems exist?

  \item What are the advantages and disadvantages of these electoral systems?

  \item What kinds of election processes and procedures, governance modalities,
    and electoral systems are appropriate within blockchain ecosystems and
    decentralized organizations?

  \item Of the election processes available, which are blockchain ecosystems
    capable of supporting?

  \item What is the current state of research with respect to digital and
   online voting?

  \item Can blockchain technologies be leveraged to improve the reliability,
    verifiability, or security of elections?
\end{enumerate}

% \begin{displayquote}
%     It might reasonably be said that the problem of establishing a modern utopia
%     is the problem of establishing a totally equitable, all-inclusive system for
%     making social and political decisions; that is, the problem of establishing
%     a voting system.
% \end{displayquote}

% 3 questions we want to answer:
%
% \begin{enumerate}
%     \item Can blockchain tech improve current voting systems? (PoA on machines)
%     \item Can we build democratic systems on-chain? (For decentralized
%         organizations)
%     \item Can we conduct chain governance on-chain?
% \end{enumerate}

\section{Overview of Materials}

%%
% TODO: Consider using \subsections{} instead of a list.
%
% \subsection{Chapter 1}
%%

This material is divided into six chapters:

\begin{enumerate}
  \item \emph{\nameref{chap:intro}}, which introduces this material, poses
    initial questions, and an offers overview of the subject matter to be
    reviewed.

  \item \emph{\nameref{chap:background}}, which introduces background materials
    regarding:

    \begin{itemize}
      % \item \emph{Governance}, decision-making, consensus mechanisms, common
      %   modalities of governance, and democratic systems.

      \item \emph{Elections and Electoral Systems}, their modalities, component
        parts, risks, an overview of some of the most common implementations
        of electoral systems, and an introduction to the tools available for
        analyzing and selecting electoral systems.

      \item \emph{Blockchain Technologies}, Bitcoin and Ethereum, their basic
        concepts and abstractions, internal data structures, algorithms,
        network architectures and topologies, the properties thereof, and
        the Ethereum Virtual Machine (EVM).
    \end{itemize}

  \item \emph{\nameref{chap:literature}}, which introduces and explores:

    \begin{itemize}
      \item \emph{Internet Voting}, its general procedures and requirements,
        reviews several novel Internet voting experiments: their architectures,
        criticisms, and outcomes.

      \item \emph{End-to-End Verifiable Voting}, its purpose, the requirements
        to support end-to-end verifiability (technical and non-functional),
        architectural options, existing systems, and common cryptographic
        techniques available for use in electoral systems.
    \end{itemize}

  \item \emph{\nameref{chap:methods}}, which covers the methodologies used
    within this work, borrowing from the tools and methodologies, introduced in
    Chapter 2's \emph{Background}, for selecting electoral systems, and from the
    technical and non-functional requirements introduced in Chapter 3's
    \emph{Literature Review} on end-to-end verifiability:

    \begin{itemize}
      \item \emph{Requirements and Design Principles}, the guiding principles
        and requirements which this research aims to support.

      \item \emph{Architecture and Implementation}, the overall design to
        support various electoral systems: registration and authorization of
        voters, voter representation and vote delegation, vote casting and
        tallying processes, and contracts for holding elections backed by
        various electoral systems: First-Past-the-Post (FPTP), Range Vote (RV),
        etc.
    \end{itemize}

  \item \emph{\nameref{chap:results}}, which reviews the results produced
    through methodologies covered in Chapter 4's \emph{Methods} in two parts:

    \begin{itemize}
      \item \emph{Test Results}, the testing methodologies and frameworks
        leveraged, an analysis of the algorithmic complexities, election
        simulations and costs, and test results of the various electoral systems
        implemented.

      \item \emph{Analysis}, which reviews the shortcomings, limitations, and
        areas of improvement required with respect to the methodologies chosen
        in Chapter 4's \emph{Methods}, offers an interpretation of the results
        produced in Chapter 5's \emph{Results}, takes a look back at the initial
        questions introduced in Chapter 1's \emph{Introduction}, offers some
        context and perspective with respect to how this material fits into the
        greater body of previous work, and offers some suggestions for future
        work.
    \end{itemize}

  \item \emph{\nameref{chap:conclusion}}, summarizes the material covered in the
    previous chapters, reviews the major results produced in this work, and
    offers closing thoughts.
\end{enumerate}
